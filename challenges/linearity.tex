\subsection{Challenge: Linearity and Behavioural Type Systems}
\label{sec:challenge:linearity-beh-types}
This challenge formalises a proof that requires reasoning about the linearity of channels.
Linearity is the notion that a channel must be used exactly once by a process.
This is necessary to prove properties about session type systems, and the key
issue of this challenge is reasoning about the linearity of context splitting operations.
Linear reasoning is also necessary to formalise, \eg, linear and affine types for the \(\pi\)-calculus and cut elimination in linear logics.

The setting for this challenge is a small calculus with a session type
system, the syntax and semantics of which are given below. The
calculus is a fragment of the one presented in~\cite{Vasconcelos2012},
formulated in the dual style of~\cite{barber96tr}.

The main objective of this challenge is to prove type preservation (also
known as subject reduction), \ie, that well-typed
processes can only transition to processes which are also well-typed
in the same context.
The second objective is to prove type safety, \ie, that well-typed processes are also well-formed in the sense that they do not use endpoints in a non-dual way.

\subsubsection{Syntax.}
The syntax is given by the grammar
\[
\begin{array}{r@{\qquad}c@{\qquad}l}
  v,w & \Coloneqq & a \quad\mid\quad l \\
   P,Q & \Coloneqq & \Pend \quad\mid\quad \Pout{x}{v}{P} \quad\mid\quad \Pin{x}{l}{P} \quad\mid\quad \PBpar{P}{Q} \quad\mid\quad  \Presd{x}{y}{P}
\end{array}
\]
where a \emph{value} \( v, w, \dots \) is either a base value \( a \) or a variable \( l \).

The output process \( \Pout{x}{v}{P} \) sends the value \( v \) via \( x \) and then continues as \( P \).
The intention is that the value \( v \) must be a base value when it is actually sent, and this is enforced in the semantics later on.
The input process \( \Pin{x}{l}{P} \) waits for a base value from \( x \) and then continues as \( P \) with the received value substituted for the variable \( l \).
The process \( \Presd{x}{y}{P} \) represents a \emph{session} with endpoints named \( x \) and \( y \) which are bound in \( P \). In \( P \), the names \( x \) and \( y \) can be used to exchange messages over the session (sending on \( x \) and receiving on \( y \) or vice versa).
Note that in this calculus channels cannot be sent in messages, therefore the topology of the communication network described by a process cannot change.
Also, there is no recursion or replication in the syntax, hence no infinite behaviours can be expressed. In particular, we only model linear (as opposed to shared) channels.

\subsubsection{Semantics.}
We describe the actions that the system can perform through a small step operational semantics.
As usual, we use  a \emph{structural congruence} relation that equates processes that we deem to be indistinguishable. Structural congruence is the smallest congruence relation that satisfies the following axioms:
\begin{mathpar}
  \inferrule[Sc-Par-Comm]{ }{\scong{\Ppar{P}{Q}}{\Ppar{Q}{P}}}
  \and
  \inferrule[Sc-Par-Assoc]{ }{\scong{\Ppar{(\Ppar{P}{Q})}{R}}{\Ppar{P}{(\Ppar{Q}{R})}}}
  \and
  \inferrule[Sc-Par-Inact]{ }{\scong{\Ppar{P}{\Pend}}{P}}
  \and
  \inferrule[Sc-Res-Par]{\{x,y\} \cap \freenames{Q} = \emptyset}{\scong{\Ppar{\Presd{x}{y}{P}}{Q}}{\Presd{x}{y}{(\Ppar{P}{Q})}}}
  \and
  \inferrule[Sc-Res-Inact]{ }{\scong{\Presd{x}{y}{\Pend}}{\Pend}}
  \and
  \inferrule[Sc-Res]{ }{\scong{\Presd{x_1}{y_1}{\Presd{x_2}{y_2}{P}}}{\Presd{x_2}{y_2}{\Presd{x_1}{y_1}{P}}}}
\end{mathpar}

The operational semantics are defined as the following relation on processes:
\begin{mathpar}
  \inferrule[R-Com]{ }{\reduces{\Presd{x}{y}{(\Ppar{\Pout{x}{a}{P}}{\Ppar{\Pin{y}{l}{Q}}{R}})}}{\Presd{x}{y}{(\Ppar{P}{\Ppar{\subst{Q}{a}{l}}{R}})}}}
  \and
  \inferrule[R-Res]{\reduces{P}{Q}}{\reduces{\Presd{x}{y}{P}}{\Presd{x}{y}{Q}}}
  \and
  \inferrule[R-Par]{\reduces{P}{Q}}{\reduces{\Ppar{P}{R}}{\Ppar{Q}{R}}}
  \and
  \inferrule[R-Struct]{\scong{P}{P'} \\ \reduces{P'}{Q'} \\ \scong{Q}{Q'}}{\reduces{P}{Q}}
\end{mathpar}

Note that reductions are allowed only for restricted pairs of session endpoints. This makes it possible to formulate subject reduction so that the typing context is exactly the same before and after each reduction.
%
Note also that due to rule \TirName{R-Com}, the process \( \Pin{y}{l}{P} \) can receive \emph{any} base value.
Since the rule \TirName{R-Com} only applies to sending base values, there is no way to send a variable or a name.

\subsubsection{Session Types.}
Our process syntax allows us to write processes that are ill formed in
the sense that they either use the endpoints bound by a restriction to
communicate in a way that does not follow the intended duality, or
attempt to send something which is not a base value.  As an example,
the process
\( \Presd{x}{y}{(\Ppar{\Pout{x}{a}{\Pend}}{\Pout{y}{a}{\Pend}})} \)
attempts to send a base value on both \( x \) and \( y\), whereas one
of the names should be used for receiving in order to guarantee
progress.  Another example is the process
\( \Presd{x}{y}{(\Ppar{\Pout{x}{l}{\Pend}}{\Pin{y}{l}{\Pend}})} \),
which attempts to send a variable that is not instantiated at the time
of sending.
%
To prevent these issues, we introduce a \emph{session type system} which
rules out ill-formed processes.

\subsubsection{Syntax.}
Our type system does not type processes directly, but instead focuses on the channels used in the process.
The syntax of \emph{session types} \( S, T \), unrestricted typing contexts \( \Gamma \) and linear typing contexts \( \Delta \) is as follows:
\[
  \begin{array}{r@{\qquad}c@{\qquad}l}
  S,T & \Coloneqq & \Tend \quad\mid\quad \Tbase \quad\mid\quad \Tin{S} \quad\mid\quad \Tout{S} \\
    \Gamma & \Coloneqq & \Cempty \quad\mid\quad \Gamma, l \\
                    \Delta &::= & \Cempty \quad\mid\quad \Cadd{\Delta}{\hastype{x}{S}}
  \end{array}
\]
The \emph{end type} \( \Tend \) describes an endpoint over which no further interaction is possible.
The \emph{base type} \( \Tbase \) describes base values.
The \emph{input type} \( \Tin{S} \) describes endpoints used for receiving a value and then according to \( S \).
The \emph{output type} \( \Tout{S} \) describes endpoints used for sending a value and then according to \( S \).

Typing contexts gather type information about names and variables.
\emph{Unrestricted} contexts are simply sets of names since we only have one
base type. \emph{Linear contexts} associate a type to endpoints. We use
the comma as split/union, overloaded to singletons, and \( \Cempty \) as the
empty context. We extend the Barendregt convention~\cite{Barendregt1984} to contexts, so that all
entries are distinct.  Note that the order in which information is added to a
type context does not matter.

Since we need to determine whether endpoints are used in complementary ways to determine whether processes are well formed, we need to formally define the dual of a type as follows:
\begin{mathpar}
  \inferrule{}{\dual{\Tin{S}} = \Tout{\dual{S}}}
  \and
  \inferrule{}{\dual{\Tout{S}} = \Tin{\dual{S}}}
  \and
  \inferrule{}{\dual{\Tend} = \Tend}
\end{mathpar}
Note that the dual function is partial since it is undefined for the base type.

\subsubsection{Typing Rules.}
Our type system is aimed at maintaining two invariants:
\begin{enumerate}
\item No endpoint is used simultaneously by parallel processes;
\item The two endpoints of the same session have dual types.
\end{enumerate}
The first invariant is maintained by linearly splitting type contexts when typing compositions of processes, the second by requiring duality when typing restrictions.

We have two typing judgments: one for values, and one for processes.
The typing rules for values are:
\begin{mathpar}
  \inferrule[T-Base]{ }{\typev{\Gamma}{\hastype{a}{\Tbase}}} \and
  \inferrule[T-Var]{ }{\typev{\Cadd{\Gamma}l }{\hastype{l}{\Tbase}}}
\end{mathpar}
The typing rules for processes are as follows:
\begin{mathpar}

  \inferrule[T-Inact]{\tend\Delta }{\types{\Gamma;\Delta}{\Pend}}
  \and
  \inferrule[T-Par]{\types{\Gamma;\Delta_1}{P} \\ \types{\Gamma;\Delta_2}{Q}}
  {\types{\Gamma; \Csplit{\Delta_1}{\Delta_2}}{\Ppar{P}{Q}}}
  \and
  \inferrule[T-Res]{\types{\Gamma; (\Cadd{\Cadd{\Delta}{\hastype{x}{T}}}{\hastype{y}{\dual{T}}}}{P})}{\types{\Gamma;\Delta}{\Presd{x}{y}{P}}}
  \and
    \inferrule[T-Out]{
      \typev{\Gamma}{\hastype{v}{\Tbase}} \\ \types{\Gamma; \Cupdate{\Delta}{\hastype{x}{T}}}{P}}{\types{\Gamma; (\Csplit{\Delta}{\hastype{x}{\Tout{T}}})}{\Pout{x}{v}{P}}}
    \and
    \inferrule[T-IN]{
      \types{(\Gamma,  l ); (\Cupdate{\Delta}{\hastype{x}{T}})}{P}}{\types{\Gamma; (\Csplit{\Delta}{\hastype{x}{\Tin{T}}})}{\Pin{x}{l}{P}}}
\end{mathpar}
Note that we do not need a judgment for typing channels, since it is already folded into the \TirName{T-In} and \TirName{T-Out} rules.

\subsubsection{Challenge.}
The objective of this challenge is to prove subject reduction and type safety for our calculus with session types. We start with some lemmata:

\begin{lemma}[Weakening]\mbox{}
  \label{lemma:weak}
  \begin{enumerate}
  \item If \( \types{\Gamma; \Delta}{P} \), then
    \( \types{(\Cadd{\Gamma}{l});\Delta}{P} \).
  \item If \( \types{\Gamma; \Delta}{P} \),
    then
    \( \types{\Gamma;(\Cadd{\Delta}{\hastype{x}{\Tend}})}{P} \).
  \end{enumerate}
\end{lemma}
\begin{proof}
  By induction on the given derivations.
\end{proof}
\begin{lemma}[Strengthening]\mbox{}
  \label{lemma:strenD}
  If \(\types{\Gamma; (\Cadd{\Delta}{\hastype{x}{T}})}{P}\) and
  $x\not\in \freenames{P}$, then \(\types{\Gamma; \Delta}{P}\).
\end{lemma}
\begin{proof}
  By induction on the derivation of
  $\types{\Gamma; (\Cadd{\Delta}{\hastype{x}{T}})}{P}$.
  \begin{description}
  \item[\TirName{T-End}] Since $\tend{\Delta}$, we can just reapply
    the rule without $\hastype{x}{T}$.

  \item[\TirName{T-Par}] In this case, we have that
    $\Cadd{\Delta}{\hastype{x}{T}} = \Csplit{\Delta_0}{\Delta_1}$.  By cases on which
    context $x$ is in, we just apply the induction hypothesis on
    that context.
  \item[\TirName{T-Res}] Without loss of generality, we assume that
    $x \notin \{y,z\}$, for $P=\Presd{y}{z}{P'}$. Since
    $\types{\Gamma;(\Cadd{\Cadd{\Cadd{\Delta}{\hastype{y}{T_0}}}{\hastype{z}{\dual{T_0}}}}{\hastype{x}{T}})}{P'}$, by
    induction hypothesis, we have
    $\types{\Gamma;(\Cadd{\Cadd{\Delta}{\hastype{y}{T_0}}}{\hastype{z}{\dual{T_0}}})}{P'}$. Applying
    again \TirName{T-Res}, we have
    $\types{\Gamma;\Delta}{\Presd{y}{z}{P'}}$.
  \end{description}
  The remaining cases are analogous.
\end{proof}

\begin{lemma}[Substitution]\mbox{}
  \label{le:subst}
  If $\types{(\Cadd{\Gamma}{l});\Delta}{P}$ and
  ${\typev{\Gamma}{\hastype{a}{\Tbase}}}$ then
  \( \types{\Gamma;\Delta}{\subst{P}{a}{l}} \).
\end{lemma}
\begin{proof}
  By induction on the derivation of $\types{(\Cadd{\Gamma}{l});\Delta}{P}$.
  \begin{description}
  \item[\TirName{T-End}] Immediate since $\tend{\Delta}$.

  \item[\TirName{T-Par}] For $P = \Ppar{P_0}{P_1}$, we apply the
    induction hypothesis on the derivations for $P_0$ and $P_1$.

  \item[\TirName{T-Res}] Immediate on the premise of the rule.

  \item[\TirName{T-Out}] Let $P = \Pout{x}{v}{P'}$. We have that
    $\typev{(\Cadd{\Gamma}{l})}{\hastype{v}{\Tbase}}$.  We must show
    $\typev{\Gamma}{\hastype{v}{\Tbase}}$ to build the conclusion with
    the induction hypothesis and \TirName{T-Out}. We proceed by cases
    on the structure of $\typev{(\Cadd{\Gamma}{l})}{\hastype{v}{\Tbase}}$.
  \end{description}
  The remaining cases are analogous.

\end{proof}

To prove that congruence preserves typing we need to spell out in more detail the former. First, we denote with $\sconga\cdot\cdot$ the relation induced by the six axioms.

\begin{lemma}[Preservation for $\stackrel{a}{\equiv}$]
  \label{le:presequiva}
  If \( \sconga{P}{Q} \), then \( \types{\Gamma;\Delta}{P} \)  iff \( \types{\Gamma;\Delta}{Q} \).
\end{lemma}
\begin{proof}
  By case analysis on the \TirName{Sc} rule applied:
  \begin{description}
  \item[\TirName {Par-Comm/Assoc}] By rearranging sub-derivations noting that
    order does not matter for linear contexts
  \item[\TirName{Par-Inact}]
   Right-to-left by strengthening~\ref{lemma:strenD}. Vice-versa,
    by picking $T$ to be $\Tend$ and weakening~\ref{lemma:weak} part 2.
  \item[\TirName{Res-Par}] By case analysis on $x : T$ being linear or $\Tend$ and applying weakening and strengthening accordingly.
  \item[\TirName{Res-Inact:}] Use weakening
  \item[\TirName{Res}] Noting that order does not matter
  \end{description}
\end{proof}

Now, following Sangiorgi and Walker~\cite{picalcbook}, we give rules for the compatible
equivalence relation induced by $\sconga\cdot\cdot$, which we still
write as $\scong\cdot\cdot$: namely, add to reflexivity, symmetry and
transitivity the following condition:
 \begin{mathpar}
   \inferrule[Cong]{\sconga P Q}{\scong {C[P]}{C[Q]}}
 \end{mathpar}

\begin{lemma}[Preservation for $\equiv$]
  \label{le:presequiv}
    If \( \scong{P}{Q} \), then \( \types{\Gamma;\Delta}{P} \)  iff \( \types{\Gamma;\Delta}{Q} \).
\end{lemma}
\begin{proof}
  By induction on the structure of the derivation of \( \scong{P}{Q} \), with an inner induction of the structure of a process context.
  \begin{description}
  \item[\TirName {Refl}] Immediate
      \item[\TirName {Sym}] By IH.
      \item[\TirName {Trans}]  By two appeal to the IH
      \item[\TirName {Cong}] By induction on the structure of $C$. If
        the context is a hole, apply Lemma~\ref{le:presequiva}. In the
        step case, apply the IH: for example if $C$ has the form
        ${\Ppar {C'} R}$, noting that ${\PBpar{ C'} R}[P]$ is equal to
        ${\Ppar {C'[P]} R}$ we have
        \( \types{\Gamma;\Delta}{{\PBpar {C'} R}[P]} \) iff
        \( \types{\Gamma;\Delta}{{\PBpar {C'} R}[Q]} \) by rule T-Par and
        the IH.
  \end{description}
\end{proof}

\begin{theorem}[Subject reduction]
  If \( \types{\Gamma;\Delta}{P} \) and \( \reduces{P}{Q} \), then \( \types{\Gamma;\Delta}{Q} \).
\end{theorem}
\begin{proof}
  By induction on the derivation of $\reduces{P}{Q}$. The cases \TirName{R-Par} and \TirName{R-Res} follow immediately by IH. Case \TirName{R-Struct} appeals twice to
  preservation of $\equiv$ (Lemma~\ref{le:presequiv}) and to the
  IH. For \TirName{R-Com}, suppose that \TirName{T-Res} introduces in $\Delta$ the
  assumptions $\Cadd{\hastype{x}{\Tout{U}}}{\hastype{y}{\Tin{\dual{U}}}}$.  Building the only derivation for the
  hypothesis, we know that $\Delta = \Delta_1, \Delta_2, \Delta_3$
  where $\types{\Gamma;\Delta_3}{R}$. We also have
  $\types{\Gamma;(\Cadd{\Delta_1}{\hastype{x}{U}})}{P}$, ${\cal D}_{2} $ a proof of
  $\types{\Gamma,l;(\Cadd{\Delta_2}{\hastype{y}{\dual{U}}})}{Q}$ and $\cal V $ a proof of
  $\typev{\Gamma}{\hastype{a}{\Tbase}}$.  From ${\cal D}_{2}$ and $\cal V$ we use the
  substitution \cref{le:subst} to obtain $\types{\Gamma;\Cadd{\Delta_2}{\hastype{y}{\dual{U}}}}{\subst{Q}{a}{l}}$. We then conclude the proof with rules \TirName{T-Par} (twice) and \TirName{T-Res}.
\end{proof}

To formulate safety, we need to formally define what we mean by
well-formed process.  We say that a process \( P \) is \emph{prefixed
  at variable \( x \)} if \( P\equiv\Pout{x}{v}{P'} \) or
\( P\equiv\Pin{x}{l}{P'} \) for some $P'$.
%
A process $P$ is then \emph{well formed} if, for every $P_1$, $P_2$,
and $R$ such that
\( P\equiv \Presd{x_1}{y_1}{\dots
  \Presd{x_n}{y_n}{(\Ppar{\Ppar{P_1}{P_2}}{R})}} \), with
\( n \geq 0 \), it holds that, if \( P_1 \) is prefixed at \( x_1 \)
and \( P_2 \) is prefixed at \( y_1 \) (or vice versa), then
\( \Ppar{P_1}{P_2}
\equiv\Ppar{\Pout{x_1}{a}{P_1'}}{\Pin{y_1}{l}{P_2'}} \), for some
$P_1'$ and $P_2'$.

Note that well-formed processes do not necessarily reduce. For example, the process
\begin{equation*}
  \Presd{x_1}{y_1}{\Presd{x_2}{y_2}{(\Ppar{\Pout{x_1}{a}{\Pin{y_2}{l}{\Pend}}}{\Pout{y_2}{x_2}{\Pin{y_1}{l}{\Pend}}})}}
\end{equation*}
is well formed but also irreducible.

\begin{theorem}[Type safety]
  If \( \types{\Gamma;\Cempty}{P} \), then \( P \) is well-formed.
\end{theorem}
\begin{proof}
  In order to prove that $P$ is well-formed, let us consider any
  process of the form
  $\Presd{x_1}{y_1}{\dots
    \Presd{x_n}{y_n}{(\Ppar{\Ppar{P_1}{P_2}}{R})}}$ that is
  structurally congruent to $P$. Clearly, by Lemma~\ref{le:presequiv},
  well-typedness must be preserved by structural congruence. Moreover,
  assume that $P_1$ is prefixed at $x_1$ and $P_2$ is prefixed at
  $y_1$ such that $\scong{P_1}{\Pout{x_1}{v}{P_1'}}$ (the opposite
  case proceeds similarly). We need to show that
  $\scong{P_2}{\Pin{y_1}{l}{P_2'}}$. This can be easily done by
  contradiction. In fact, if $\scong{P_2}{\Pout{y_1}{v}{P_2'}}$ then
  the typing rule for restriction would be violated since the type of
  $x_1$ and $y_1$ cannot be dual.
\end{proof}

\begin{corollary}
  If \( \types{\Gamma;\Cempty}{P} \) and \( \reduces{P}{Q} \), then \( Q \) is well formed.
\end{corollary}
