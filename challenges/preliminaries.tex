\subsection{Preliminaries}
\label{sec:prelim}

We list here some common notions that we will use in the following. Since the calculi under study are somewhat different, we will note in each section which will apply.

We assume the existence of some type of \emph{base values}, represented by the symbols
\( a, b, \dots \), the existence of some type
of \emph{variable names}, represented by the symbols
\( l, m, \dots \), and the existence of some type of \emph{names},
represented by the symbols \( x, y, \dots \). \ednote{AM: are we assuming that names can be compared for equality everywhere or only in C2?} % 

The syntax of processes will include:
the process \( \Pend \) or \emph{inaction}, a process which can do nothing. The process \( \Ppar{P}{Q} \) is the \emph{composition} of process \( P \) and process \( Q \).
The two components can proceed independently of each other, or they can interact via shared names.
%The process \( \Pchoice{P}{Q} \) is a non-deterministic \emph{choice} between continuing as the process \( P \) or as the process \( Q \).

For communication, processes include \emph{input} and \emph{output}, whose
signature will depend on the calculus being value or name-passing.
We use here the metavariables $c,k$ to abstract over this choice.
The process \( \Pout{x}{c}{P} \) is an \emph{output}, which can send
\( c \) via \( x \), then continue as \( P \).  % The intention is that
% the value \( v \) must be a base value when it is actually sent, and
% this will be enforced in the semantics later on
The process \( \Pin{x}{k}{P} \) is an \emph{input}, which can receive a $c$
via \( x \), then continue as \( P \) with the received element
substituted for \( k \).  The input operator thus
binds \( k \) in \( P \).

The process \( \Pres{x}{P} \) is the \emph{restriction} of the name
\( x \) to \( P \), binding \( x \) in \( P \).


The process \( \Preplicate{P} \) is the \emph{replication} of the process \( P \).
It can be thought of as the infinite composition \( \Ppar{P}{\Ppar{P}{\cdots}} \).
Replication makes it possible to express infinite behaviors.

We will use the notation \( \freenames{P} \) to denote the set of
names that occur free,
% (i.e.\ not bound by a restriction) in \( P \),
\( \boundnames{P} \) to denote the set of names that occur bound
% (by a restriction)
in \( P \) and  \( \freevars{P} \)
to denote the set of variable names that occur free
% (i.e.\ not bound by an input)
in \( P \).  We will use the notation \( \boundvars{P} \)
for the set of variable names that occur bound
% (by an input)
in \( P \).  We will use the notation \( \subst{P}{a}{l} \) to denote
the process \( P \) with base value \( a \) substituted for variable name
\( l \). Similarly, \( \subst{P}{x}{y} \) denotes the process
\( P \) with name \( x \) substituted for name \( y \).
% \begin{metanote}
% AM:  Should we spell out the definitions of subsitutions?
% \end{metanote}

Two processes \( P \) and \( Q \) are \( \alpha \)-convertible,
written \( \alphacon{P}{Q} \), if \( Q \) can be obtained from \( P \)
by a finite number of substitutions of bound variable names.  As a
convention, we will identify \( \alpha \)-convertible processes and we
assume that bound names and bound variable names of any processes are
chosen to be different from the names and variable names that occur
free in any other entities under consideration, such as processes,
substitutions, and sets of names or variable names.  This is justified
because any overlapping names and variable names may be
\( \alpha \)-converted such that the assumption is satisfied.


A \emph{context} is obtained by taking a process and replacing a single occurrence of \( \Pend \) in it with the special \emph{hole} symbol \( \ctxhole \).
As a convention, we do \emph{not} identify \( \alpha \)-convertible contexts.

We will think of contexts as functions between processes.
A context \( C \) can be \emph{applied} to a process \( P \), written \( \applyctx{C}{P} \), by replacing the hole in C by \( P \), thus obtaining another process.
The replacement should be literal, so names and variable names that are free in \( P \) can become bound in \( \applyctx{C}{P} \).

We say that an equivalence relation \( \mathcal{S} \) is a \emph{congruence} if \( (P,Q) \in \mathcal{S} \) implies that for any context \( C \), \( (\applyctx{C}{P}, \applyctx{C}{Q}) \in \mathcal{S} \).
