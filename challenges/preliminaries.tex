\subsection{Preliminaries}
\label{sec:prelim}

First, we list some common notions and conventions that we use in the challenges. Since the calculi under study are somewhat different, each section lists the changes that apply.

We assume the existence of some set of \emph{base values}, represented by the symbols
\( a, b, \dots \),  of some set
of \emph{variables}, represented by the symbols
\( l, m, \dots \), and  of some set of \emph{names},
represented by the symbols \( x, y, \dots \).%
\footnote{%
  Unlike the standard $\pi$-calculus, we distinguish variables from names to
  better control the expressiveness of the calculi under study, and the scope of
  the corresponding challenges: the key distinction is that names are used as communication channels (and can be sent and received in the scope extrusion challenge), whereas variables are only bound by inputs and cannot be restricted, sent nor received, in the style of value-passing CCS \cite{Milner1989}.%
} %
We assume that all of these sets are infinite and that their elements can be
compared for equality.

The syntax of processes includes:
the process \( \Pend \) or \emph{inaction}, a process which can do nothing. The process \( \Ppar{P}{Q} \) is the \emph{parallel composition} of process \( P \) and process \( Q \).
The two components can proceed independently of each other, or they can interact via shared names.

For communication, processes include \emph{input} and \emph{output}, whose
signature signature depends on the calculus being value-passing or name-passing.
We use here the metavariables $c,k$ to abstract over this choice
--- i.e.~$c$ may be either a value or a variable or a name, whereas $k$ may
be a variable or a name.
The process \( \Pout{x}{c}{P} \) is an \emph{output}, which can send
\( c \) via \( x \), then continue as \( P \).
The process \( \Pin{x}{k}{P} \) is an \emph{input}, which can receive a $c$
via \( x \), then continue as \( P \) with the received element
substituted for \( k \).  The input operator thus
binds \( k \) in \( P \).

The process \( \Pres{x}{P} \) is the \emph{restriction} of the name
\( x \) to \( P \), binding \( x \) in \( P \).


The process \( \Preplicate{P} \) is the \emph{replication} of the process \( P \).
It can be thought of as the infinite composition \( \Ppar{P}{\Ppar{P}{\cdots}} \).
Replication makes it possible to express infinite behaviours.

We use the notation \( \freenames{P} \) to denote the set of
names that occur free,
\( \boundnames{P} \) to denote the set of names that occur bound
in \( P \) and  \( \freevars{P} \)
to denote the set of variables that occur free
in \( P \).  We use the notation \( \boundvars{P} \)
for the set of variables that occur bound
in \( P \).  We use the notation \( \subst{P}{a}{l} \) to denote
the process \( P \) with base value \( a \) substituted for variable
\( l \). Similarly, \( \subst{P}{x}{y} \) denotes the process
\( P \) with name \( x \) substituted for name \( y \).
We use the notation \( \applysubst{\sigma}{P} \) to denote the process \( P \) with a finite number of arbitrary substitutions applied to it.

Two processes \( P \) and \( Q \) are \( \alpha \)-convertible,
written \( \alphacon{P}{Q} \), if \( Q \) can be obtained from \( P \)
by a finite number of substitutions of bound variables.  As a
convention, we identify \( \alpha \)-convertible processes and we
assume that bound names and bound variables of any processes are
chosen to be different from the names and variables that occur
free in any other entities under consideration, such as processes,
substitutions, and sets of names or variables.  This is justified
because any overlapping names and variables may be
\( \alpha \)-converted such that the assumption is satisfied.


A \emph{context} is obtained by taking a process and replacing a single occurrence of \( \Pend \) in it with the special \emph{hole} symbol \( \ctxhole \).
As a convention, we do \emph{not} identify \( \alpha \)-convertible contexts.
A context acts as a function between processes:
a context \( C \) can be \emph{applied} to a process \( P \), written \( \applyctx{C}{P} \), by replacing the hole in C by \( P \), thus obtaining another process.
The replacement is literal, so names and variables that are free in \( P \) can become bound in \( \applyctx{C}{P} \).

We say that an equivalence relation \( \mathcal{S} \) is  \emph{compatible} if \( (P,Q) \in \mathcal{S} \) implies that for any context \( C \), \( (\applyctx{C}{P}, \applyctx{C}{Q}) \in \mathcal{S} \).
