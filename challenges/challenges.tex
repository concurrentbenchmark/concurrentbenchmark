\documentclass[a4paper]{article}

\usepackage{amsmath}
\usepackage{amssymb}
\usepackage{amsthm}
\usepackage[utf8]{inputenc}
\usepackage{mathpartir}
\usepackage{bm}
\usepackage{graphicx}

\newtheorem{lemma}{Lemma}
\newtheorem{theorem}{Theorem}

\begin{document}

\section{Challenge: Scope Extrusion}
The idea of this challenge is to formalize a proof that requires scope extrusion.
Scope extrusion is the notion that a process can send restricted names to another process, as long as the restriction can safely be ``extruded'' (i.e.\ expanded) to include the recieving process.
This e.g.\ allows a process to set up a private connection by sending a restricted name to another process, then using this name for further communication.

The setting for this challenge is an untyped \( \pi \)-calculus, the syntax and semantics of which are presented below.
The objective of this challenge is to prove that barbed bisimulation for this calculus is an equivalence relation.

\section{Challenge: Coinduction}
The idea of this challenge is to formalize a proof that requires coinductive techniques.
Coinduction is a proof technique for infinite structures, which arise in this context due to systems with behaviours that continue indefinitely.
Coinduction is the dual of induction: whereas induction is useful for proving properties of least fixed points, coinduction is useful for proving properties of greatest fixed points.

The setting for this challenge is an untyped calculus of communicating systems with replication of processes, the syntax and semantics of which are presented below.
The objective of this challenge is to prove that strong barbed bisimulation for this calculus is an equivalence relation.

\section{Challenge: Linearity}
The setting for this challenge is a restricted \( \pi \)-calculus with a session type system.

The objective of this challenge is to prove type preservation.

\section{Bonus challenges}

\subsection{Linearity and Scope Extrusion}
The setting for this challenge is the \( \pi \)-calculus with a session type system.

The objective of this challenge is to prove type preservation.

\subsection{Scope Extrusion and Coinduction}
The setting for this challenge is a simply typed \( \pi \)-calculus.

The objective of this challenge is to prove bisimulation.

\subsection{Coinduction and Linearity}
The setting for this challenge is the Calculus of Communicating Systems with a session type system.

The objective of this challenge is to prove type preservation.

\section{Future work}
\begin{itemize}
\item Multiparty session types
\item Choreography
\item Encodings
\item Code extraction
\end{itemize}

\end{document}