\subsection{Unprocessed mechanisations}
\begin{itemize}
\item \url{https://doi.org/10.4230/LIPIcs.ITP.2023.28}
\item \url{https://doi.org/10.1007/978-3-030-85315-0_8}
\item \url{https://doi.org/10.4230/LIPIcs.ITP.2022.27}
\item \url{https://doi.org/10.1007/978-3-662-54434-1_34}
\item \url{https://doi.org/10.1145/3354166.3354184}
\item \url{https://doi.org/10.1017/S0960129514000231}
\item \url{https://doi.org/10.1016/j.entcs.2007.11.010}
\item \url{https://doi.org/10.1016/S1571-0661(04)80507-3}
\item \url{https://doi.org/10.1007/BFb0028392}
\item T. F. Melham. 1994. A mechanized theory of the $\pi$-calculus in HOL. Nordic J. of Computing 1, 1 (Spring 1994), 50–76. (\url{https://dl.acm.org/doi/10.5555/640186.640190}, but paper is missing)
\item Otmane Ait-Mohamed. Vérification de l'équivalence du $\pi$-calcul dans HOL. RR-2412, INRIA. 1994 (\url{https://hal.science/inria-00074263/}, in French)
\item \url{https://doi.org/10.1017/S0956796802004653}
\item \url{https://doi.org/10.1007/3-540-45315-6_24}
\item \url{https://inria.hal.science/inria-00072970}
\item \url{https://www.lix.polytechnique.fr/Labo/Dale.Miller/lProlog/examples/pi-calculus/toc.html}
\item \url{https://doi.org/10.1007/BFb0105404}
\item \url{https://doi.org/10.1007/3-540-44755-5_16}
\item \url{https://doi.org/10.1007/3-540-44929-9_30}
\item \url{https://doi.org/10.1007/10721959_33}
\item \url{https://doi.org/10.1016/S0304-3975(00)00095-5}
\item \url{https://doi.org/10.1017/S096012951700010X}
\item \url{https://doi.org/10.1016/j.entcs.2007.11.013}
\item \url{https://doi.org/10.1145/1656242.1656248}
\item \url{https://doi.org/10.1184/R1/6587294.v1}
\item \url{https://doi.org/10.6092/issn.1972-5787/4650}
\item \url{https://doi.org/10.2168/LMCS-5(2:16)2009}
\item \url{https://doi.org/10.4204/EPTCS.203.1}
\item \url{https://doi.org/10.1007/978-3-030-45237-7_17}
\item Uma Zalakain. 2019. Type-checking session-typed $\pi$-calculus with Coq. Master's thesis, University of Glasgow. \url{https://umazalakain.info/static/Z19/msc-thesis.pdf}
\item Adam Michael Petz. 2016. A Semantics for Attestation Protocols using Session Types in Coq. Master's thesis, University of Kansas. \url{http://hdl.handle.net/1808/24137}
\item \url{https://pure.itu.dk/en/publications/representing-session-types}
\item \url{https://doi.org/10.48550/arXiv.1603.03727}
\item \url{https://doi.org/10.1145/3498684}
\item \url{https://doi.org/10.4230/LIPIcs.ITP.2021.15}
\item \url{https://doi.org/10.1007/978-3-319-22102-1_19}
\item \url{https://doi.org/10.1017/S0960129513000170}
\item \url{https://doi.org/10.1007/s10817-015-9336-2}
\item \url{https://doi.org/10.1007/978-3-540-69407-6_33}
\item \url{https://doi.org/10.1016/j.entcs.2007.08.017}
\item \url{https://doi.org/10.1145/3453483.3454041}
\item \url{https://doi.org/10.4204/EPTCS.314.3}
\item \url{https://doi.org/10.7494/csci.2017.18.3.1413}
\item \url{https://doi.org/10.1145/3371074}
\item \url{https://doi.org/10.1145/3158116}
\end{itemize}

A total of 44 papers that we currently know about.

\subsection{Template}
Abstract

\subsubsection{Approach}
\begin{itemize}
\item Proof assistant
\item 
\end{itemize}

\subsubsection{Quotes about challenges encountered}


\subsubsection{Suggestions in the paper}


