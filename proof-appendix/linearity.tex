\subsection{Proofs: Linearity and behavioral type systems}
\begin{proof}[of \cref{lemma:weak}]
  By induction on the given derivations.
\end{proof}

\begin{proof}[of \cref{lemma:strenD}]
  By induction on the derivation of
%%  last applied rule of the proof of --AM changed for uniformity
  $\types{\Gamma; (\Cadd{\Delta}{\hastype{x}{T}})}{P}$.
  % , for each case inverting $\types{\Gamma; (\Delta, x : T)}{P}$ to
  % the uniquely applicable rule.
  \begin{description}
  \item[\TirName{T-End}] Since $\tend{\Delta}$, we can just reapply
    the rule without $\hastype{x}{T}$.
    % [$P = \Pend$] we Apply \TirName{T-Inact}.

  \item[\TirName{T-Par}] In this case, we have that
    $\Cadd{\Delta}{\hastype{x}{T}} = \Csplit{\Delta_0}{\Delta_1}$.  By cases on which
    context $x$ is in, we just apply the induction hypothesis on
    that context.
    %
    %
    % [$P = \Ppar{P_0}{P_1}$] By inversion on the typing we have  $ \Delta, x : T = \Delta_0 , \Delta_1$.
    %   By cases on which environment $x$ is in, apply IH on that environment.
  \item[\TirName{T-Res}] Without loss of generality, we assume that
    $x \notin \{y,z\}$, for $P=\Presd{y}{z}{P'}$. Since
    $\types{\Gamma;(\Cadd{\Cadd{\Cadd{\Delta}{\hastype{y}{T_0}}}{\hastype{z}{\dual{T_0}}}}{\hastype{x}{T}})}{P'}$, by
    induction hypothesis, we have
    $\types{\Gamma;(\Cadd{\Cadd{\Delta}{\hastype{y}{T_0}}}{\hastype{z}{\dual{T_0}}})}{P'}$. Applying
    again \TirName{T-Res}, we have
    $\types{\Gamma;\Delta}{\Presd{y}{z}{P'}}$.
    % inversion,
    % $\Delta, x : T = \Delta', y : T_0, z : \dual{T_1}, x : T $. From
    % the assumptions present we can apply IH.
    %
    % [$P = \Presd{y}{z}{P}$] We may assume $x \notin \{y,z\}$. By
    % inversion,
    % $\Delta, x : T = \Delta', y : T_0, z : \dual{T_1}, x : T $. From
    % the assumptions present we can apply IH.
  \end{description}
  The remaining cases are analogous.
\end{proof}

\begin{proof}[of \cref{le:subst}]
  By induction on the derivation of $\types{(\Cadd{\Gamma}{l});\Delta}{P}$.
  \begin{description}
  \item[\TirName{T-End}] Immediate since $\tend{\Delta}$.

  \item[\TirName{T-Par}] For $P = \Ppar{P_0}{P_1}$, we apply the
    induction hypothesis on the derivations for $P_0$ and $P_1$.

  \item[\TirName{T-Res}] Immediate on the premise of the rule.

  \item[\TirName{T-Out}] Let $P = \Pout{x}{v}{P'}$. We have that
    $\typev{(\Cadd{\Gamma}{l})}{\hastype{v}{\Tbase}}$. \\We must show
    $\typev{\Gamma}{\hastype{v}{\Tbase}}$ to build the conclusion with
    the induction hypothesis and \TirName{T-Out}. We proceed by cases
    on the structure of\\$\typev{(\Cadd{\Gamma}{l})}{\hastype{v}{\Tbase}}$.
    % \item[$P = \Pin{x}{v}{P}$] Immediate after inverting typing.
  \end{description}
  The remaining cases are analogous.
\end{proof}

\begin{proof}[of \cref{le:presequiva}]
  By case analysis on the \TirName{Sc} rule applied:
  \begin{description}
  \item[\TirName {Par-Comm/Assoc}] By rearranging sub-derivations noting that
    order does not matter for linear contexts.
  \item[\TirName{Par-Inact}]
   Right-to-left by strengthening (\cref{lemma:strenD}). Vice versa,
    by picking $T$ to be $\Tend$ and weakening (\cref{lemma:weak}(2)).
  \item[\TirName{Res-Par}] By case analysis on $\hastype{x}{T}$ being linear or $\Tend$ and applying weakening (\cref{lemma:weak}) and strengthening (\cref{lemma:strenD}) accordingly.
  \item[\TirName{Res-Inact:}] By weakening (\cref{lemma:weak}). %~\ref{lemma:weak} part 2
  \item[\TirName{Res}] By rearranging sub-derivations noting that
    order does not matter for linear contexts.
  \end{description}
\end{proof}

\begin{proof}[of \cref{le:presequiv}]
  By induction on the structure of the derivation of \( \scong{P}{Q} \), with an inner induction of the structure of a process context.
  \begin{description}
  \item[\TirName {Refl}] Immediate.
      \item[\TirName {Sym}] By the induction hypothesis.
      \item[\TirName {Trans}]  By two applications of the induction hypothesis.
      \item[\TirName {Cong}] By induction on the structure of $C$. If
        the context is a hole, apply \cref{le:presequiva}. In the
        step case, apply the induction hypothesis: for example if $C$ has the form
        ${\Ppar {C'} R}$, noting that ${\PBpar{ C'} R}[P]$ is equal to
        ${\Ppar {C'[P]} R}$ we have
        \( \types{\Gamma;\Delta}{{\PBpar {C'} R}[P]} \) iff
        \( \types{\Gamma;\Delta}{{\PBpar {C'} R}[Q]} \) by rule \TirName{T-Par} and
        the induction hypothesis.
  \end{description}
\end{proof}

\begin{proof}[of \cref{thm:subject-reduction}]
  By induction on the derivation of $\reduces{P}{Q}$. The cases \TirName{R-Par} and \TirName{R-Res} follow immediately by the induction hypothesis. Case \TirName{R-Struct} appeals twice to
  preservation of $\equiv$ (Lemma~\ref{le:presequiv}) and to the
  induction hypothesis. For \TirName{R-Com}, suppose that \TirName{T-Res} introduces in $\Delta$ the
  assumptions $\Cadd{\hastype{x}{\Tout{U}}}{\hastype{y}{\Tin{\dual{U}}}}$.  Building the only derivation for the
  hypothesis, we know that $\Delta = \Delta_1, \Delta_2, \Delta_3$
  where $\types{\Gamma;\Delta_3}{R}$. We also have
  $\types{\Gamma;(\Cadd{\Delta_1}{\hastype{x}{U}})}{P}$, ${\cal D}_{2} $ a proof of
  $\types{\Gamma,l;(\Cadd{\Delta_2}{\hastype{y}{\dual{U}}})}{Q}$ and $\cal V $ a proof of
  $\typev{\Gamma}{\hastype{a}{\Tbase}}$.  From ${\cal D}_{2}$ and $\cal V$ we use the
  substitution \cref{le:subst} to obtain $\types{\Gamma;\Cadd{\Delta_2}{\hastype{y}{\dual{U}}}}{\subst{Q}{a}{l}}$. We then conclude the proof with rules \TirName{T-Par} (twice) and \TirName{T-Res}.
\end{proof}

\begin{proof}[of \cref{thm:type-safety}]
  In order to prove that $P$ is well-formed, let us consider any
  process of the form
  $\Presd{x_1}{y_1}{\dots
    \Presd{x_n}{y_n}{(\Ppar{\Ppar{P_1}{P_2}}{R})}}$ that is
  structurally congruent to $P$. Clearly, by \cref{le:presequiv},
  well-typedness must be preserved by structural congruence. Moreover,
  assume that $P_1$ is prefixed at $x_1$ and $P_2$ is prefixed at
  $y_1$ such that $\scong{P_1}{\Pout{x_1}{v}{P_1'}}$ (the opposite
  case proceeds similarly). We need to show that
  $\scong{P_2}{\Pin{y_1}{l}{P_2'}}$. This can be easily done by
  contradiction. In fact, if $\scong{P_2}{\Pout{y_1}{v}{P_2'}}$ then
  the typing rule for restriction would be violated since the type of
  $x_1$ and $y_1$ cannot be dual.
  %
  % We proceed by induction on the definition of $\types{}{}$. The key
  % cases are that for parallel composition and that for name
  % restriction. In the case of parallel composition, linearity (and
  % context splitting) ensure that every process in parallel will have
  % different channel names. Then, the restriction rule enforces
  % duality, therefore an output will be matched by an input.
\end{proof}
