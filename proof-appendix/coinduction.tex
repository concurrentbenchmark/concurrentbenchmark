\subsection{Proofs: Coinduction and infinite processes}

\begin{proof}[of \cref{thm:bisim-equiv}]
  % We need to prove that $\sbbisim{}{}$ is reflexive, symmetric, and
  % commutative.
  We prove the three properties separately: 
  \begin{itemize}
  \item Reflexivity is straightforward: for any $P$, we need to show
    that $P\sbbisim{}{} P$. Both conditions ~\ref{eq:bisim1} and \ref{eq:bisim2} follow trivially.
  \item Symmetry follows immediately by definition.
%    since bisimilarity is defined as a {\em symmetric} relation.
  \item For transitivity, we need to prove that if $P\sbbisim{}{} Q$
    and $Q\sbbisim{}{} R$ then $P\sbbisim{}{} R$. In order to do so,
    we prove that the relation
    $\mathcal R=\{(P,R)\mid \exists Q\text{ such that }P\sbbisim{}{} Q
    \land Q\sbbisim{}{} R\}$ is a strong barbed bisimulation. Let us
    assume that $(P,R)\in\mathcal R$. Hence, there exists a $Q$ such
    that $P\sbbisim{}{} Q$ and $Q\sbbisim{}{} R$. Clearly, if
    $\observable{P}{\mu}$ then, by $P\sbbisim{}{} Q$,
    $\observable{Q}{\mu}$. And, by $Q\sbbisim{}{} R$,
    $\observable{R}{\mu}$. Moreover, if $\transition{P}{\Atau}{P'}$
    there exists $Q'$ such that $\transition{Q}{\Atau}{Q'}$ and
    $P'\sbbisim{}{}Q'$. And also, $\transition{R}{\Atau}{R'}$ with
    $Q'\sbbisim{}{}R'$. Finally, by definition of $\mathcal R$,
    $(P',R')\in\mathcal R$.
  \end{itemize}  
\end{proof}

\begin{proof}[of \cref{lemma:sbcong-largest}]
  We first prove that \( \sbcong{}{} \) is indeed a congruence, i.e.~it is an
  equivalence relation that is preserved by all contexts.  Proving that \(
  \sbcong{}{} \) is an equivalence is easy; to prove that \( \sbcong{}{} \)  is
  preserved by all contexts, we show that $\forall C: \sbcong{P}{Q}$ implies
  $\sbcong{\applyctx{C}{P}}{\applyctx{C}{Q}}$, by structural induction on the
  context $C$.

  To prove that \(\sbcong{}{}\) is the largest congruence included in
  \(\sbbisim{}{}\), we show that for any congruence $\mathcal{S} \subseteq
  {\sbbisim{}{}}$ we have $\mathcal{S} \subseteq {\sbcong{}{}}$. Take any $P,Q$
  such that $P \mathrel{\mathcal{S}} Q$ (hence, $\sbbisim{P}{Q}$): since $S$ is
  a congruence by hypothesis, this implies $\forall C: \applyctx{C}{P}
  \mathrel{\mathcal{S}} \applyctx{C}{Q}$ (hence,
  $\sbbisim{\applyctx{C}{P}}{\applyctx{C}{Q}}$). %
  Therefore, by the definition of \(\sbcong{}{}\), we have $\sbcong{P}{Q}$,
  from which we conclude $\mathcal{S} \subseteq {\sbcong{}{}}$.
\end{proof}

\begin{proof}[of \cref{lemma:up-to}, sketch]
  We check that \( \uptosbb{}{\mathcal{S}}{} \) is a strong barbed bisimulation and is thus included in \( \sbbisim{}{} \).
\end{proof}

\begin{proof}[of \cref{thm:coinduction}]
  Since \( \sbcong{}{} \) is the largest congruence included in \( \sbbisim{}{} \), it suffices to show that if \( \sbbisim{\Ppar{\applysubst{\sigma}{P}}{R}}{\Ppar{\applysubst{\sigma}{Q}}{R}} \) for any \( R \) and \( \sigma \), then \( \sbbisim{\Ppar{\applysubst{\sigma}{\applyctx{C}{P}}}{R}}{\Ppar{\applysubst{\sigma}{\applyctx{C}{Q}}}{R}} \) for any \( C \), \( R \) and \( \sigma \).
  We proceed by induction on \( C \).
  \begin{description}
  \item[\( C = \Pin{x}{z}{C'} \)] Let \( \mathcal{S} = \{ (\Ppar{\applysubst{\sigma}{\applyctx{C}{P}}}{R}, \Ppar{\applysubst{\sigma}{\applyctx{C}{Q}}}{R}) \mid \text{$R$ and $\sigma$ arbitrary} \}~\cup \sbbisim{}{} \).
    We can easily check that \( \mathcal{S} \) is a strong barbed bisimulation, noting that \( \sbbisim{}{} \) is preserved by restriction and is contained in \( \mathcal{S} \).
  \item[\( C = \Ppar{C'}{S} \)] Then by the induction hypothesis,
    \begin{equation*}
      \sbbisim{\sbbisim{\Ppar{\applysubst{\sigma}{\applyctx{C}{P}}}{R}}{\Ppar{\applysubst{\sigma}{\applyctx{C'}{P}}}{(\Ppar{\applysubst{\sigma}{S}}{R})}}}{\sbbisim{\Ppar{\applysubst{\sigma}{\applyctx{C'}{Q}}}{(\Ppar{\applysubst{\sigma}{S}}{R})}}{\Ppar{\applysubst{\sigma}{\applyctx{C}{Q}}}{R}}}
    \end{equation*}
    for any \( R \) and \( \sigma \).
  \item[\( C = \Pres{z}{C'} \)] Then by the induction hypothesis we have \( \sbbisim{\Ppar{\applysubst{\sigma}{\applyctx{C'}{P}}}{R}}{\Ppar{\applysubst{\sigma}{\applyctx{C'}{Q}}}{R}} \) for any \( R \) and \( \sigma \).
    Without loss of generality, we assume that \( z \notin \freenames{R} \cup \names{\sigma} \).
    Then, using that \( \sbbisim{}{} \) is preserved by restriction, we have
    \begin{equation*}
      \sbbisim{\sbbisim{\Ppar{\applysubst{\sigma}{\applyctx{C}{P}}}{R}}{\Pres{z}{(\Ppar{\applysubst{\sigma}{\applyctx{C'}{P}}}{R})}}}{\sbbisim{\Pres{z}{(\Ppar{\applysubst{\sigma}{\applyctx{C'}{Q}}}{R})}}{\Ppar{\applysubst{\sigma}{\applyctx{C}{Q}}}{R}}}
    \end{equation*}
  \item[\( C = \Preplicate{C'} \)] Let \( \mathcal{S} = \{ (\Ppar{\applysubst{\sigma}{\applyctx{C}{P}}}{R}, \Ppar{\applysubst{\sigma}{\applyctx{C}{Q}}}{R}) \mid \text{$R$ and $\sigma$ arbitrary} \} \).
    Using \cref{lemma:up-to}, it suffices to show that \( \mathcal{S} \) is a strong barbed bisimulation up to \( \sbbisim{}{} \).
    To this end, let
    \begin{align*}
      A &= \applysubst{\sigma}{\applyctx{C'}{P}}, & A' &= \applysubst{\sigma}{\applyctx{C}{P}} \\
      B &= \applysubst{\sigma}{\applyctx{C'}{Q}}, & B' &= \applysubst{\sigma}{\applyctx{C}{Q}} ,
    \end{align*}
    noting that \( A' = \Preplicate{A} \) and \( B' = \Preplicate{B} \).

    Suppose \( \transition{\Ppar{R}{A'}}{\Atau}{S} \) for some \( S \).
    Then we can show by a case analysis on the derivation of this transition that there exists a \( T \) such that \( \transition{\Ppar{R}{(\Ppar{A}{A})}}{\Atau}{T} \) and \( \sbbisim{S}{\Ppar{T}{A'}} \).
    Using the induction hypothesis twice, we note that \( \sbbisim{\Ppar{R}{B'}}{\Ppar{R}{\Ppar{(\Ppar{A}{A})}{B'}}} \).
    Since by rule \TirName{Par-L}, \( \transition{\Ppar{\Ppar{R}{(\Ppar{A}{A})}}{B'}}{\Atau}{\Ppar{T}{B'}} \), there must thus exist a \( U \) such that \( \transition{\Ppar{R}{B'}}{\Atau}{U} \), \( \sbbisim{U}{\Ppar{T}{B'}} \) and \( \uptosbb{S}{\mathcal{S}}{U} \) as required.
    The proof for \( \transition{\Ppar{R}{B'}}{\Atau}{S} \) is analogous.
  \end{description}
  The remaining cases are similar.
\end{proof}