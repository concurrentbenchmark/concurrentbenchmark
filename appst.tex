% \appendix
\subsection{Dual typing rules}
\renewcommand{\Csplit}[2]{#1,  #2}
\renewcommand{\Cupdate}[2]{#1, #2}

We (AM) reformulate the rules using a dual presentation, which does without the qualifiers. Hopefully, they should be equivalent. Here $\Gamma$ is unrestricted and $\Delta$ is linear. They can be seen as multisets, where comma is split/union, overloaded to singletons and $\cdot$ is the empty multiset.
\begin{mathpar}
  \inferrule[T-Base]{ }{\types{\Gamma;\cdot}{\hastype{a}{\Tbase}}} \and
  \inferrule[T-Var]{ }{\types{\Cadd{\Gamma}{\hastype{l}{\Tbase}};\cdot}{\hastype{l}{\Tbase}}}
\end{mathpar}
The only typing rule for names is:
\begin{mathpar}
  \inferrule[T-Name]{ }{\types{{\Gamma; \hastype{x}{T}}}{\hastype{x}{T}}}
\end{mathpar}
The typing rules for processes are as follows:
\begin{mathpar}

  \inferrule[T-Inact]{ }{\types{\Gamma;\cdot}{\Pend}}
  \and
  \inferrule[T-Par]{\types{\Gamma;\Delta_1}{P} \\ \types{\Gamma;\Delta_2}{Q}}
  {\types{\Gamma; \Csplit{\Delta_1}{\Delta_2}}{\Ppar{P}{Q}}}
  \and
  \inferrule[T-Res]{\types{\Gamma; \Cadd{\Cadd{\Delta}{\hastype{x}{T}}}{\hastype{y}{\dual{T}}}}{P}}{\types{\Gamma}{\Presd{x}{y}{P}}}
  \and
  \inferrule[T-Out]{%\types{\Gamma;\Delta_1}{\hastype{x}{\Tout{T}}} \\
    \types{\Gamma;\cdot}{\hastype{v}{\Tbase}} \\ \types{\Gamma; \Cupdate{\Delta}{\hastype{x}{T}}}{P}}{\types{\Gamma; \Csplit{\hastype{x}{\Tout{T}}}{{\Delta}}}{\Pout{x}{v}{P}}}
  \and
    \inferrule[T-IN]{%\types{\Gamma;\Delta_1}{\hastype{x}{\Tout{T}}} \\
    \types{\Gamma;\cdot}{\hastype{v}{\Tbase}} \\ \types{\Gamma; \Cupdate{\Delta}{\hastype{x}{T}}}{P}}{\types{\Gamma; \Csplit{\hastype{x}{\Tin{T}}}{{\Delta}}}{\Pin{x}{v}{P}}}

%   \inferrule[T-In]{}%\types{\Gamma;\Delta_1}{\hastype{x}{\Tin{U}}}
% %    \\ \types{\Gamma;\cdot}\hastype{l}{\Tbase} \qquad \Delta,\hastype{x}{U}}{P}}
%       {\types{\Gamma; \Csplit{\hastype{x}{\Tin{U}}}{\Delta}}{\Pin{x}{l}{P}}}
\end{mathpar}

%%% Local Variables:
%%% mode: latex
%%% TeX-master: "main"
%%% End:
