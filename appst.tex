% \appendix
\subsection{Dual typing rules}
\renewcommand{\Cempty}{\cdot}
\renewcommand{\Csplit}[2]{#1,  #2}
\renewcommand{\Cupdate}[2]{#1, #2}
\newcommand{\TinV}[2]{{?#1}.#2}
\newcommand{\ToutV}[2]{{!#1}.#2}
\newcommand{\tend}[1]{\mathbf{end}(#1)}

\begin{metanote}
  AM: by the typing rules, all types in $\Gamma$ are \textbf{base}, so we  just collect the variables, w/o the types. Similarly we just need to check that $v$ is well-formed w/o tracking \textbf{base}. We do not do that here now. % In fact, we may not need $\Gamma$ at all.
\end{metanote}

\[
  \begin{array}{rcl}
  S,T & ::= & \Tend \mid \Tbase \mid \Tin{S} \mid \Tout{S} \\
    \Gamma &::= & \Cempty \mid \Gamma, l \\
                    \Delta &::= & \Cempty \mid \Cadd{\Delta}{\hastype{x}{S}}
     \end{array}
\]

Here $\Gamma$ is unrestricted and $\Delta$ is linear. The former is a set, the latter a multiset, where comma is split/union, overloaded to singletons and $\cdot$ is the empty multiset.  $\tend \Phi$ denotes that all types in the range of $\Phi$ are $\Tend$.
\begin{mathpar}
  \inferrule[T-Base]{ }{\typev{\Gamma}{\hastype{a}{\Tbase}}} \and
  \inferrule[T-Var]{ }{\typev{\Cadd{\Gamma}l }{\hastype{l}{\Tbase}}}
\end{mathpar}
The typing rule for names is:
\begin{mathpar}
  \inferrule[T-Name]{ {\tend\Delta }}{\types{{\Gamma; (\Delta,\hastype{x}{T})}}{\hastype{x}{T}}}
\end{mathpar}
The typing rules for processes are as follows:
\begin{mathpar}

  \inferrule[T-Inact]{\tend\Delta }{\types{\Gamma;\Delta}{\Pend}}
  \and
  \inferrule[T-Par]{\types{\Gamma;\Delta_1}{P} \\ \types{\Gamma;\Delta_2}{Q}}
  {\types{\Gamma; \Csplit{\Delta_1}{\Delta_2}}{\Ppar{P}{Q}}}
  \and
  \inferrule[T-Res]{\types{\Gamma; (\Cadd{\Cadd{\Delta}{\hastype{x}{T}}}{\hastype{y}{\dual{T}}}}{P})}{\types{\Gamma}{\Presd{x}{y}{P}}}
  \and
    \inferrule[T-Out]{%\types{\Gamma;\Delta_1}{\hastype{x}{\Tout{T}}} \\
      \typev{\Gamma}{\hastype{v}{\Tbase}} \\ \types{\Gamma; \Cupdate{\Delta}{\hastype{x}{T}}}{P}}{\types{\Gamma; (\Csplit{\Delta}{\hastype{x}{\Tout{T}}})}{\Pout{x}{v}{P}}}
   %  \and
   % \inferrule[T-Out]{\types{\Gamma;\Delta_1}{\hastype{x}{\Tout{T}}} \\
   %   \typev{\Gamma}{\hastype{v}{\Tbase}} \\ \types{\Gamma; (\Cupdate{\Delta}{\hastype{x}{T}})}{P}}{\types{\Gamma; \Csplit{\Delta_1}{\Delta_2}}{\Pout{x}{v}{P}}}
    \and
    \inferrule[T-IN]{%\types{\Gamma;\Delta_1}{\hastype{x}{\Tout{T}}} \\
      %\typev{\Gamma}{\hastype{v}{\Tbase}} \\
      \types{(\Gamma, \hastype l \Tbase); (\Cupdate{\Delta}{\hastype{x}{T}})}{P}}{\types{\Gamma; (\Csplit{\Delta}{\hastype{x}{\Tin{T}}})}{\Pin{x}{l}{P}}}
  % \inferrule[T-IN]{\typev{\Gamma}{\hastype{v}{\Tbase}} \\
  %   \types{(\Gamma, \hastype l \Tbase); \Delta}{\types{\Gamma; \Csplit {\hastype{x}{\Tout{T}}}{\Delta}}
  %     {\Pin{x}{l}{P}}}}
% \and
% \inferrule[UN-L]
% {\types{(\Gamma,l : T); \Delta}{P\{x/ l\}}}
% {\types{\Gamma; (\Delta, x: \un{T})}{P}}
\end{mathpar}
% Note that the input and output rules rules can be simplified by unfolding the variable sub-derivation, e.g.:
% \begin{mathpar}
%   \inferrule[T-Out*]{%\types{\Gamma;\Delta_1}{\hastype{x}{\Tout{T}}} \\
%     \typev{\Gamma}{\hastype{v}{\Tbase}} \\ \types{\Gamma; \Cupdate{\Delta}{\hastype{x}{T}}}{P}}{\types{\Gamma; \Csplit{\hastype{x}{\Tout{T}}}{{\Delta}}}{\Pout{x}{v}{P}}}
% \end{mathpar}

\begin{lemma}[Weakening]\mbox{}
  \label{lemma:weak}
  \begin{enumerate}
  \item If \( \types{\Gamma; \Delta}{P} \) then
    \( \types{\Cadd{\Gamma}{\hastype{l}{T}};\Delta}{P} \).
      \item If \( \types{\Gamma; \Delta}{P} \) and \( \tend{\Delta} \) then
    \( \types{\Gamma;\Cadd{\Delta}{\hastype{x}{T}}}{P} \).
\end{enumerate}
\end{lemma}
\begin{metanote}
  Linear weakening holds only if we allow $\tend\Delta$ in T-NAME.
\end{metanote}
\begin{proof}
  By induction on the given derivations.
\end{proof}
\begin{lemma}[Strengthening]\mbox{}
  \label{lemma:strenD}
  If \( \types{\Gamma; \Delta, z : T}{P} \) and $z\not\in fv(P)$, then \( \types{\Gamma; \Delta}{P} \).
\end{lemma}
\begin{proof}
  By induction on the given derivation.
\end{proof}

\begin{lemma}[Substitution]\mbox{}
  If $\types{(\Gamma, l);\Delta}{P}$ and ${\typev{\Gamma}{\hastype{a}{\Tbase}}}$ then
  \( \types{\Gamma;\Delta}{\subst{P}{l}{a}} \).
\end{lemma}
\begin{proof}
  By induction on the first derivation.
\end{proof}
\begin{lemma}[Preservation for $\equiv$]
  If \( \types{\Gamma;\Delta}{P} \) and \( \scong{P}{Q} \) then \( \types{\Gamma;\Delta}{Q} \).
\end{lemma}
\begin{metanote}
  AM: I'm still confused about this proof, in particular Vasco's approach to consider the axioms as symmetrization of rewrite rules
\end{metanote}.
\begin{proof}
  By induction on the structure of a process context. In the base case, proceed by case analysis on the \textsl{Sc} rule applied:
  \begin{description}
  \item[Res:] use weakening \ref{lemma:weak} part 2
  \item[Comm/Assoc:] by rearranging subderivations noting that  order does not matter for linear context
  \item[Inact] Left-to-right by strengthening~\ref{lemma:strenD}. Viceversa, by weakening \ref{lemma:weak} part 2.
  \item[Res-Par] Follows Vasco's proof by case analysis on $x : T$ being linear and applying weak and stren accordingly.
  \item[Res] order does not matter
  \end{description}
  For the step case, TBD
\end{proof}
\begin{theorem}[Subject reduction]
  If \( \types{\Gamma;\Delta}{P} \) and \( \reduces{P}{Q} \) then \( \types{\Gamma;Delta}{Q} \).
\end{theorem}
\begin{proof}
  TBD
\end{proof}
% keep: AM (Vasco's)
% \newpage
% \begin{mathpar}
%   \inferrule[T-BT]{ }{\types{\Gamma}{\hastype{true}{bool}}} \and
%     \inferrule[T-BF]{ }{\types{\Gamma}{\hastype{false}{bool}}} \and
%     \inferrule[T-VarU]{ }{\types{\Cadd{\Gamma}{\hastype{x}{T}};\cdot}{\hastype{x}{T}}}
%     \and
%       \inferrule[T-VarL]{ }{\types{{\Gamma; \hastype{x}{T}}}{\hastype{x}{T}}}
% \end{mathpar}

% The typing rules for processes are as follows:
% \begin{mathpar}

%   \inferrule[T-Inact]{ }{\types{\Gamma;\cdot}{\Pend}}
%   \and
%   \inferrule[T-Par]{\types{\Gamma;\Delta_1}{P} \\ \types{\Gamma;\Delta_2}{Q}}
%   {\types{\Gamma; \Csplit{\Delta_1}{\Delta_2}}{\Ppar{P}{Q}}}
%   \and
%   \inferrule[T-Res]{\types{\Gamma; (\Cadd{\Cadd{\Delta}{\hastype{x}{T}}}{\hastype{y}{\dual{T}}}}{P})}{\types{\Gamma}{\Presd{x}{y}{P}}}
%   \and
%   \inferrule[T-Out]{\types{\Gamma;\Delta_1}{\hastype{x}{\ToutV{T}{U}}} \\
%     \typev{\Gamma}{\hastype{v}{T}} \\ \types{\Gamma; (\Cupdate{\Delta}
%       {\hastype{x}{U}})}{P}}{\types{\Gamma; \Csplit{\Delta_1}{\Delta_2}}{\Pout{x}{v}{P}}}
%   \and
%     \inferrule[T-IN]{\types{\Gamma;\Delta_1}{\hastype{x}{\ToutV{T} U}} \\
%       \typev{\Gamma}{\hastype{v}{T}} \\ \types{\Gamma; (\Cupdate{\Delta_2, \hastype y T}{\hastype{x}{U}})}{P}}{\types{\Gamma; \Csplit {\Delta_1}{\Delta_2}}{\Pin{x}{y}{P}}}
% \and
% \inferrule[UN-L]
% {\types{(\Gamma,x : T); \Delta}{P}}
% {\types{\Gamma; (\Delta, x: \un{T})}{P}}
% \end{mathpar}
% \newpage

%%% Local Variables:
%%% mode: latex
%%% TeX-master: "main"
%%% End:
