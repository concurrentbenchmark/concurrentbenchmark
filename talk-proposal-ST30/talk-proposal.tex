\documentclass[submission,copyright,creativecommons]{eptcs}
\providecommand{\event}{ST30}

\usepackage[utf8]{inputenc}
\usepackage{graphicx}
\usepackage{amsmath}
\usepackage{amssymb}
\usepackage{amsthm}
\usepackage{mathpartir}
\usepackage{bm}
\usepackage{hyperref}
\usepackage{cleveref}
\usepackage{underscore}
\usepackage[T1]{fontenc}

\newtheorem{lemma}{Lemma}
\newtheorem{theorem}{Theorem}
\newtheorem{corollary}{Corollary}

\newcommand{\Pend}{\bm{0}}
\newcommand{\Ppar}[2]{#1 \mid #2}
\newcommand{\Pres}[2]{(\bm{\nu} #1)~#2}
\newcommand{\Pout}[3]{#1 ! #2 . #3}
\newcommand{\Pin}[3]{#1 ? (#2) . #3}
\newcommand{\Pchoice}[2]{#1 + #2}
\newcommand{\Preplicate}[1]{{!}#1}

\newcommand{\freenames}[1]{\textrm{fn}(#1)}
\newcommand{\boundnames}[1]{\textrm{bn}(#1)}
\newcommand{\names}[1]{\textrm{n}(#1)}

\newcommand{\freevars}[1]{\textrm{fv}(#1)}
\newcommand{\boundvars}[1]{\textrm{bv}(#1)}
\newcommand{\vars}[1]{\textrm{vars}(#1)}

\newcommand{\alphacon}[2]{#1 =_{\alpha} #2}
\newcommand{\subst}[3]{#1\{#2/#3\}}

\newcommand{\Aoutf}[2]{#1 ! #2}
\newcommand{\Aoutb}[2]{#1 ! (#2)}
\newcommand{\Ain}[2]{#1 ? #2}
\newcommand{\Atau}{\tau}

\newcommand{\transition}[3]{#1 \xrightarrow{#2} #3}

\newcommand{\observable}[2]{#1\downarrow_{#2}}
\newcommand{\obsin}[1]{#1?}
\newcommand{\obsout}[1]{#1!}

\newcommand\sbullet[1][.5]{\mathbin{\hbox{\scalebox{#1}{$\bullet$}}}}
\newcommand{\sbbisim}[2]{#1 \overset{\sbullet}{\sim} #2}
\newcommand{\nsbbisim}[2]{#1 \not\overset{\sbullet}{\sim} #2}

\newcommand{\ctxhole}{[\cdot]}
\newcommand{\applyctx}[2]{#1[#2]}
\newcommand{\sbcong}[2]{#1 \simeq^c #2}

\newcommand{\applysubst}[2]{#2#1}

\newcommand{\Presd}[3]{(\bm{\nu} #1#2)~#3}
\newcommand{\scong}[2]{#1 \equiv #2}

\newcommand{\reduces}[2]{#1 \rightarrow #2}

\newcommand{\Tend}{\mathbf{end}}
\newcommand{\Tbase}{\mathbf{base}}
\newcommand{\Tin}[1]{{?}.#1}
\newcommand{\Tout}[1]{{!}.#1}

\newcommand{\hastype}[2]{#1 : #2}

\newcommand{\un}[1]{\mathbf{un}(#1)}
\newcommand{\lin}[1]{\mathbf{lin}(#1)}

\newcommand{\Cempty}{\varnothing}
\newcommand{\Cadd}[2]{#1, #2}
\newcommand{\Csplit}[2]{#1 \circ #2}
\newcommand{\Cupdate}[2]{#1 + #2}

\newcommand{\dual}[1]{\overline{#1}}

\newcommand{\types}[2]{#1 \vdash #2}

%% Local Variables:
%% mode: latex
%% TeX-master: "main"
%% End:


\title{The Concurrent Calculi Formalisation Benchmark\\{\normalsize(Talk Proposal)}}
\def\titlerunning{The Concurrent Calculi Formalisation Benchmark (Talk Proposal)}

\author{
     Marco Carbone \institute{IT University of Copenhagen} \email{maca@itu.dk}
\and David Castro-Perez \institute{University of Kent} \email{D.Castro-Perez@kent.ac.uk}
\and Francisco Ferreira \institute{Royal Holloway, University of London} \email{Francisco.FerreiraRuiz@rhul.ac.uk}    \and Lorenzo Gheri \institute{University of Oxford} \email{lorenzo.gheri@cs.ox.ac.uk}
\and Frederik Krogsdal Jacobsen \institute{Technical University of Denmark \thanks{The work was done while visiting the University of Oxford}} \email{fkjac@dtu.dk}
\and Alberto Momigliano \institute{Università degli Studi di Milano} \email{momigliano@di.unimi.it}
\and Luca Padovani \institute{Università di Camerino} \email{luca.padovani@unicam.it}
\and Alceste Scalas \institute{Technical University of Denmark} \email{alcsc@dtu.dk}
\and Martin Vassor \institute{University of Oxford} \email{martin.vassor@cs.ox.ac.uk}
\and Nobuko Yoshida \institute{University of Oxford} \email{nobuko.yoshida@cs.ox.ac.uk}
}
\def\authorrunning{Carbone, Castro-Perez, Ferreira, Gheri, Jacobsen, Momigliano, Padovani, Scalas, Vassor \& Yoshida}

\hypersetup{
  bookmarksnumbered,
  pdftitle    = {\titlerunning},
  pdfauthor   = {\authorrunning},
  pdfsubject  = {ST30},
  pdfkeywords = {mechanisation, process calculi, coinduction, scope extrusion, linearity}
}

\begin{document}
\maketitle

\begin{abstract}
  The POPLMark challenge sparked a flurry of work on machine-checked proofs.
  However, POPLMark (and its follow-up POPLMark Reloaded) were explicitly limited in scope,
  with benchmark problems that do not address concurrency. For this reason, we propose a new collection of benchmark problems with a focus on the issues encountered when mechanising message-passing concurrency using process calculi.
  Our benchmark problems concern linear handling of environments, scope extrusion, and coinductive reasoning.
  Our goal is to clarify, compare and advance the state of the art.
\end{abstract}

\section{Introduction}
The programming language community has increasingly embraced mechanisations since the influential POPLMark challenge~\cite{POPLMark}, but we are still far from the vision of a mechanised proof accompanying every paper in the field.
Since POPLMark, the number of papers with accompanying mechanised proofs has grown steadily, but many mechanisations still rely on ad hoc techniques.
POPLMark Reloaded~\cite{POPLMarkReloaded} aimed to identify and develop best practices and tool support for proofs using logical relations.
The authors of both POPLMark and POPLMark Reloaded were careful to note that their benchmarks were only a beginning, and they specifically mention reasoning about concurrency using coinduction and linear environments as points for future work.
Papers in concurrency theory do often include mechanised proofs (e.g.~\cite{DBLP:conf/pldi/Castro-Perez0GY21,DBLP:conf/tacas/CastroFY20,lmcs:9985, DBLP:journals/jar/CruzFilipeMP23, Tirore:2023}), and others also directly discuss formalisation techniques (e.g.~\cite{DBLP:journals/jar/BengtsonPW16, DBLP:conf/tphol/Gay01, DBLP:conf/ppdp/Thiemann19, DBLP:conf/forte/ZalakainD21}). However, our experience is that choosing techniques and tools remains a challenge with few guidelines.
We propose a new collection of benchmark problems specifically designed to tackle the issues encountered when mechanising message-passing concurrency using process calculi.
A goal of our collection of benchmark problems is to push for the development of guidelines and tutorials, and our challenge problems are a simple setting in which to demonstrate and compare techniques.
Another goal is to strengthen the culture of mechanisation by rallying the community to collaborate on exploring and developing new tools and techniques.
We have identified three key aspects which cause issues when mechanising concurrency theory: linearity, scope extrusion in name passing systems and infinite behaviour modelled coinductively.
Each part of our benchmark exercises the three aspects independently to enable comparison and to keep the challenge problems small.
The three aspects are of course not the only ones that may cause issues in mechanisations, but they are fundamental to the field and will thus appear in most mechanisations of concurrency theory.
Actual mechanisations will additionally often require more than just one aspect, but we do not address the combination of techniques.
In the rest of this paper we discuss the design of the challenge and briefly describe the three challenge problems.

\section{Design of the challenge}
We have identified three key aspects fundamental to mechanisations of concurrent calculi and designed our challenge problems to exercise these.
Like previous benchmark authors, we would like to stress that our challenge is not meant to be comprehensive: applications such as multiparty session types, choreographies, conversation types and psi-calculi are not directly covered by our challenge problems.
Most applications will however still need the basic techniques that our challenge problems exercise.
We have designed our challenge problems to exercise the three aspects independently, and they can be solved individually and in any order.
We have designed each problem to be small and easy to understand with basic knowledge of textbook concurrency theory, process calculi and type theory.
Solutions to our challenge problems should be compared on three measures: the mechanisation overhead, this being the amount of manually written infrastructure and setup needed to express the definitions in the mechanisation; the adequacy of the formal statements in the mechanisation, i.e.\ whether the proven theorems are easily recognisable as the theorems from the challenge; and the cost of entry for the tools and techniques employed, i.e.\ the difficulty of learning to use the techniques.
Solutions to our challenges do not need to strictly follow the definitions and lemmas set out in the problem text, but solutions which deviate from the original challenges will need to present more elaborate argumentation for their adequacy.

\section{The challenge problems}
In this section we will briefly describe our three challenge problems, referring to the appendix for the full text of each challenge.
Our \textbf{linearity challenge} (Appendix~\ref{sec:challenge:linearity-beh-types}) is to prove type preservation (also known as subject reduction) for a process calculus with a session type system.
Session type systems require linear handling of typing contexts, and the proof involves reasoning about the linearity of context splitting operations.
In name passing systems, scope extrusion is the expansion of the scope of a name when the name is sent outside of its original scope.
Our \textbf{scope extrusion challenge} (Appendix~\ref{sec:challenge:name-passing-scope-extrusion}) is to prove that barbed bisimulation is an equivalence relation for an untyped name passing calculus without infinite behaviour, but with non-deterministic choice.
The proof involves reasoning about names that are ``in the process'' of being scope extruded, which presents issues for some approaches to mechanising binders.
Our \textbf{coinduction challenge} (Appendix~\ref{sec:challenge:coinduction}) is to prove that strong barbed bisimulation is an equivalence relation for an untyped calculus of communicating systems with (infinite) replication of processes.
The proof involves reasoning coinductively about the infinite behaviours of the replication operator.
We invite anyone interested to send us solutions to the challenge problems and to publish experience reports detailing the advantages and disadvantages of their approaches.
The interested reader can find more information about the current status of the challenge at:
\begin{center}
  \url{https://concurrentbenchmark.github.io/}
\end{center}

\bibliographystyle{eptcs}
\bibliography{../references}

\appendix

\section{Challenges}
\documentclass[a4paper]{article}

\usepackage{amsmath}
\usepackage{amssymb}
\usepackage{amsthm}
\usepackage[utf8]{inputenc}
\usepackage{mathpartir}
\usepackage{bm}
\usepackage{graphicx}

\newtheorem{lemma}{Lemma}
\newtheorem{theorem}{Theorem}

\newcommand{\Pend}{\bm{0}}
\newcommand{\Ppar}[2]{#1 \mid #2}
\newcommand{\Pres}[2]{(\bm{\nu} #1)~#2}
\newcommand{\Pout}[3]{#1 ! #2 . #3}
\newcommand{\Pin}[3]{#1 ? (#2) . #3}
\newcommand{\Pchoice}[2]{#1 + #2}
\newcommand{\Preplicate}[1]{{!}#1}

\newcommand{\freenames}[1]{\textrm{fn}(#1)}
\newcommand{\boundnames}[1]{\textrm{bn}(#1)}
\newcommand{\names}[1]{\textrm{n}(#1)}

\newcommand{\alphacon}[2]{#1 =_{\alpha} #2}
\newcommand{\subst}[3]{#1\{#2/#3\}}

\newcommand{\Aoutf}[2]{#1 ! #2}
\newcommand{\Aoutb}[2]{#1 ! (#2)}
\newcommand{\Ain}[2]{#1 ? #2}
\newcommand{\Atau}{\tau}

\newcommand{\transition}[3]{#1 \xrightarrow{#2} #3}

\newcommand{\observable}[2]{#1\downarrow_{#2}}
\newcommand{\obsin}[1]{#1?}
\newcommand{\obsout}[1]{#1!}

\newcommand\sbullet[1][.5]{\mathbin{\hbox{\scalebox{#1}{$\bullet$}}}}
\newcommand{\sbbisim}[2]{#1 \overset{\sbullet}{\sim} #2}

\begin{document}

\section{Challenge: Scope Extrusion}
The idea of this challenge is to formalize a proof that requires scope extrusion.
Scope extrusion is the notion that a process can send restricted names to another process, as long as the restriction can safely be ``extruded'' (i.e.\ expanded) to include the recieving process.
This e.g.\ allows a process to set up a private connection by sending a restricted name to another process, then using this name for further communication.

The setting for this challenge is an untyped \( \pi \)-calculus, the syntax and semantics of which are presented below.
The objective of this challenge is to prove that barbed bisimulation for this calculus is an equivalence relation.

\subsection{Syntax}
We assume the existence of some type of \emph{names}, values of which we will denote by \( x, y, \dots \).
The syntax is then:
\begin{align*}
  P,Q :=&&& \Pend \\
  |&&& \Pout{x}{y}{P} \\
  |&&& \Pin{x}{y}{P} \\
  |&&& \Ppar{P}{Q} \\
  |&&& \Pres{x}{P} \\
  |&&& \Pchoice{P}{Q}
\end{align*}
The process \( \Pend \) is \emph{inaction}: a process which can do nothing.
The process \( \Pout{x}{y}{P} \) is an \emph{output}, which can send the name \( y \) via \( x \), then continue as \( P \).
The process \( \Pin{x}{y}{P} \) is an \emph{input}, which can receive a name via \( x \), then continue as \( P \) with the received name substituted for \( y \).
The input operator thus binds the name \( y \) in \( P \).
The process \( \Ppar{P}{Q} \) is the \emph{composition} of process \( P \) and process \( Q \).
The two components can proceed independently of each other, or they can interact via shared names.
The process \( \Pres{x}{P} \) is the \emph{restriction} of the name \( x \) to \( P \).
Components in \( P \) can use the name \( x \) to interact with each other, but not with processes outside of the restriction.
The restriction operator thus binds the name \( x \) in \( P \).
Note that the scope of a restriction may change when processes interact.
The process \( \Pchoice{P}{Q} \) is a non-deterministic \emph{choice} between continuing as the process \( P \) or as the process \( Q \).
Note that there is no recursion or replication in the syntax, and thus no infinite behaviours can be expressed.

We will use the notation \( \freenames{P} \) to denote the set of names that occur free (i.e.\ not bound by a restriction or an input) in \( P \).
We will use the notation \( \boundnames{P} \) to denote the set of names that occur bound (by a restriction or an input) in \( P \).
We will use the notation \( \subst{P}{x}{y} \) to denote the process \( P \) with \( x \) substituted for \( y \).

Two processes \( P \) and \( Q \) are \( \alpha \)-convertible, written \( \alphacon{P}{Q} \), if \( Q \) can be obtained from \( P \) by a finite number of substitutions of bound names.
As a convention, we will identify \( \alpha \)-convertible processes.

As a convention, we assume that the bound names occurring in any collection of processes are chosen to be different from the free names occurring in those processes and from the names occurring in any substitutions applied to the processes.
This is justified because any overlapping names may be \( \alpha \)-converted such that the assumption is satisfied.

\subsection{Semantics}
The semantics of the system describe the actions that the system can perform by defining a labelled transition relation on processes.
The transitions are labelled by \emph{actions}, the syntax of which are as follows:
\begin{align*}
  \alpha := &&& \Aoutf{x}{y} \\
  |&&& \Ain{x}{y} \\
  |&&& \Aoutb{x}{y} \\
  |&&& \Atau
\end{align*}
The \emph{free output action} \( \Aoutf{x}{y} \) is sending the name \( y \) via \( x \).
The \emph{input action} \( \Ain{x}{y} \) is receiving the name \( y \) via \( x \).
The \emph{bound output action} \( \Aoutb{x}{y} \) is sending a fresh name \( y \) via \( x \).
The \emph{internal action} \( \Atau \) is performing some unobservable action, e.g.\ internal communication.

We will again use the notation \( \freenames{\alpha} \) to denote the set of names that occur free in the action \( \alpha \) and the notation \( \boundnames{\alpha} \) to denote the set of names that occur bound in the action \( \alpha \).
In the free output action \( \Aoutf{x}{y} \) and the input action \( \Ain{x}{y} \), both \( x \) and \( y \) are free names.
In the bound output action \( \Aoutb{x}{y} \), \( x \) is a free name, while \( y \) is a bound name.
We will further use the notation \( \names{\alpha} \) to denote the union of \( \freenames{\alpha} \) and \( \boundnames{\alpha} \), i.e.\ the set of all names that occur in the action \( \alpha \).

The transition relation is then defined by the following rules:
\begin{mathpar}
  \inferrule[Out]{ }{\transition{\Pout{x}{y}{P}}{\Aoutf{x}{y}}{P}}
  \and
  \inferrule[In]{ }{\transition{\Pin{x}{z}{P}}{\Ain{x}{y}}{\subst{P}{y}{z}}}
  \and
  \inferrule[Sum-L]{\transition{P}{\alpha}{P'}}{\transition{\Pchoice{P}{Q}}{\alpha}{P'}}
  \and
  \inferrule[Sum-R]{\transition{P}{\alpha}{P'}}{\transition{\Pchoice{Q}{P}}{\alpha}{P'}}
  \and
  \inferrule[Par-L]{\transition{P}{\alpha}{P'} \\ \boundnames{\alpha} \cap \freenames{Q} = \emptyset}{\transition{\Ppar{P}{Q}}{\alpha}{\Ppar{P'}{Q}}}
  \and
  \inferrule[Par-R]{\transition{Q}{\alpha}{Q'} \\ \boundnames{\alpha} \cap \freenames{P} = \emptyset}{\transition{\Ppar{P}{Q}}{\alpha}{\Ppar{P}{Q'}}}
  \and
  \inferrule[Comm-L]{\transition{P}{\Aoutf{x}{y}}{P'} \\ \transition{Q}{\Ain{x}{y}}{Q'}}{\transition{\Ppar{P}{Q}}{\Atau}{\Ppar{P'}{Q'}}}
  \and
  \inferrule[Comm-R]{\transition{P}{\Ain{x}{y}}{P'} \\ \transition{Q}{\Aoutf{x}{y}}{Q'}}{\transition{\Ppar{P}{Q}}{\Atau}{\Ppar{P'}{Q'}}}
  \and
  \inferrule[Close-L]{\transition{P}{\Aoutb{x}{z}}{P'} \\ \transition{Q}{\Ain{x}{z}}{Q'} \\ z \notin \freenames{Q}}{\transition{\Ppar{P}{Q}}{\tau}{\Pres{z}{\Ppar{P'}{Q'}}}}
  \and
  \inferrule[Open]{\transition{P}{\Aoutf{x}{z}}{P'} \\ z \neq x}{\transition{\Pres{z}{P}}{\Aoutb{x}{z}}{P'}}
  \and
  \inferrule[Close-R]{\transition{P}{\Ain{x}{z}}{P'} \\ \transition{Q}{\Aoutb{x}{z}}{Q'} \\ z \notin \freenames{P}}{\transition{\Ppar{P}{Q}}{\tau}{\Pres{z}{\Ppar{P'}{Q'}}}}
  \and
  \inferrule[Res]{\transition{P}{\alpha}{P'} \\ z \notin \names{\alpha}}{\transition{\Pres{z}{P}}{\alpha}{\Pres{z}{P'}}}
\end{mathpar}
Note that there is no rule for inferring transitions from \( \Pend \), and that there is no rule for inferring an action of an input or output process except those that match the input/output capability.
Note also that due to rule \TirName{In}, the process \( \Pin{x}{z}{P} \) can receive \emph{any} name.

As a convention, we assume that bound names of any processes or actions are chosen to be different from the names that occur free in any other entities under consideration, such as processes, actions, substitutions, and sets of names.
The convention has one exception, namely that in the transition \( \transition{P}{\Aoutb{x}{z}}{Q} \), the name \( z \) (which occurs bound in \( P \) and the action \( \Aoutb{x}{z} \)) may occur free in \( Q \).
Without this exception it would be impossible to express scope extrusion.

\subsection{Bisimilarity}
Our notion of process equivalence relations builds on a notion of \emph{observables}.
If we allow ourselves only to observe internal transitions, we will relate either too few processes (in the strong case where we relate only processes with exactly the same number of internal transitions) or every process (in the weak case where we relate processes with any amount of internal transitions).
We must therefore allow ourselves to observe more, and here choose to define the observables of a process as the names it can use for sending and receiving.

To this end, we define the \emph{observability predicate} \( \observable{P}{\mu} \) by the following rules:
\begin{align}
  \observable{P}{\obsin{x}}  &\quad \textrm{if \( P \) can perform an input action via \( x \).} \\
  \observable{P}{\obsout{x}} &\quad \textrm{if \( P \) can perform an output action via \( x \).}
\end{align}

Two processes \( P \) and \( Q \) are then \emph{strongly barbed bisimilar}, written \( \sbbisim{P}{Q} \), if:
\begin{gather}
  \observable{P}{\mu}~\textrm{implies}~\observable{Q}{\mu} \\
  \transition{P}{\Atau}{P'}~\textrm{implies}~\transition{Q}{\Atau}{Q'}~\textrm{for some \( Q' \) with}~\sbbisim{P'}{Q'} \\
  \observable{Q}{\mu}~\textrm{implies}~\observable{P}{\mu} \\
  \transition{Q}{\Atau}{Q'}~\textrm{implies}~\transition{P}{\Atau}{P'}~\textrm{for some \( P' \) with}~\sbbisim{P'}{Q'}
\end{gather}
Note that, since our calculus does not have infinite behaviour, we do not specify that the relation is the largest of its kind, and the relation is thus not coinductively defined.

\subsection{Challenge}
The objective of this challenge is to prove the following theorem:
\begin{theorem}
  \( \sbbisim{}{} \) is an equivalence relation, that is, the relation is reflexive, symmetric, and transitive.
\end{theorem}

\section{Challenge: Coinduction}
The idea of this challenge is to formalize a proof that requires coinductive techniques.
Coinduction is a proof technique for infinite structures, which arise in this context due to systems with behaviours that continue indefinitely.
Coinduction is the dual of induction: whereas induction is useful for proving properties of least fixed points, coinduction is useful for proving properties of greatest fixed points.

The setting for this challenge is an untyped calculus of communicating systems with replication of processes, the syntax and semantics of which are presented below.
The objective of this challenge is to prove that strong barbed bisimulation for this calculus is an equivalence relation.

\section{Challenge: Linearity}
The setting for this challenge is a restricted \( \pi \)-calculus with a session type system.

The objective of this challenge is to prove type preservation.

\section{Bonus challenges}

\subsection{Linearity and Scope Extrusion}
The setting for this challenge is the \( \pi \)-calculus with a session type system.

The objective of this challenge is to prove type preservation.

\subsection{Scope Extrusion and Coinduction}
The setting for this challenge is a simply typed \( \pi \)-calculus.

The objective of this challenge is to prove bisimulation.

\subsection{Coinduction and Linearity}
The setting for this challenge is the Calculus of Communicating Systems with a session type system.

The objective of this challenge is to prove type preservation.

\section{Future work}
\begin{itemize}
\item Multiparty session types
\item Choreography
\item Encodings
\item Code extraction
\end{itemize}

\end{document}

\end{document}
