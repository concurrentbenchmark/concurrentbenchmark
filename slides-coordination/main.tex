\documentclass[aspectratio=169,hyperref={pdfpagelabels=false}]{beamer}
\usepackage{helvet}
\usepackage[english]{babel}
\usepackage{pgfplots}
\usepackage{pgf}
\pgfplotsset{compat=newest}
\usepackage{booktabs}
\usepackage[T1]{fontenc}
\usepackage[utf8]{inputenc}
\usepackage{lipsum}
\usepackage{tcolorbox}
\usepackage{xcolor}
\usepackage{listings}
\usepackage[nopatch=footnote]{microtype}
\usepackage{float}
\usepackage{siunitx}
\usepackage{multicol}
\usepackage{hyperref}
\usepackage{dsfont}
\usepackage{caption}
\usepackage{subcaption}

% You might want to make the front/back page background colour the first colour in the plot cycle list.
\pgfplotscreateplotcyclelist{DTU}{%
dtured,         fill=dtured,        \\%
blue,           fill=blue,          \\%
brightgreen,    fill=brightgreen    \\%
navyblue,       fill=navyblue       \\%
yellow,         fill=yellow         \\%
orange,         fill=orange         \\%
grey,           fill=grey           \\%
red,            fill=red            \\%
green,          fill=green          \\%
purple,         fill=purple         \\%
}

% Table of contents (TOC) and numbering of headings
\setcounter{tocdepth}{1}    % Depth of table of content: sub sections will not be included in table of contents
\setcounter{secnumdepth}{2} % Depth of section numbering: sub sub sections are not numbered



\newcommand{\setcolor}[1]{\def\chosencolor{#1}}
\newcommand{\setevent}[1]{\def\event{#1}}

\usetheme{DTU}
\setbeamersize{text margin left=17mm}
\def\insertframetitle{}

\newcommand{\inserttitlepage}{
    \begin{frame}[plain]{}
        \color{white}\maketitle
    \end{frame}

    \setbeamercolor{background canvas}{bg = white}
}


\usepackage[utf8]{inputenc}
\usepackage{graphicx}
\usepackage{natbib}
\usepackage{amsmath}
\usepackage{amssymb}
\usepackage{amsthm}
\usepackage{mathpartir}
\usepackage{bm}
\usepackage{hyperref}
\usepackage{cleveref}

\title[The Concurrent Calculi Formalisation Benchmark]{\texorpdfstring{The Concurrent Calculi\\Formalisation Benchmark}{Concurrent Calculi Formalisation Benchmark}}
\author{Marco Carbone
\and David Castro-Perez
\and Francisco Ferreira
\and \texorpdfstring{\\Lorenzo Gheri}{Lorenzo Gheri}
\and \texorpdfstring{\underline{Frederik Krogsdal Jacobsen}}{Frederik Krogsdal Jacobsen}
\and Alberto Momigliano
\and Luca Padovani
\and Alceste Scalas
\and Dawit Tirore
\and Martin Vassor
\and Nobuko Yoshida
\and Daniel Zackon}
\date{June 20, 2024}

\setevent{COORDINATION 2024}
\setcolor{dtured}

\makeatletter
\def\beamer@andtitle{\unskip, }
\makeatother

\newcommand{\highlight}[1]{\colorbox{\chosencolor!40}{$\displaystyle \!#1\!$}}

\newcommand{\Pend}{\bm{0}}
\newcommand{\Ppar}[2]{#1 \mid #2}
\newcommand{\Pres}[2]{(\bm{\nu} #1)~#2}
\newcommand{\Pout}[3]{#1 ! #2 . #3}
\newcommand{\Pin}[3]{#1 ? (#2) . #3}
\newcommand{\Pchoice}[2]{#1 + #2}
\newcommand{\Preplicate}[1]{{!}#1}

\newcommand{\freenames}[1]{\textrm{fn}(#1)}
\newcommand{\boundnames}[1]{\textrm{bn}(#1)}
\newcommand{\names}[1]{\textrm{n}(#1)}

\newcommand{\freevars}[1]{\textrm{fv}(#1)}
\newcommand{\boundvars}[1]{\textrm{bv}(#1)}
\newcommand{\vars}[1]{\textrm{vars}(#1)}

\newcommand{\alphacon}[2]{#1 =_{\alpha} #2}
\newcommand{\subst}[3]{#1\{#2/#3\}}

\newcommand{\Aoutf}[2]{#1 ! #2}
\newcommand{\Aoutb}[2]{#1 ! (#2)}
\newcommand{\Ain}[2]{#1 ? #2}
\newcommand{\Atau}{\tau}

\newcommand{\transition}[3]{#1 \xrightarrow{#2} #3}

\newcommand{\observable}[2]{#1\downarrow_{#2}}
\newcommand{\obsin}[1]{#1?}
\newcommand{\obsout}[1]{#1!}

\newcommand\sbullet[1][.5]{\mathbin{\hbox{\scalebox{#1}{$\bullet$}}}}
\newcommand{\sbbisim}[2]{#1 \overset{\sbullet}{\sim} #2}
\newcommand{\nsbbisim}[2]{#1 \not\overset{\sbullet}{\sim} #2}

\newcommand{\ctxhole}{[\cdot]}
\newcommand{\applyctx}[2]{#1[#2]}
\newcommand{\sbcong}[2]{#1 \simeq^c #2}

\newcommand{\applysubst}[2]{#2#1}

\newcommand{\Presd}[3]{(\bm{\nu} #1#2)~#3}
\newcommand{\scong}[2]{#1 \equiv #2}

\newcommand{\reduces}[2]{#1 \rightarrow #2}

\newcommand{\Tend}{\mathbf{end}}
\newcommand{\Tbase}{\mathbf{base}}
\newcommand{\Tin}[1]{{?}.#1}
\newcommand{\Tout}[1]{{!}.#1}

\newcommand{\hastype}[2]{#1 : #2}

\newcommand{\un}[1]{\mathbf{un}(#1)}
\newcommand{\lin}[1]{\mathbf{lin}(#1)}

\newcommand{\Cempty}{\varnothing}
\newcommand{\Cadd}[2]{#1, #2}
\newcommand{\Csplit}[2]{#1 \circ #2}
\newcommand{\Cupdate}[2]{#1 + #2}

\newcommand{\dual}[1]{\overline{#1}}

\newcommand{\types}[2]{#1 \vdash #2}

%% Local Variables:
%% mode: latex
%% TeX-master: "main"
%% End:


\begin{document}
\inserttitlepage

%% \begin{frame}{Introduction}
%%   \begin{itemize}
%%   \item Why this challenge? We want to formalise session types, but that's a big challenge... we isolate smaller parts
%%   \item Why mechanized proofs?
%%     \begin{itemize}
%%     \item Certified code generation
%%     \item No mistakes in overlooked cases
%%     \item New insights
%%     \end{itemize}
%%   \item Basic concurrency + session types
%%   \item Why these challenges and not other stuff?
%%   \item Why is this related to session types (why is session types not the main thing?)
%%   \end{itemize}
%% \end{frame}

\begin{frame}{Introduction}

  We want (everyone) to mechanise concurrent systems!

  \ \\
  \visible<2->{
  Proof assistants are (fun and) useful:
    \begin{itemize}
    \item certified code generation
    \item no mistakes in overlooked cases
    \item new insights
    \end{itemize}
  }

  \ \\
  \visible<3->{
    Realisation: mechanising concurrent systems is a big effort.
  }
  
\end{frame}

\begin{frame}{The Benchmark Approach}
  Mixing concurrent calculi with the POPLMark spirit!

  \ \\
      We want to encourage:
    \begin{itemize}
    \item comparison of different approaches
    \item the development of guidelines, tutorials, techniques, libraries...
    \item reusability
    \end{itemize}


  \ \\
  \visible<2->{
  Three fundamental challenges on concurrency and session types:
  \begin{enumerate}
  \item linearity and behavioural type systems
  \item name passing and scope extrusion
  \item coinduction and infinite processes
  \end{enumerate}
  }
  
\ \\
  \url{https://concurrentbenchmark.github.io/}
\end{frame}

%%% LINEARITY

\begin{frame}{Linearity and behavioural type systems}
  %Handling contexts linearly + small example

    Processes:

      \[
      \begin{array}{rcl}
        v,w & ::= & a \quad\mid\quad l \\
        P,Q & ::=& \Pend \quad\mid\quad \Pout{x}{v}{P} \quad\mid\quad \Pin{x}{l}{P} \quad\mid\quad \PBpar{P}{Q} \quad\mid\quad  \Presd{x}{y}{P}
      \end{array}
      \]

  
  \visible<2->{
  Semantics:

  \begin{mathpar}
  \inferrule[R-Com]{ }{\reduces{\Presd{x}{y}{(\Ppar{\highlight{x ! a} . P}{\Ppar{\highlight{y ? (l)} . Q}{R}})}}{\Presd{x}{y}{(\Ppar{P}{\Ppar{\subst{Q}{a}{l}}{R}})}}}
  \and
  \inferrule[R-Res]{\reduces{P}{Q}}{\reduces{(\bm{\nu} \highlight{x y})~P}{(\bm{\nu} \highlight{x y})~Q}}
  \and
  \inferrule[R-Par]{\reduces{P}{Q}}{\reduces{\Ppar{P}{R}}{\Ppar{Q}{R}}}
  \and
  \inferrule[R-Struct]{\scong{P}{P'} \\ \reduces{P'}{Q'} \\ \scong{Q}{Q'}}{\reduces{P}{Q}}
  \end{mathpar}

  }


\end{frame}

\begin{frame}{Linearity and behavioural type systems}
  %Handling contexts linearly + small example

  %Desiderata:
\begin{enumerate}
\item No endpoint is used simultaneously by parallel processes. %;
\item The two endpoints of the same session are used dually.
\end{enumerate}
\ \\ \ \\
\visible<2->{
  Types:
  \begin{footnotesize}
\[
\begin{array}{rcl}
    S,T & ::= & \Tend \quad\mid\quad \Tbase \quad\mid\quad \Tin{S} \quad\mid\quad \Tout{S} \\
    \Gamma &::= & \Cempty \quad\mid\quad \Gamma, l \qquad\quad
    \begin{array}{rcl}
      \Delta &::= & \Cempty \quad\mid\quad \Cadd{\Delta}{\hastype{x}{S}}
    \end{array}
\end{array}
\]
  \end{footnotesize}
  
  Typing rules:
  \begin{footnotesize}
\begin{mathpar}

  \inferrule[T-Inact]{\tend\Delta }{\types{\Gamma;\Delta}{\Pend}}
  \and
  \inferrule[T-Par]{\types{\Gamma;\highlight{\Delta_1}}{P} \\ \types{\Gamma;\highlight{\Delta_2}}{Q}}
  {\types{\Gamma; \highlight{\Csplit{\Delta_1}{\Delta_2}}}{\Ppar{P}{Q}}}
  \and
  \inferrule[T-Res]{\types{\Gamma; (\Cadd{\Cadd{\Delta}{\hastype{x}{\highlight{T}}}}{\hastype{y}{\highlight{\dual{T}}}}}{P})}{\types{\Gamma}{\Presd{x}{y}{P}}}
  \and
    \inferrule[T-Out]{
      \typev{\Gamma}{\hastype{v}{\Tbase}} \\ \types{\Gamma; \Cupdate{\Delta}{\hastype{x}{T}}}{P}}{\types{\Gamma; (\Csplit{\Delta}{\hastype{x}{\Tout{T}}})}{\Pout{x}{v}{P}}}
    \and
    \inferrule[T-IN]{
      \types{(\Gamma,  l ); (\Cupdate{\Delta}{\hastype{x}{T}})}{P}}{\types{\Gamma; (\Csplit{\Delta}{\hastype{x}{\Tin{T}}})}{\Pin{x}{l}{P}}}
\end{mathpar}
\end{footnotesize}

}
  
\end{frame}

\begin{frame}{Linearity and behavioural type systems}
  %Handling contexts linearly + small example
  Challenge:

  \ \\
  \begin{theorem}[Subject reduction]
  If \( \types{\Gamma;\Delta}{P} \) and \( \reduces{P}{Q} \) then \( \types{\Gamma;\Delta}{Q} \)
  \end{theorem}

  \ \\
  \begin{theorem}[Type safety]
  If \( \types{\Gamma;\Cempty}{P} \), then \( P \) is well formed
  \end{theorem}
\end{frame}


%%% SCOPE EXTRUSION

\begin{frame}{Name passing and scope extrusion}
  %Handling opening and closing of restrictions + small example


    Processes:
      \[P,Q := \Pend\quad|\quad (\Ppar{P}{Q}) \quad|\quad\Pout{x}{y}{P}\quad|\quad\Pin{x}{y}{P}\quad|\quad \Pres{x}{P}\]
    

  \ \\
    One relevant example:
      \[\Ppar{(\Pres{y}{\Pout{x}{y}{P}})}{(\Pin{x}{z}{Q})}\]
      
\end{frame}

\begin{frame}{Name passing and scope extrusion}
  %Handling opening and closing of restrictions + small example

  First approach: structural congruence and reduction \\

  \[
  \begin{array}{ll}
    \Ppar{(\Pres{y}{\Pout{x}{y}{P}})}{(\Pin{x}{z}{Q})} & \equiv\\[2em]
    \Pres{y}{ (\Ppar{\Pout{x}{y}{P}}{\Pin{x}{z}{Q}})} & \rightarrow\\[2em]
    \Pres{y}{(\Ppar{P}{\subst{Q}{y}{z}})} & %\visible<4->{\dots}
  \end{array}
  \]
\end{frame}

\definecolor{myred}{RGB}{255,160,160}
\definecolor{mygreen}{RGB}{156,255,186}
\newcommand{\hg}[1]{\colorbox{mygreen}{$\displaystyle #1$}}
\newcommand{\hr}[1]{\colorbox{myred}{$\displaystyle #1$}}

\begin{frame}{Name passing and scope extrusion}
  %Handling opening and closing of restrictions + small example

  Second approach: labelled transition system \\

    \[
    \begin{array}{l}
      \inferrule{
        \inferrule{\transition{\Pout{x}{y}{P}}{\Aoutf{x}{y}}{P} \\ x\neq y}
                  {\hr{\transition{\Pres{y}{\Pout{x}{y}{P}}}{\Aoutb{x}{y}}{P}}}
             \\
                \transition{\Pin{x}{z}{Q}}{\Ain{x}{y}}{\subst{Q}{y}{z}} \\ y \notin \freenames{Q}
      }{
            \hg{\transition{\Ppar{(\Pres{y}{\Pout{x}{y}{P}})}{(\Pin{x}{z}{Q})}}{\tau}{\Pres{y}{(\Ppar{P}{\subst{Q}{y}{z}})}}}
  }
        
      \\ \ \\
                  \inferrule[\colorbox{myred}{Open}]{\transition{P}{\Aoutf{x}{z}}{P'} \\ z \neq x}{\transition{\Pres{z}{P}}{\Aoutb{x}{z}}{P'}}\hfill
          \inferrule[\colorbox{mygreen}{Close-L}]{\transition{P}{\Aoutb{x}{z}}{P'} \\ \transition{Q}{\Ain{x}{z}}{Q'} \\ z \notin \freenames{Q}}{\transition{\Ppar{P}{Q}}{\tau}{\Pres{z}{\Ppar{P'}{Q'}}}}
    \end{array}
    \]
  

\end{frame}

\begin{frame}{Name passing and scope extrusion}
  %Handling opening and closing of restrictions + small example
  Challenge:

  \ \\
  \begin{theorem}\label{thm:se-trans-implies-red}
  \( \transition{P}{\Atau}{Q} \) implies \( \reduces{P}{Q} \).
\end{theorem}
  \begin{theorem}
  \( \reduces{P}{Q} \) implies the existence of a \( Q' \) such that \( \transition{P}{\Atau}{Q'} \) and \( \scong{Q}{Q'} \).
\end{theorem}
\end{frame}

%%% COINDUCTION

\begin{frame}{Coinduction and infinite processes}
  %Up-to techniques + small example
  Describing the behaviour of recursive loops in programs.
\begin{displaymath}
  \begin{array}{r@{\qquad}c@{\qquad}l}
    v,w & ::= & a \ \mid\ l \\
    P,Q & ::= & \Pend
               \ \mid\ \Pout{x}{v}{P}
               \ \mid\  \Pin{x}{l}{P}
               \ \mid\  \PBpar{P}{Q}
               \ \mid\  \Pres{x}{}{P}
               \ \mid\  \hg{!P}
  \end{array}
\end{displaymath}
  \ \\ \ \\

  \[
  \begin{array}{c}
    \inferrule[Rep]{\transition{P}{\alpha}{P'}}{\transition{\Preplicate{P}}{\alpha}{\Ppar{P'}{\Preplicate{P}}}} \\[6mm]
    \alpha\ ::= \ \ \Aoutf{x}{a} \ \mid\ \Ain{x}{a} \ \mid\ \Atau
  \end{array}
  \]

\end{frame}

\begin{frame}{Coinduction and infinite processes}

  Observability predicate:
  \begin{align*}
  \observable{P}{\obsin{x}}  &\quad \textrm{if \( P \) can perform an input action via \( x \).} \\
  \observable{P}{\obsout{x}} &\quad \textrm{if \( P \) can perform an output action via \( x \).}
  \end{align*}

  Strong barbed bisimilarity:\\
  \textrm{the \emph{largest} symmetric relation such that, whenever \( \sbbisim{P}{Q} \):}
\begin{gather}
  \observable{P}{\mu}~\textrm{implies}~\observable{Q}{\mu}\label{eq:bisim1} \\
  \transition{P}{\Atau}{P'}~\textrm{implies}~\transition{Q}{\Atau}{\sbbisim{}{} P'}%\label{eq:bisim2}
\end{gather}
  

  \visible<2->{NOT a congruence:
    \begin{align*}
      &\Pout{x}{a}{\Pout{y}{b}{\Pend}} &&\overset{\sbullet}{\sim} &&\Pout{x}{a}{\Pend} \\
      &\Pout{x}{a}{\Pout{y}{b}{\Pend}} \mid \Pin{x}{l}{\Pend} &&\not\overset{\sbullet}{\sim} &&\Pout{x}{a}{\Pend} \mid \Pin{x}{l}{\Pend}
    \end{align*}

  }
  


\end{frame}


\begin{frame}{Coinduction and infinite processes}

Strong barbed congruence:\\  
\textrm{
  \( \sbcong{P}{Q} \), if \( \sbbisim{\applyctx{C}{P}}{\applyctx{C}{Q}} \) for every context \( C \).}

\ \\
\begin{lemma}
  \( \sbcong{}{} \) is the largest congruence included in
  \( \sbbisim{}{} \).
\end{lemma}


\ \\
\visible<2->{
  Challenge:

    \ \\
\begin{theorem}
  \( \sbcong{P}{Q} \) if, for any process \( R \) and
  substitution \( \sigma \),
  \(
  \sbbisim{\Ppar{\applysubst{\sigma}{P}}{R}}{\Ppar{\applysubst{\sigma}{Q}}{R}}
  \).
\end{theorem}


  
}


\end{frame}

\begin{frame}{Was this tedious?}
  \visible<2->{
  \ \\
    A community effort towards:
    \begin{itemize}
    \item tutorial formalisations for different approaches
    \item comparing different approaches
    \item establishing ``best practices''
    \item investigating strengths and weaknesses of proof assistants
    \item suggesting and developing new features of proof assistants
    \end{itemize}
}
  
\end{frame}

%% \begin{frame}{What's the point?}
%%   \begin{itemize}
%%   \item Showing off cool proof assistant features
%%   \item Tutorial formalizations showcasing various approaches
%%   \item Discovering the limits of current proof assistants and suggesting new features
%%   \item Comparing approaches, and maybe developing new techniques
%%   \item Establishing best practices which work across different ``flavours'' of session types
%%   \end{itemize}
%% \end{frame}

%% \begin{frame}{How to join}
%%   \begin{itemize}
%%   \item Website
%%   \item Mailing list
%%   \item Technical report
%%   \item Active formalisation efforts
%%   \end{itemize}
%% \end{frame}

\begin{frame}{Why contribute and how to get involved}

  Why:
  \begin{itemize}
  \item solving your problems (and other people's)
  \item connecting different parts of the community
  \item conducting your own mechanisation
  \item publication, both experience reports/tutorials and novelties
  \item learn a new proof assistant with cool features
  \end{itemize}

  \ \\
  {\LARGE \url{https://concurrentbenchmark.github.io/}}
  
\end{frame}

\begin{frame}{The long and winding road}
  ``How close are we to a world where every paper on \alt<2->{\textbf{concurrency}}{programming languages} is accompanied by an electronic appendix with machine-checked proofs?''
  
\end{frame}

\end{document}
