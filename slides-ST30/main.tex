\documentclass[aspectratio=169,hyperref={pdfpagelabels=false}]{beamer}
\usepackage{helvet}
\usepackage[english]{babel}
\usepackage{pgfplots}
\usepackage{pgf}
\pgfplotsset{compat=newest}
\usepackage{booktabs}
\usepackage[T1]{fontenc}
\usepackage[utf8]{inputenc}
\usepackage{lipsum}
\usepackage{tcolorbox}
\usepackage{xcolor}
\usepackage{listings}
\usepackage[nopatch=footnote]{microtype}
\usepackage{float}
\usepackage{siunitx}
\usepackage{multicol}
\usepackage{hyperref}
\usepackage{dsfont}
\usepackage{caption}
\usepackage{subcaption}

% You might want to make the front/back page background colour the first colour in the plot cycle list.
\pgfplotscreateplotcyclelist{DTU}{%
dtured,         fill=dtured,        \\%
blue,           fill=blue,          \\%
brightgreen,    fill=brightgreen    \\%
navyblue,       fill=navyblue       \\%
yellow,         fill=yellow         \\%
orange,         fill=orange         \\%
grey,           fill=grey           \\%
red,            fill=red            \\%
green,          fill=green          \\%
purple,         fill=purple         \\%
}

% Table of contents (TOC) and numbering of headings
\setcounter{tocdepth}{1}    % Depth of table of content: sub sections will not be included in table of contents
\setcounter{secnumdepth}{2} % Depth of section numbering: sub sub sections are not numbered



\newcommand{\setcolor}[1]{\def\chosencolor{#1}}
\newcommand{\setevent}[1]{\def\event{#1}}

\usetheme{DTU}
\setbeamersize{text margin left=17mm}
\def\insertframetitle{}

\newcommand{\inserttitlepage}{
    \begin{frame}[plain]{}
        \color{white}\maketitle
    \end{frame}

    \setbeamercolor{background canvas}{bg = white}
}


\usepackage[utf8]{inputenc}
\usepackage{graphicx}
\usepackage{natbib}
\usepackage{amsmath}
\usepackage{amssymb}
\usepackage{amsthm}
\usepackage{mathpartir}
\usepackage{bm}
\usepackage{hyperref}
\usepackage{cleveref}

\title[Concurrent Calculi Formalisation Benchmark]{\texorpdfstring{Concurrent Calculi\\Formalisation Benchmark}{Concurrent Calculi Formalisation Benchmark}}
\author{Marco Carbone
\and David Castro-Perez
\and Francisco Ferreira
\and \texorpdfstring{\underline{Lorenzo Gheri}}{Lorenzo Gheri}
\and Frederik Krogsdal Jacobsen
\and Alberto Momigliano
\and Luca Padovani
\and Alceste Scalas
\and Dawit Tirore
\and Martin Vassor
\and Nobuko Yoshida
\and Daniel Zackon}
\date{October 22, 2023}

\setevent{30 Years of Session Types}
\setcolor{dtured}

\makeatletter
\def\beamer@andtitle{\unskip, }
\makeatother

\newcommand{\Pend}{\bm{0}}
\newcommand{\Ppar}[2]{#1 \mid #2}
\newcommand{\Pres}[2]{(\bm{\nu} #1)~#2}
\newcommand{\Pout}[3]{#1 ! #2 . #3}
\newcommand{\Pin}[3]{#1 ? (#2) . #3}
\newcommand{\Pchoice}[2]{#1 + #2}
\newcommand{\Preplicate}[1]{{!}#1}

\newcommand{\freenames}[1]{\textrm{fn}(#1)}
\newcommand{\boundnames}[1]{\textrm{bn}(#1)}
\newcommand{\names}[1]{\textrm{n}(#1)}

\newcommand{\freevars}[1]{\textrm{fv}(#1)}
\newcommand{\boundvars}[1]{\textrm{bv}(#1)}
\newcommand{\vars}[1]{\textrm{vars}(#1)}

\newcommand{\alphacon}[2]{#1 =_{\alpha} #2}
\newcommand{\subst}[3]{#1\{#2/#3\}}

\newcommand{\Aoutf}[2]{#1 ! #2}
\newcommand{\Aoutb}[2]{#1 ! (#2)}
\newcommand{\Ain}[2]{#1 ? #2}
\newcommand{\Atau}{\tau}

\newcommand{\transition}[3]{#1 \xrightarrow{#2} #3}

\newcommand{\observable}[2]{#1\downarrow_{#2}}
\newcommand{\obsin}[1]{#1?}
\newcommand{\obsout}[1]{#1!}

\newcommand\sbullet[1][.5]{\mathbin{\hbox{\scalebox{#1}{$\bullet$}}}}
\newcommand{\sbbisim}[2]{#1 \overset{\sbullet}{\sim} #2}
\newcommand{\nsbbisim}[2]{#1 \not\overset{\sbullet}{\sim} #2}

\newcommand{\ctxhole}{[\cdot]}
\newcommand{\applyctx}[2]{#1[#2]}
\newcommand{\sbcong}[2]{#1 \simeq^c #2}

\newcommand{\applysubst}[2]{#2#1}

\newcommand{\Presd}[3]{(\bm{\nu} #1#2)~#3}
\newcommand{\scong}[2]{#1 \equiv #2}

\newcommand{\reduces}[2]{#1 \rightarrow #2}

\newcommand{\Tend}{\mathbf{end}}
\newcommand{\Tbase}{\mathbf{base}}
\newcommand{\Tin}[1]{{?}.#1}
\newcommand{\Tout}[1]{{!}.#1}

\newcommand{\hastype}[2]{#1 : #2}

\newcommand{\un}[1]{\mathbf{un}(#1)}
\newcommand{\lin}[1]{\mathbf{lin}(#1)}

\newcommand{\Cempty}{\varnothing}
\newcommand{\Cadd}[2]{#1, #2}
\newcommand{\Csplit}[2]{#1 \circ #2}
\newcommand{\Cupdate}[2]{#1 + #2}

\newcommand{\dual}[1]{\overline{#1}}

\newcommand{\types}[2]{#1 \vdash #2}

%% Local Variables:
%% mode: latex
%% TeX-master: "main"
%% End:


\begin{document}
\inserttitlepage

\begin{frame}{Introduction}
  \begin{itemize}
  \item Why this challenge? We want to formalise session types, but that's a big challenge... we isolate smaller parts
  \item Why mechanized proofs?
    \begin{itemize}
    \item Certified code generation
    \item No mistakes in overlooked cases
    \item New insights
    \end{itemize}
  \item Basic concurrency + session types
  \item Why these challenges and not other stuff?
  \item Why is this related to session types (why is session types not the main thing?)
  \end{itemize}
\end{frame}

\begin{frame}{Introduction}

  We want to mechanise Session Types!

  \ \\
  \visible<2->{
  Proof assistants are (fun and) useful:
    \begin{itemize}
    \item certified code generation
    \item no mistakes in overlooked cases
    \item new insights
    \end{itemize}
  }

  \ \\
  \visible<3->{
    Realisation: mechanising (multiparty) session types is a big effort.
  }
  
\end{frame}

\begin{frame}{The Benchmark Approach}
  Mixing session types with the POPLMark spirit!

  \ \\
  Three fundamental challenges on concurrency and session types:
  \begin{enumerate}
  \item linearity and behavioral type systems
  \item name passing and scope extrusion
  \item coinduction and infinite processes
  \end{enumerate}

  \ \\
  \visible<2->{
    What we want to encourage:
    \begin{itemize}
    \item the comparison of different approaches for accessibility
    \item the development of guidelines, tutorials, techniques, libraries...
    \item reusabilty
    \end{itemize}
  }
  
\ \\
  \url{https://concurrentbenchmark.github.io/}
\end{frame}

\begin{frame}{Linearity and behavioural type systems}
  %Handling contexts linearly + small example
  \visible<2->{
    Processes:
    \begin{footnotesize}
      \[
      \begin{array}{rcl}
        v,w & ::= & a \quad\mid\quad l \\
        P,Q & ::=& \Pend \quad\mid\quad \Pout{x}{v}{P} \quad\mid\quad \Pin{x}{l}{P} \quad\mid\quad \PBpar{P}{Q} \quad\mid\quad  \Presd{x}{y}{P}
      \end{array}
      \]
    \end{footnotesize}
  }
  
  \visible<3->{
  Semantics:
  \begin{footnotesize}
  \begin{mathpar}
  \inferrule[R-Com]{ }{\reduces{\Presd{x}{y}{(\Ppar{\Pout{x}{a}{P}}{\Ppar{\Pin{y}{l}{Q}}{R}})}}{\Presd{x}{y}{(\Ppar{P}{\Ppar{\subst{Q}{a}{l}}{R}})}}}
  \and
  \inferrule[R-Res]{\reduces{P}{Q}}{\reduces{\Presd{x}{y}{P}}{\Presd{x}{y}{Q}}}
  \and
  \inferrule[R-Par]{\reduces{P}{Q}}{\reduces{\Ppar{P}{R}}{\Ppar{Q}{R}}}
  \and
  \inferrule[R-Struct]{\scong{P}{P'} \\ \reduces{P'}{Q'} \\ \scong{Q}{Q'}}{\reduces{P}{Q}}
  \end{mathpar}
  \end{footnotesize}
  }


\end{frame}

\begin{frame}{Linearity and behavioural type systems}
  %Handling contexts linearly + small example

  %Desiderata:
\begin{enumerate}
\item No endpoint is used simultaneously by parallel processes. %;
\item The two endpoints of the same session have dual types.
\end{enumerate}
\ \\ \ \\
\visible<2->{
  Types:
  \begin{footnotesize}
\[
\begin{array}{rcl}
    S,T & ::= & \Tend \quad\mid\quad \Tbase \quad\mid\quad \Tin{S} \quad\mid\quad \Tout{S} \\
    \Gamma &::= & \Cempty \quad\mid\quad \Gamma, l \qquad\quad
    \begin{array}{rcl}
      \Delta &::= & \Cempty \quad\mid\quad \Cadd{\Delta}{\hastype{x}{S}}
    \end{array}
\end{array}
\]
  \end{footnotesize}
  
  Typing rules:
  \begin{footnotesize}
\begin{mathpar}

  \inferrule[T-Inact]{\tend\Delta }{\types{\Gamma;\Delta}{\Pend}}
  \and
  \inferrule[T-Par]{\types{\Gamma;\Delta_1}{P} \\ \types{\Gamma;\Delta_2}{Q}}
  {\types{\Gamma; \Csplit{\Delta_1}{\Delta_2}}{\Ppar{P}{Q}}}
  \and
  \inferrule[T-Res]{\types{\Gamma; (\Cadd{\Cadd{\Delta}{\hastype{x}{T}}}{\hastype{y}{\dual{T}}}}{P})}{\types{\Gamma}{\Presd{x}{y}{P}}}
  \and
    \inferrule[T-Out]{
      \typev{\Gamma}{\hastype{v}{\Tbase}} \\ \types{\Gamma; \Cupdate{\Delta}{\hastype{x}{T}}}{P}}{\types{\Gamma; (\Csplit{\Delta}{\hastype{x}{\Tout{T}}})}{\Pout{x}{v}{P}}}
    \and
    \inferrule[T-IN]{
      \types{(\Gamma,  l ); (\Cupdate{\Delta}{\hastype{x}{T}})}{P}}{\types{\Gamma; (\Csplit{\Delta}{\hastype{x}{\Tin{T}}})}{\Pin{x}{l}{P}}}
\end{mathpar}
\end{footnotesize}

}
  
\end{frame}

\begin{frame}{Linearity and behavioural type systems}
  %Handling contexts linearly + small example
  
  \begin{theorem}[Subject reduction]
  If \( \types{\Gamma;\Delta}{P} \) and \( \reduces{P}{Q} \) then \( \types{\Gamma;\Delta}{Q} \).
  \end{theorem}

  \ \\
  \visible<2->{
  \begin{theorem}[Type safety]
  If \( \types{\Cempty}{P} \), then \( P \) is well formed.
  \end{theorem}
  
  In particular, no $\PBpar{\Pout{x}{v}{P}}{\Pout{x}{v'}{P'}}$.
  }
\end{frame}




\begin{frame}{Name passing and scope extrusion}
  %Handling opening and closing of restrictions + small example

  \[P,Q := \Pend\quad|\quad \Ppar{P}{Q}\quad|\quad\Pout{x}{y}{P}\quad|\quad\Pin{x}{y}{P}\quad|\quad \Pres{x}{P}\]

  \ \\

  \visible<2->{
    One relevant example:
    \[\Ppar{(\Pres{y}{\Pout{x}{y}{P}})}{(\Pin{x}{z}{Q})}\]
  }
  
      
\end{frame}

\begin{frame}{Name passing and scope extrusion 1: structural congruence}
  %Handling opening and closing of restrictions + small example

  First approach:\ \\

  \[
  \begin{array}{ll}
    \Ppar{(\Pres{y}{\Pout{x}{y}{P}})}{(\Pin{x}{z}{Q})} & \visible<2->{\equiv}\\
    \visible<2->{\Pres{y}{ (\Ppar{\Pout{x}{y}{P}}{\Pin{x}{z}{Q}})}} & \visible<3->{\rightarrow}\\
    \visible<3->{\Pres{y}{(\Ppar{P}{\subst{Q}{y}{z}})}} & \visible<4->{\dots}
  \end{array}
  \]
  \ \\ \ \\

  \visible<2->{
    \alt<3->
        {
          \[
          \begin{array}{lll}
            \inferrule[R-Com]{ }{\reduces{\Ppar{\Pout{x}{y}{P}}{\Pin{x}{z}{Q}}}{\Ppar{P}{\subst{Q}{y}{z}}}}
            & \text{ and } &
            \inferrule[R-Res]{\reduces{P}{Q}}{\reduces{\Pres{x}{P}}{\Pres{x}{Q}}}
          \end{array}
          \]
        }
        {
          \[
          \inferrule[Sc-Res-Par]{x \notin \freenames{Q}}{\scong{\Ppar{\Pres{x}{P}}{Q}}{\Pres{x}{(\Ppar{P}{Q})}}}
          \]
        }
  }  
      
\end{frame}

\definecolor{myred}{RGB}{255,160,160}
\definecolor{mygreen}{RGB}{156,255,186}
\newcommand{\hg}[1]{\colorbox{mygreen}{$\displaystyle #1$}}
\newcommand{\hr}[1]{\colorbox{myred}{$\displaystyle #1$}}

\begin{frame}{Name passing and scope extrusion 2: labelled transition system}
  %Handling opening and closing of restrictions + small example

  Second approach:\ \\

  \[
  \begin{array}{llll}
    \Ppar{(\Pres{y}{\Pout{x}{y}{P}})}{(\Pin{x}{z}{Q})} & \visible<2->{\xrightarrow{\tau}} &
    \visible<2->{\Pres{y}{(\Ppar{P}{\subst{Q}{y}{z}})}} & \visible<6->{\dots}
  \end{array}
  \]
  \ \\ \ \\
\begin{footnotesize}
  \visible<2->{
    \[
    \begin{array}{l}
      \inferrule{
        \inferrule{\transition{\Pout{x}{y}{P}}{\Aoutf{x}{y}}{P}\\x\neq y}{\transition{\Pres{y}{\Pout{x}{y}{P}}}{\Aoutb{x}{y}}{P}} \\
        \transition{\Pin{x}{z}{Q}}{\Ain{x}{y}}{\subst{Q}{y}{z}} \\
        z \notin \freenames{Q}
  }{
        \alt<3->
            {\transition{\Ppar{(\Pres{y}{\Pout{x}{y}{P}})}{(\Pin{x}{z}{Q})}}{\tau}{\Pres{y}{(\Ppar{P}{\subst{Q}{y}{z}})}}}
            {\hr{\transition{\Ppar{(\Pres{y}{\Pout{x}{y}{P}})}{(\Pin{x}{z}{Q})}}{\tau}{\Pres{y}{(\Ppar{P}{\subst{Q}{y}{z}})}}}}
  }
        
      \\ \ \\
      \visible<3->{
      \alt<4->
          {
            \alt<5->
                {
                  \begin{array}{lll}
                    \inferrule[Out]{ }{\transition{\Pout{x}{y}{P}}{\Aoutf{x}{y}}{P}}
                    & \text{ and } &
                    \inferrule[In]{ }{\transition{\Pin{x}{z}{P}}{\Ain{x}{y}}{\subst{P}{y}{z}}}
                  \end{array}
                }
                {
                  \inferrule[Open]{\transition{P}{\Aoutf{x}{z}}{P'} \\ z \neq x}{\transition{\Pres{z}{P}}{\Aoutb{x}{z}}{P'}}
                } 
          }
          {
          \inferrule[Close-L]{\transition{P}{\Aoutb{x}{z}}{P'} \\ \transition{Q}{\Ain{x}{z}}{Q'} \\ z \notin \freenames{Q}}{\transition{\Ppar{P}{Q}}{\tau}{\Pres{z}{\Ppar{P'}{Q'}}}}
          }
          }
    \end{array}
    \]
  }  
\end{footnotesize}      
\end{frame}


\begin{frame}{Coinduction and infinite processes}
  %Up-to techniques + small example
  Describing the behaviour of recursive loops in programs.
\begin{displaymath}
  \begin{array}{r@{\qquad}c@{\qquad}l}
    v,w & ::= & a \ \mid\ l \\
    P,Q & ::= & \Pend
               \ \mid\ \Pout{x}{v}{P}
               \ \mid\  \Pin{x}{l}{P}
               \ \mid\  \PBpar{P}{Q}
               \ \mid\  \Pres{x}{}{P}
               \ \mid\  \alt<2->{\hg{!P}}{!P}
  \end{array}
\end{displaymath}
  \ \\ \ \\
\visible<2->{
  \[
  \begin{array}{c}
    \inferrule[Rep]{\transition{P}{\alpha}{P'}}{\transition{\Preplicate{P}}{\alpha}{\Ppar{P'}{\Preplicate{P}}}} \\[6mm]
    \alpha\ ::= \ \ \Aoutf{x}{a} \ \mid\ \Ain{x}{a} \ \mid\ \Atau
  \end{array}
  \]

}

\end{frame}

\begin{frame}{Coinduction and infinite processes}

  Observability predicate:
  \begin{align*}
  \observable{P}{\obsin{x}}  &\quad \textrm{if \( P \) can perform an input action via \( x \).} \\
  \observable{P}{\obsout{x}} &\quad \textrm{if \( P \) can perform an output action via \( x \).}
  \end{align*}

  Strong barbed bisimilarty:\\
  \textrm{the \emph{largest} symmetric relation such that, whenever \( \sbbisim{P}{Q} \):}
\begin{gather}
  \observable{P}{\mu}~\textrm{implies}~\observable{Q}{\mu}\label{eq:bisim1} \\
  \transition{P}{\Atau}{P'}~\textrm{implies}~\transition{Q}{\Atau}{\sbbisim{}{} P'}%\label{eq:bisim2}
\end{gather}
  

  \visible<2->{Equivalence, but NOT A CONGRUENCE: $\ \sbbisim{\Pout{x}{a}{\Pout{y}{b}{\Pend}}}{\Pout{x}{a}{\Pend}}\ $, but
   in the context
$C = \Ppar{\ctxhole}{\Pin{x}{l}{\Pend}}$, 
$\ \nsbbisim{\Ppar{\Pout{x}{a}{\Pout{y}{b}{\Pend}}}{\Pin{x}{l}{\Pend}}}{\Pout{x}{a}{\Pend}\Ppar{}{\Pin{x}{l}{\Pend}}}\ $.}
  


\end{frame}


\begin{frame}{Coinduction and infinite processes}

Strong barbed congruence:\\  
\textrm{
 \( \sbcong{P}{Q} \), if \( \sbbisim{\applyctx{C}{P}}{\applyctx{C}{Q}} \) for every context \( C \).}
   \ \\ \ \\ \ \\

PLACEHOLDER: results and upto.

\end{frame}

\begin{frame}{What's the point?}
  \begin{itemize}
  \item Showing off cool proof assistant features
  \item Tutorial formalizations showcasing various approaches
  \item Discovering the limits of current proof assistants and suggesting new features
  \item Comparing approaches, and maybe developing new techniques
  \item Establishing best practices which work across different ``flavours'' of session types
  \end{itemize}
\end{frame}

\begin{frame}{How to join}
  \begin{itemize}
  \item Website
  \item Mailing list
  \item Technical report
  \item Active formalisation efforts
  \end{itemize}
\end{frame}

\end{document}
