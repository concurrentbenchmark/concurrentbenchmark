\documentclass{article}

\usepackage[utf8]{inputenc}
\usepackage{graphicx}
\usepackage{natbib}
\usepackage{amsmath}
\usepackage{amssymb}
\usepackage{amsthm}
\usepackage{mathpartir}
\usepackage{bm}
\usepackage{hyperref}
\usepackage{cleveref}
\usepackage{draftwatermark}

\SetWatermarkText{Draft}
\SetWatermarkScale{5}
\SetWatermarkColor[gray]{0.90}

\newcommand{\Pend}{\bm{0}}
\newcommand{\Ppar}[2]{#1 \mid #2}
\newcommand{\Pres}[2]{(\bm{\nu} #1)~#2}
\newcommand{\Pout}[3]{#1 ! #2 . #3}
\newcommand{\Pin}[3]{#1 ? (#2) . #3}
\newcommand{\Pchoice}[2]{#1 + #2}
\newcommand{\Preplicate}[1]{{!}#1}

\newcommand{\freenames}[1]{\textrm{fn}(#1)}
\newcommand{\boundnames}[1]{\textrm{bn}(#1)}
\newcommand{\names}[1]{\textrm{n}(#1)}

\newcommand{\freevars}[1]{\textrm{fv}(#1)}
\newcommand{\boundvars}[1]{\textrm{bv}(#1)}
\newcommand{\vars}[1]{\textrm{vars}(#1)}

\newcommand{\alphacon}[2]{#1 =_{\alpha} #2}
\newcommand{\subst}[3]{#1\{#2/#3\}}

\newcommand{\Aoutf}[2]{#1 ! #2}
\newcommand{\Aoutb}[2]{#1 ! (#2)}
\newcommand{\Ain}[2]{#1 ? #2}
\newcommand{\Atau}{\tau}

\newcommand{\transition}[3]{#1 \xrightarrow{#2} #3}

\newcommand{\observable}[2]{#1\downarrow_{#2}}
\newcommand{\obsin}[1]{#1?}
\newcommand{\obsout}[1]{#1!}

\newcommand\sbullet[1][.5]{\mathbin{\hbox{\scalebox{#1}{$\bullet$}}}}
\newcommand{\sbbisim}[2]{#1 \overset{\sbullet}{\sim} #2}
\newcommand{\nsbbisim}[2]{#1 \not\overset{\sbullet}{\sim} #2}

\newcommand{\ctxhole}{[\cdot]}
\newcommand{\applyctx}[2]{#1[#2]}
\newcommand{\sbcong}[2]{#1 \simeq^c #2}

\newcommand{\applysubst}[2]{#2#1}

\newcommand{\Presd}[3]{(\bm{\nu} #1#2)~#3}
\newcommand{\scong}[2]{#1 \equiv #2}

\newcommand{\reduces}[2]{#1 \rightarrow #2}

\newcommand{\Tend}{\mathbf{end}}
\newcommand{\Tbase}{\mathbf{base}}
\newcommand{\Tin}[1]{{?}.#1}
\newcommand{\Tout}[1]{{!}.#1}

\newcommand{\hastype}[2]{#1 : #2}

\newcommand{\un}[1]{\mathbf{un}(#1)}
\newcommand{\lin}[1]{\mathbf{lin}(#1)}

\newcommand{\Cempty}{\varnothing}
\newcommand{\Cadd}[2]{#1, #2}
\newcommand{\Csplit}[2]{#1 \circ #2}
\newcommand{\Cupdate}[2]{#1 + #2}

\newcommand{\dual}[1]{\overline{#1}}

\newcommand{\types}[2]{#1 \vdash #2}
\newcommand{\typev}[2]{#1 \vdash_v #2}

\newif\ifreadfile
\newcommand{\myfile}[1]{%
  \ifreadfile
    \input{#1}
  \else
    % do nothing
  \fi
}


%% Local Variables:
%% mode: latex
%% TeX-master: "main"
%% End:


\begin{document}

\title{Working Draft: Lack of Unified Approaches Angle}

\maketitle

\begin{abstract}
  \noindent
  \begin{itemize}
  \item POPLMark made formalizations more common
  \item POPLMark does not cover the specific issues present in concurrency
  \item We propose three challenge problems concerning scope extrusion, linearity, and coinductive reasoning
  \end{itemize}
\end{abstract}

\section{Introduction}
\begin{itemize}
\item The PL community has embraced mechanization in the last decade
  \begin{itemize}
  \item POPLMark helped get the community on track for more mechanizations
  \item Since POPLMark, the number of formal proofs accompanying papers has grown
  \end{itemize}
\item The concurrency community has not embraced mechanization to the same extent
\item We propose a new benchmark for the specific issues present in concurrency
  \begin{itemize}
  \item A goal is to help develop momentum for more mechanizations
  \item A goal is to help develop a culture of mechanization
  \item A goal is to rally the community for collaborating on mechanizations
  \end{itemize}
\item Benchmarks cannot address every technique and field at once
  \begin{itemize}
  \item Challenges have to be simple enough to be manageable
  \item Challenges should not be encumbered by too many technical details
  \end{itemize}
\item POPLMark Reloaded extends the scope to proofs using logical relations
  \begin{itemize}
  \item Logical relations are crucial in many current developments
  \end{itemize}
\item Our benchmark extends the scope to process calculi and behavioural types
\item There are three key aspects that cause issues when mechanizing process calculi and behavioural types
  \begin{itemize}
  \item Scope extrusion in name passing systems generalises binders that makes techniques from lambda calculi inconvenient
  \item Linearity in behavioural type systems causes problems since linearity is generally not well supported
  \item Infinite behaviour modelled with coinduction causes problems since coinduction is generally not well supported
  \end{itemize}
\item Each part of our benchmark exercises the three aspects independently
  \begin{itemize}
  \item Real research projects will often need more than just one aspect
  \item We exercise each in isolation to keep solutions small
  \item We do not address how to combine techniques for the three aspects
  \end{itemize}
\item The problems we propose have been mechanized before
  \begin{itemize}
  \item A goal is to enable comparison between solutions using different techniques
  \item Our challenges are simple settings where we can compare techniques without distractions
  \end{itemize}
\end{itemize}

\section{Rationale for choosing the problems}
\begin{itemize}
\item Our challenges should cover the three aspects independently
  \begin{itemize}
  \item People can choose to implement only a single challenge
  \item People can choose to work on the challenges in any order
  \end{itemize}
\item Our challenges should be easy to understand and implement
  \begin{itemize}
  \item This is to obtain as many solutions as possible
  \item Our challenges should be small and quick to read
  \end{itemize}
\item Our challenges do not need to be met exactly
  \begin{itemize}
  \item People can choose to modify the challenge slightly to show the best aspects of their solution
  \item The spirit of each challenge is what matters
  \item Our challenges should be as technique-agnostic as possible
  \end{itemize}
\end{itemize}

\section{The challenge problems}
\begin{itemize}
\item Combining the techniques in the challenges is more of a research problem than a benchmark
\item Combining two techniques at a time can be interesting bonus challenges
\item The scope extrusion challenge is to prove that barbed bisimulation is an equivalence relation for an untyped \(\pi\)-calculus without infinite behaviour, but with non-deterministic choice
\item The coinduction challenge is to prove that strong barbed bisimulation is an equivalence relation for an untyped calculus of communicating systems with (infinite) replication of processes
\item The linearity challenge is to prove type preservation (aka.\ subject reduction) for a process calculus with a session type system
\end{itemize}

\section{Bonus challenges}
\begin{itemize}
\item Combining linearity and scope extrusion can be done by proving type preservation for the \(\pi\)-calculus with a session type system
\item Combining scope extrusion and coinduction can be done by proving that bisimulation is an equivalence relation for a simply typed \(\pi\)-calculus
\item Combining coinduction and linearity can be done by proving type preservation for CCS with a session type system
\end{itemize}

\section{The solutions}
\begin{itemize}
\item Should include a skeleton/sketch
\item Full details in an appendix
\end{itemize}

\section{Future work}
\begin{itemize}
\item Multiparty session types
\item Choreography
\item Encodings
\item Code extraction
\item Conversation types
\item Psi-calculus
\item Model checking for system properties
\item More general types than session types
\end{itemize}

\section{Conclusion and related work}

\end{document}