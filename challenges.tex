\subsection{Challenge: Linearity and behavioural type systems}
\label{sec:challenge:linearity-beh-types}

This challenge formalises a proof that requires reasoning about linearity of channels.
Linearity is the notion that a channel must be used exactly once in a process.
This is necessary to prove properties about session type systems.
Linear reasoning is also necessary to formalise e.g.\ linear and affine types for the pi-calculus and cut elimination in linear logics.

The setting for this challenge is a small calculus with a session type system, the syntax and semantics of which are presented below.
The objective of this challenge is to prove type preservation (also known as subject reduction) for the type system, i.e.\ that well-typed processes can only transition to processes which are also well-typed in the same context.

\subsubsection{Syntax}
We assume the existence of some type of \emph{base values}, values of which we will denote by \( a, b, \dots \), the existence of some type of \emph{variable names}, values of which we will denote by \( l, m, \dots \), and the existence of some type of \emph{names}, values of which we will denote by \( x, y, \dots \).
The syntax is then:
\begin{align*}
  v,w :=&&& a \\
  |&&& l \\
  P,Q :=&&& \Pend \\
  |&&& \Pout{x}{v}{P} \\
  |&&& \Pin{x}{l}{P} \\
  |&&& \Ppar{P}{Q} \\
  |&&& \Presd{x}{y}{P}
\end{align*}
A \emph{value} \( v, w, \dots \) is either a base value \( a \) or a variable name \( l \).
The process \( \Pend \) is \emph{inaction}: a process which can do nothing.
The process \( \Pout{x}{a}{P} \) is an \emph{output}, which can send the value \( v \) via \( x \), then continue as \( P \).
The intention is that the value \( v \) must be a base value when it is actually sent, and this will be enforced in the semantics later on.
The process \( \Pin{x}{l}{P} \) is an \emph{input}, which can receive a base value via \( x \), then continue as \( P \) with the received value substituted for the variable name \( l \).
The input operator thus binds the variable name \( l \) in \( P \).
The process \( \Ppar{P}{Q} \) is the \emph{composition} of process \( P \) and process \( Q \).
The two components can proceed independently of each other, or they can interact via shared names.
The process \( \Presd{x}{y}{P} \) is the \emph{restriction} of the channel with endpoints named \( x \) and \( y \) to \( P \).
Components in \( P \) can use the names \( x \) and \( y \) to interact with each other (sending on \( x \) and receiving on \( y \) or vice versa), but not with processes outside of the restriction.
The restriction operator thus binds the names \( x \) and \( y \) in \( P \).
Note that the scope of a restriction may not change when processes interact, since it is only possible to send and receive values, and not names.
Note that there is no recursion of replication in the syntax, and thus no infinite behaviours can be expressed.

We will use the notation \( \freenames{P} \) to denote the set of names that occur free (i.e.\ not bound by a restriction) in \( P \).
We will use the notation \( \boundnames{P} \) to denote the set of names that occur bound (by a restriction) in \( P \).
We will use the notation \( \freevars{P} \) to denote the set of variable names that occur free (i.e.\ not bound by an input) in \( P \).
We will use the notation \( \boundvars{P} \) to denote the set of variable names that occur bound (by an input) in \( P \).
We will use the notation \( \subst{P}{v}{w} \) to denote the process \( P \) with value \( v \) substituted for value \( w \).

Two processes \( P \) and \( Q \) are \( \alpha \)-convertible, written \( \alphacon{P}{Q} \), if \( Q \) can be obtained from \( P \) by a finite number of substitutions of bound variable names.
As a convention, we will identify \( \alpha \)-convertible processes.

As a convention, we assume that bound names and bound variable names of any processes are chosen to be different from the names and variable names that occur free in any other entities under consideration, such as processes, substitutions, and sets of names or variable names.
This is justified because any overlapping names and variable names may be \( \alpha \)-converted such that the assumption is satisfied.

\subsubsection{Semantics}
We describe the actions that the system can perform using a small step operational semantics.
To simplify the definition of the reduction relation, we will first factor out a structural congruence relation on processes, and for this we need a few definitions.

\subsubsection{Contexts and congruences}
A \emph{context} is obtained by taking a process and replacing a single occurrence of \( \Pend \) in it with the special \emph{hole} symbol \( \ctxhole \).
As a convention, we do \emph{not} identify \( \alpha \)-convertible contexts.

We will think of contexts as functions between processes.
A context \( C \) can be \emph{applied} to a process \( P \), written \( \applyctx{C}{P} \), by replacing the hole in C by \( P \), thus obtaining another process.
The replacement should be literal, so names and variable names that are free in \( P \) can become bound in \( \applyctx{C}{P} \).

We say that an equivalence relation \( \mathcal{S} \) is a \emph{congruence} if \( (P,Q) \in \mathcal{S} \) implies that for any context \( C \), \( (\applyctx{C}{P}, \applyctx{C}{Q}) \in \mathcal{S} \).

\emph{Structural congruence} is the smallest congruence relation that satisfies the following axioms:
\begin{mathpar}
  \inferrule[Sc-Par-Comm]{ }{\scong{\Ppar{P}{Q}}{\Ppar{Q}{P}}}
  \and
  \inferrule[Sc-Par-Assoc]{ }{\scong{\Ppar{(\Ppar{P}{Q})}{R}}{\Ppar{P}{(\Ppar{Q}{R})}}}
  \and
  \inferrule[Sc-Par-Inact]{ }{\scong{\Ppar{P}{\Pend}}{P}}
  \and
  \inferrule[Sc-Res-Par]{\{x,y\} \not\subseteq \freenames{Q}}{\scong{\Ppar{\Presd{x}{y}{P}}{Q}}{\Presd{x}{y}{(\Ppar{P}{Q})}}}
  \and
  \inferrule[Sc-Res-Inact]{ }{\scong{\Presd{x}{y}{\Pend}}{\Pend}}
  \and
  \inferrule[Sc-Res]{ }{\scong{\Presd{x_1}{y_1}{\Presd{x_2}{y_2}{P}}}{\Presd{x_2}{y_2}{\Presd{x_1}{y_1}{P}}}}
\end{mathpar}

The operational semantics is then defined as the following binary relation on processes:
\begin{mathpar}
  \inferrule[R-Com]{ }{\reduces{\Presd{x}{y}{(\Ppar{\Pout{x}{v}{P}}{\Ppar{\Pin{y}{l}{Q}}{R}})}}{\Presd{x}{y}{(\Ppar{P}{\Ppar{\subst{Q}{v}{l}}{R}})}}}
  \and
  \inferrule[R-Res]{\reduces{P}{Q}}{\reduces{\Presd{x}{y}{P}}{\Presd{x}{y}{Q}}}
  \and
  \inferrule[R-Par]{\reduces{P}{Q}}{\reduces{\Ppar{P}{R}}{\Ppar{Q}{R}}}
  \and
  \inferrule[R-Struct]{\scong{P}{P'} \\ \reduces{P'}{Q'} \\ \scong{Q}{Q'}}{\reduces{P}{Q}}
\end{mathpar}
Note that there is no rule for inferring an action of an input or output process except those that match the input/output capability.
Note also that due to rule \TirName{R-Com}, the process \( \Pin{y}{l}{P} \) can receive \emph{any} base value.
On the other hand, since the rule \TirName{R-Com} only applies to sending base values, there is no way to send a variable name or a name.

\subsubsection{Session types}
Our process syntax allows us to write processes that are not well-formed in the sense that they either use the endpoints bound by a restriction to communicate in a way that does not follow the intended duality of communication, or attempt to send something which is not a base value.
As an example, the following process attempts to send a base value on both \( x \) and \( y\), whereas the intention of binding names is that one of the names is used for sending and the other for receiving: \( \Presd{x}{y}{(\Ppar{\Pout{x}{a}{\Pend}}{\Pout{y}{a}{\Pend}})} \).
Another example is the following process, which attempts to send a variable that is not resolved at the time of sending: \( \Presd{x}{y}{(\Ppar{\Pout{x}{l}{\Pend}}{\Pin{y}{l}{\Pend}})} \).

To alleviate this issue, we introduce a \emph{session type system}, which will detect processes that are not well-formed.
To this end, we need to first formally define what we mean by well-formed.
We say that a process \( P \) is \emph{prefixed at variable \( x \)} if it is of the form \( \Pout{x}{v}{P} \) or \( \Pin{x}{l}{P} \).
A process is then \emph{well-formed} if, for each of its structurally congruenct processes of the form \( \Presd{x_1}{y_1}{\dots \Presd{x_n}{y_n}{(\Ppar{\Ppar{P}{Q}}{R})}} \), with \( n \geq 0 \), it holds that, if \( P \) is prefixed at \( x_1 \) and \( Q \) is prefixed at \( y_1 \), then \( \Ppar{P}{Q} \) is of the form \( \Ppar{\Pout{x_1}{a}{P'}}{\Pin{y_1}{l}{Q'}} \).

Note that well-formed processes do not necessarily reduce, since e.g.\ the process
\begin{equation*}
\Presd{x_1}{y_1}{\Presd{x_2}{y_2}{(\Ppar{\Pout{x_1}{a}{\Pin{y_2}{l}{\Pend}}}{\Pout{y}{x_2}{\Pin{y_1}{l}{\Pend}}})}}
\end{equation*}
is well-formed, but has no reduction.
\paragraph{Syntax}
Our type system does not type processes directly, but instead focuses on the channels used in the process.
The syntax of \emph{session types} \( S, T \) and \emph{type contexts} \( \Gamma \) is as follows:
\begin{align*}
  S,T := &&& \Tend \\
  |&&& \Tbase \\
  |&&& \Tin{S} \\
  |&&& \Tout{S} \\
  \Gamma := &&& \Cempty \\
  |&&& \Cadd{\Gamma}{\hastype{x}{S}}
\end{align*}
The \emph{end type}, \( \Tend \), is the type of an endpoint on which no further interaction is possible.
The \emph{base type}, \( \Tbase \), is the type of base values.
The \emph{input type}, \( \Tin{S} \), is the type of an endpoint which is ready to receive a value, then continue with type \( S \).
The \emph{output type}, \( \Tout{S} \), is the type of an endpoint which is ready to send a value, then continue with type \( S \).

Typing contexts gather type information about names.
The \emph{empty type context}, \( \Cempty \), does not contain any information.
The \emph{type context with \( x \) added}, \( \Cadd{\Gamma}{\hastype{x}{T}} \) adds the information that \( x \) has type \( T \) to the existing type context \( \Gamma \).
All of the names added to a type context must be distinct.
We treat type contexts as sets, not ordered lists, and the order in which information is added to a type context thus does not matter.

Since we need to determine whether endpoints are used dually to determine whether processes are well-formed, we will need to formally define the dual of a type.
The dual of a type, \( \dual{T} \), is defined by the following equations:
\begin{mathpar}
  \inferrule{}{\dual{\Tin{S}} = \Tout{\dual{S}}}
  \and
  \inferrule{}{\dual{\Tout{S}} = \Tin{\dual{S}}}
  \and
  \inferrule{}{\dual{\Tend} = \Tend}
\end{mathpar}
Note that the dual function is not total, since it is not defined for base types.

\paragraph{Splitting and updating contexts}
Our type system will maintain two invariants:
\begin{enumerate}
\item Endpoints are used exactly once
\item Endpoints that are part of the same restriction have dual types
\end{enumerate}
The first invariant is maintained by linearly splitting type contexts when typing compositions of processes.
The second invariant is maintained by requiring duality when typing restrictions.

To keep track of linearity, we introduce two \emph{qualifiers}: \( \un{T} \) and \( \lin{T} \).
A type is \emph{unrestricted}, \( \un{T} \), if and only if \( T = \Tbase \) or \( T = \Tend \).
All types are \emph{linear}, \( \lin{T} \).
The qualifiers extend to contexts in the following way: \( q(\Gamma) \) if and only if \( \hastype{x}{T} \in \Gamma \) implies \( q(T) \).

The \emph{linear split} of a context, \( \Gamma = \Csplit{\Gamma_1}{\Gamma_2} \), is defined by the following rules:
\begin{mathpar}
  \inferrule[Split-Empty]{ }{\Cempty = \Csplit{\Cempty}{\Cempty}}
  \and
  \inferrule[Split-Un]{\Csplit{\Gamma_1}{\Gamma_2} = \Gamma \\ \un{T}}{\Cadd{\Gamma}{\hastype{x}{T}} = \Csplit{(\Cadd{\Gamma_1}{\hastype{x}{T}})}{(\Cadd{\Gamma_2}{\hastype{x}{T}})}}
  \and
  \inferrule[Split-Lin-L]{\Gamma = \Csplit{\Gamma_1}{\Gamma_2}}{\Cadd{\Gamma}{\hastype{x}{S}} = \Csplit{(\Cadd{\Gamma_1}{\hastype{x}{S}})}{\Gamma_2}}
  \and
  \inferrule[Split-Lin-R]{\Gamma = \Csplit{\Gamma_2}{\Gamma_2}}{\Cadd{\Gamma}{\hastype{x}{S}} = \Csplit{\Gamma_1}{(\Cadd{\Gamma_2}{\hastype{x}{S}})}}
\end{mathpar}

We will also need to update contexts with new type information in a safe way while typing input and output.
The \emph{context update}, \( \Cupdate{\Gamma}{\hastype{x}{T}} = \Gamma' \), is defined by the following rules:
\begin{mathpar}
  \inferrule[Update-Name]{\hastype{x}{U} \notin \Gamma}{\Cupdate{\Gamma}{\hastype{x}{T}} = \Cadd{\Gamma}{\hastype{x}{T}}}
  \and
  \inferrule[Update-Un]{\un{T}}{\Cupdate{(\Cadd{\Gamma}{\hastype{x}{T}})}{\hastype{x}{T}} = (\Cadd{\Gamma}{\hastype{x}{T}})}
\end{mathpar}

\subsubsection{Typing rules}
We have three typing judgements: one for values, one for names, and one for processes.
The typing rules for values are as follows:
\begin{mathpar}
  \inferrule[T-Base]{\un{\Gamma}}{\types{\Gamma}{\hastype{a}{\Tbase}}}
  \and
  \inferrule[T-Var]{\un{\Gamma}}{\types{\Cadd{\Gamma}{\hastype{v}{\Tbase}}}{\hastype{v}{\Tbase}}}
\end{mathpar}
The only typing rule for names is:
\begin{mathpar}
  \inferrule[T-Name]{\un{\Gamma}}{\types{\Cadd{\Gamma}{\hastype{x}{T}}}{\hastype{x}{T}}}
\end{mathpar}
The typing rules for processes are as follows:
\begin{mathpar}
  \inferrule[T-Inact]{\un{\Gamma}}{\types{\Gamma}{\Pend}}
  \and
  \inferrule[T-Par]{\types{\Gamma_1}{P} \\ \types{\Gamma_2}{Q}}{\types{\Csplit{\Gamma_1}{\Gamma_2}}{\Ppar{P}{Q}}}
  \and
  \inferrule[T-Res]{\types{\Cadd{\Cadd{\Gamma}{\hastype{x}{T}}}{\hastype{y}{\dual{T}}}}{P}}{\types{\Gamma}{\Presd{x}{y}{P}}}
  \and
  \inferrule[T-Out]{\types{\Gamma_1}{\hastype{x}{\Tout{T}}} \\ \types{\Gamma_2}{\hastype{v}{\Tbase}} \\ \types{\Cupdate{\Gamma_3}{\hastype{x}{T}}}{P}}{\types{\Csplit{\Gamma_1}{\Csplit{\Gamma_2}{\Gamma_3}}}{\Pout{x}{v}{P}}}
  \and
  \inferrule[T-In]{\types{\Gamma_1}{\hastype{x}{\Tin{U}}} \\ \types{\Cupdate{(\Cadd{\Gamma_2}{\hastype{l}{\Tbase}})}{\hastype{x}{U}}}{P}}{\types{\Csplit{\Gamma_1}{\Gamma_2}}{\Pin{x}{l}{P}}}
\end{mathpar}

\subsubsection{Challenge}
The objective of this challenge is to prove the following theorems:
\begin{theorem}
  If \( \types{\Gamma}{P} \) and \( \reduces{P}{Q} \) then \( \types{\Gamma}{Q} \).
\end{theorem}

\begin{theorem}
  If \( \types{\Cempty}{P} \), then \( P \) is well-formed.
\end{theorem}

\begin{corollary}
  If \( \types{\Cempty}{P} \) and \( \reduces{P}{Q} \) then \( Q \) is well-formed.
\end{corollary}

The following lemmas may be of use:
\begin{lemma}
  If \( \types{\Gamma}{P} \) and \( \un{T} \) then \( \types{\Cadd{\Gamma}{\hastype{x}{T}}}{P} \).
\end{lemma}

\begin{lemma}
  Assume \( \types{\Gamma}{P} \) and \( x \notin \freenames{P} \).
  Then:
  \begin{enumerate}
  \item \( \hastype{x}{\Tout{T}} \notin \Gamma \).
  \item \( \hastype{x}{\Tin{T}} \notin \Gamma \).
  \item if \( \Gamma = \Cadd{\Gamma'}{\hastype{x}{T}} \) then \( \types{\Gamma'}{P} \).
  \end{enumerate}
\end{lemma}

\begin{lemma}
  If \( \types{\Gamma}{P} \) and \( \scong{P}{Q} \) then \( \types{\Gamma}{Q} \).
\end{lemma}

\begin{lemma}
  If \( \types{\Gamma_1}{\hastype{v}{T}} \) and \( \types{\Cadd{\Gamma_2}{\hastype{x}{T}}}{P} \) and \( \Gamma = \Csplit{\Gamma_1}{\Gamma_2} \) then \( \types{\Gamma}{\subst{v}{x}{P}} \).
\end{lemma}

\subsection{Challenge: Name passing and scope extrusion}
\label{sec:challenge:name-passing-scope-extrusion}

This challenge formalises a proof that requires explicit scope extrusion.
Scope extrusion is the notion that a process can send restricted names to another process, as long as the restriction can safely be ``extruded'' (i.e.\ expanded) to include the recieving process.
This e.g.\ allows a process to set up a private connection by sending a restricted name to another process, then using this name for further communication.

The setting for this challenge is an untyped \( \pi \)-calculus, the syntax and semantics of which are presented below.
The objective of this challenge is to prove that barbed bisimulation for this calculus is an equivalence relation.

\subsubsection{Syntax}
We assume the existence of a type of \emph{names} that can be compared for equality, values of which we will denote by \( x, y, \dots \).
The syntax is then:
\begin{align*}
  P,Q :=&&& \Pend \\
  |&&& \Pout{x}{y}{P} \\
  |&&& \Pin{x}{y}{P} \\
  |&&& \Ppar{P}{Q} \\
  |&&& \Pres{x}{P} \\
  |&&& \Pchoice{P}{Q}
\end{align*}
The process \( \Pend \) is \emph{inaction}: a process which can do nothing.
The process \( \Pout{x}{y}{P} \) is an \emph{output}, which can send the name \( y \) via \( x \), then continue as \( P \).
The process \( \Pin{x}{y}{P} \) is an \emph{input}, which can receive a name via \( x \), then continue as \( P \) with the received name substituted for \( y \).
The input operator thus binds the name \( y \) in \( P \).
The process \( \Ppar{P}{Q} \) is the \emph{composition} of process \( P \) and process \( Q \).
The two components can proceed independently of each other, or they can interact via shared names.
The process \( \Pres{x}{P} \) is the \emph{restriction} of the name \( x \) to \( P \).
Components in \( P \) can use the name \( x \) to interact with each other, but not with processes outside of the restriction.
The restriction operator thus binds the name \( x \) in \( P \).
Note that the scope of a restriction may change when processes interact. Namely, a restricted name may be sent \emph{outside} of its scope.
The process \( \Pchoice{P}{Q} \) is a non-deterministic \emph{choice} between continuing as the process \( P \) or as the process \( Q \).
Note that there is no recursion or replication in the syntax, and thus no infinite behaviours can be expressed.

We will use the notation \( \freenames{P} \) to denote the set of names that occur free (i.e.\ not bound by a restriction or an input) in \( P \).
We will use the notation \( \boundnames{P} \) to denote the set of names that occur bound (by a restriction or an input) in \( P \).
We will use the notation \( \subst{P}{x}{y} \) to denote the process \( P \) with \( x \) substituted for \( y \).

Two processes \( P \) and \( Q \) are \( \alpha \)-convertible, written \( \alphacon{P}{Q} \), if \( Q \) can be obtained from \( P \) by a finite number of substitutions of bound names.
As a convention, we will identify \( \alpha \)-convertible processes.

As a convention, we assume that the bound names occurring in any collection of processes are chosen to be different from the free names occurring in those processes and from the names occurring in any substitutions applied to the processes.
This is justified because any overlapping names may be \( \alpha \)-converted such that the assumption is satisfied.

\subsubsection{Semantics}
The semantics of the system describe the actions that the system can perform by defining a labelled transition relation on processes.
The transitions are labelled by \emph{actions}, the syntax of which is as follows:
\begin{align*}
  \alpha := &&& \Aoutf{x}{y} \\
  |&&& \Ain{x}{y} \\
  |&&& \Aoutb{x}{y} \\
  |&&& \Atau
\end{align*}
The \emph{free output action} \( \Aoutf{x}{y} \) is sending the name \( y \) via \( x \).
The \emph{input action} \( \Ain{x}{y} \) is receiving the name \( y \) via \( x \).
The \emph{bound output action} \( \Aoutb{x}{y} \) is sending a fresh name \( y \) via \( x \).
The \emph{internal action} \( \Atau \) is performing some unobservable action, e.g.\ internal communication.

We will again use the notation \( \freenames{\alpha} \) to denote the set of names that occur free in the action \( \alpha \) and the notation \( \boundnames{\alpha} \) to denote the set of names that occur bound in the action \( \alpha \).
In the free output action \( \Aoutf{x}{y} \) and the input action \( \Ain{x}{y} \), both \( x \) and \( y \) are free names.
In the bound output action \( \Aoutb{x}{y} \), \( x \) is a free name, while \( y \) is a bound name.
We will further use the notation \( \names{\alpha} \) to denote the union of \( \freenames{\alpha} \) and \( \boundnames{\alpha} \), i.e.\ the set of all names that occur in the action \( \alpha \).

The transition relation is then defined by the following rules:
\begin{mathpar}
  \inferrule[Out]{ }{\transition{\Pout{x}{y}{P}}{\Aoutf{x}{y}}{P}}
  \and
  \inferrule[In]{ }{\transition{\Pin{x}{z}{P}}{\Ain{x}{y}}{\subst{P}{y}{z}}}
  \and
  \inferrule[Sum-L]{\transition{P}{\alpha}{P'}}{\transition{\Pchoice{P}{Q}}{\alpha}{P'}}
  \and
  \inferrule[Sum-R]{\transition{P}{\alpha}{P'}}{\transition{\Pchoice{Q}{P}}{\alpha}{P'}}
  \and
  \inferrule[Par-L]{\transition{P}{\alpha}{P'} \\ \boundnames{\alpha} \cap \freenames{Q} = \emptyset}{\transition{\Ppar{P}{Q}}{\alpha}{\Ppar{P'}{Q}}}
  \and
  \inferrule[Par-R]{\transition{Q}{\alpha}{Q'} \\ \boundnames{\alpha} \cap \freenames{P} = \emptyset}{\transition{\Ppar{P}{Q}}{\alpha}{\Ppar{P}{Q'}}}
  \and
  \inferrule[Comm-L]{\transition{P}{\Aoutf{x}{y}}{P'} \\ \transition{Q}{\Ain{x}{y}}{Q'}}{\transition{\Ppar{P}{Q}}{\Atau}{\Ppar{P'}{Q'}}}
  \and
  \inferrule[Comm-R]{\transition{P}{\Ain{x}{y}}{P'} \\ \transition{Q}{\Aoutf{x}{y}}{Q'}}{\transition{\Ppar{P}{Q}}{\Atau}{\Ppar{P'}{Q'}}}
  \and
  \inferrule[Close-L]{\transition{P}{\Aoutb{x}{z}}{P'} \\ \transition{Q}{\Ain{x}{z}}{Q'} \\ z \notin \freenames{Q}}{\transition{\Ppar{P}{Q}}{\tau}{\Pres{z}{\Ppar{P'}{Q'}}}}
  \and
  \inferrule[Open]{\transition{P}{\Aoutf{x}{z}}{P'} \\ z \neq x}{\transition{\Pres{z}{P}}{\Aoutb{x}{z}}{P'}}
  \and
  \inferrule[Close-R]{\transition{P}{\Ain{x}{z}}{P'} \\ \transition{Q}{\Aoutb{x}{z}}{Q'} \\ z \notin \freenames{P}}{\transition{\Ppar{P}{Q}}{\tau}{\Pres{z}{\Ppar{P'}{Q'}}}}
  \and
  \inferrule[Res]{\transition{P}{\alpha}{P'} \\ z \notin \names{\alpha}}{\transition{\Pres{z}{P}}{\alpha}{\Pres{z}{P'}}}
\end{mathpar}
Note that there is no rule for inferring transitions from \( \Pend \), and that there is no rule for inferring an action of an input or output process except those that match the input/output capability.
Note also that due to rule \TirName{In}, the process \( \Pin{x}{z}{P} \) can receive \emph{any} name.

As a convention, we assume that bound names of any processes or actions are chosen to be different from the names that occur free in any other entities under consideration, such as processes, actions, substitutions, and sets of names.
The convention has one exception, namely that in the transition \( \transition{P}{\Aoutb{x}{z}}{Q} \), the name \( z \) (which occurs bound in \( P \) and the action \( \Aoutb{x}{z} \)) may occur free in \( Q \).
Without this exception it would be impossible to express scope extrusion.

\subsubsection{Bisimilarity}
Our notion of process equivalence relations builds on a notion of \emph{observables}.
If we allow ourselves only to observe internal transitions, we will relate either too few processes (in the strong case where we relate only processes with exactly the same number of internal transitions) or every process (in the weak case where we relate processes with any amount of internal transitions).
We must therefore allow ourselves to observe more, and here choose to define the observables of a process as the names it can use for sending and receiving.

To this end, we define the \emph{observability predicate} \( \observable{P}{\mu} \) by the following rules:
\begin{align}
  \observable{P}{\obsin{x}}  &\quad \textrm{if \( P \) can perform an input action via \( x \).} \\
  \observable{P}{\obsout{x}} &\quad \textrm{if \( P \) can perform an output action via \( x \).}
\end{align}

Two processes \( P \) and \( Q \) are then \emph{strong barbed bisimilar}, written \( \sbbisim{P}{Q} \), if:
\begin{gather}
  \observable{P}{\mu}~\textrm{implies}~\observable{Q}{\mu} \\
  \transition{P}{\Atau}{P'}~\textrm{implies}~\transition{Q}{\Atau}{Q'}~\textrm{for some \( Q' \) with}~\sbbisim{P'}{Q'} \\
  \observable{Q}{\mu}~\textrm{implies}~\observable{P}{\mu} \\
  \transition{Q}{\Atau}{Q'}~\textrm{implies}~\transition{P}{\Atau}{P'}~\textrm{for some \( P' \) with}~\sbbisim{P'}{Q'}
\end{gather}
Note that, since our calculus does not have infinite behaviour, there is only one relation which satisfies these criteria.
Thus we do not specify that the relation is the largest of its kind, and the relation does not need to be coinductively defined.
To actually prove properties about the relation, however, we will still need either an inductive (or a coinductive) proof principle for the relation.

\subsubsection{Challenge}
The objective of this challenge is to prove the following theorem:
\begin{theorem}
  \( \sbbisim{}{} \) is an equivalence relation, that is, the relation is reflexive, symmetric, and transitive.
\end{theorem}

\subsection{Challenge: Coinduction and reasoning about infinite processes}
\label{sec:challenge:coinduction}

This challenge formalises a proof that requires coinductive techniques.
Coinduction is a proof technique for infinite structures, which arise in this context due to systems with behaviours that continue indefinitely.
Coinduction is the dual of induction: whereas induction is useful for proving properties of least fixed points, coinduction is useful for proving properties of greatest fixed points.

The setting for this challenge is an untyped calculus of communicating systems with replication of processes, the syntax and semantics of which are presented below.
The objective of this challenge is to prove that strong barbed bisimulation for this calculus is an equivalence relation. Which, to reiterate, entails the use of coinduction due to the presence of infinite behaviour.

\subsubsection{Syntax}
We assume the existence of some type of \emph{base values}, values of which we will denote by \( a, b, \dots \), the existence of some type of \emph{variable names}, values of which we will denote by \( l, m, \dots \), and the existence of some type of \emph{names}, values of which we will denote by \( x, y, \dots \).
The syntax is then:
\begin{align*}
  v,w :=&&& a \\
  |&&& l \\
  P,Q :=&&& \Pend \\
  |&&& \Pout{x}{v}{P} \\
  |&&& \Pin{x}{l}{P} \\
  |&&& \Ppar{P}{Q} \\
  |&&& \Pres{x}{P} \\
  |&&& \Preplicate{P}
\end{align*}
A \emph{value} \( v, w, \dots \) is either a base value \( a \) or a variable name \( l \).
The process \( \Pend \) is \emph{inaction}: a process which can do nothing.
The process \( \Pout{x}{a}{P} \) is an \emph{output}, which can send the value \( v \) via \( x \), then continue as \( P \).
The intention is that the value \( v \) must be a base value when it is actually sent, and this will be enforced in the semantics later on.
The process \( \Pin{x}{l}{P} \) is an \emph{input}, which can receive a base value via \( x \), then continue as \( P \) with the received value substituted for the variable name \( l \).
The input operator thus binds the variable name \( l \) in \( P \).
The process \( \Ppar{P}{Q} \) is the \emph{composition} of process \( P \) and process \( Q \).
The two components can proceed independently of each other, or they can interact via shared names.
The process \( \Pres{x}{P} \) is the \emph{restriction} of the name \( x \) to \( P \).
Components in \( P \) can use the name \( x \) to interact with each other, but not with processes outside of the restriction.
The restriction operator thus binds the name \( x \) in \( P \).
Note that the scope of a restriction may not change when processes interact, since it is only possible to send and receive values, and not names.
The process \( \Preplicate{P} \) is the \emph{replication} of the process \( P \).
It can be thought of as the infinite composition \( \Ppar{P}{\Ppar{P}{\cdots}} \).
Replication makes it possible to express infinite behaviours.

We will use the notation \( \subst{P}{v}{w} \) to denote the process \( P \) with value \( v \) substituted for value \( w \).
Two processes \( P \) and \( Q \) are \( \alpha \)-convertible, written \( \alphacon{P}{Q} \), if \( Q \) can be obtained from \( P \) by a finite number of substitutions of bound variable names.
As a convention, we will identify \( \alpha \)-convertible processes.

Also as a convention, we assume that the bound names and bound variable names occurring in any collection of processes are chosen to be different from the free names and free variable names occurring in those processes and from the names and variable names occurring in any substitutions applied to the processes.
This is justified because any overlapping names and variable names may be \( \alpha \)-converted such that the assumption is satisfied.

\subsubsection{Semantics}
The semantics of the system describe the actions that the system can perform by defining a labelled transition relation on processes.
The transitions are labelled by \emph{actions}, the syntax of which are as follows:
\begin{align*}
  \alpha := &&& \Aoutf{x}{a} \\
  |&&& \Ain{x}{a} \\
  |&&& \Atau
\end{align*}
The \emph{output action} \( \Aoutf{x}{y} \) is sending the base value \( a \) via \( x \).
The \emph{input action} \( \Ain{x}{y} \) is receiving the base value \( y \) via \( x \).
The \emph{internal action} \( \Atau \) is performing some unobservable action, e.g.\ internal communication.

We will use the notation \( \names{\alpha} \) to denote the set of names that occur in the action \( \alpha \).

The transition relation is then defined by the following rules:
\begin{mathpar}
  \inferrule[Out]{ }{\transition{\Pout{x}{a}{P}}{\Aoutf{x}{a}}{P}}
  \and
  \inferrule[In]{ }{\transition{\Pin{x}{l}{P}}{\Ain{x}{a}}{\subst{P}{a}{l}}}
  \and
  \inferrule[Par-L]{\transition{P}{\alpha}{P'}}{\transition{\Ppar{P}{Q}}{\alpha}{\Ppar{P'}{Q}}}
  \and
  \inferrule[Par-R]{\transition{Q}{\alpha}{Q'}}{\transition{\Ppar{P}{Q}}{\alpha}{\Ppar{P}{Q'}}}
  \and
  \inferrule[Comm-L]{\transition{P}{\Aoutf{x}{a}}{P'} \\ \transition{Q}{\Ain{x}{a}}{Q'}}{\transition{\Ppar{P}{Q}}{\Atau}{\Ppar{P'}{Q'}}}
  \and
  \inferrule[Comm-R]{\transition{P}{\Ain{x}{a}}{P'} \\ \transition{Q}{\Aoutf{x}{a}}{Q'}}{\transition{\Ppar{P}{Q}}{\Atau}{\Ppar{P'}{Q'}}}
  \and
  \inferrule[Res]{\transition{P}{\alpha}{P'} \\ x \notin \names{\alpha}}{\transition{\Pres{x}{P}}{\alpha}{\Pres{x}{P'}}}
  \and
  \inferrule[Rep]{\transition{P}{\alpha}{P'}}{\transition{\Preplicate{P}}{\alpha}{\Ppar{P'}{\Preplicate{P}}}}
\end{mathpar}
Note that there is no rule for inferring transitions from \( \Pend \), and that there is no rule for inferring an action of an input or output process except those that match the input/output capability.
Note also that due to rule \TirName{In}, the process \( \Pin{x}{l}{P} \) can receive \emph{any} base value.
On the other hand, since the rule \TirName{Out} only applies to base values, there is no way to send a variable name.

As a convention, we assume that bound names and bound variable names of any processes or actions are chosen to be different from the names and variable names that occur free in any other entities under consideration, such as processes, actions, substitutions, and sets of names or variable names.

\subsubsection{Bisimilarity}
Our notion of process equivalence relations builds on a notion of \emph{observables}.
If we allow ourselves only to observe internal transitions, we will relate either too few processes (in the strong case where we relate only processes with exactly the same number of internal transitions) or every process (in the weak case where we relate processes with any amount of internal transitions).
We must therefore allow ourselves to observe more, and here choose to define the observables of a process as the names it can use for sending and receiving.

To this end, we define the \emph{observability predicate} \( \observable{P}{\mu} \) by the following rules:
\begin{align}
  \observable{P}{\obsin{x}}  &\quad \textrm{if \( P \) can perform an input action via \( x \).} \\
  \observable{P}{\obsout{x}} &\quad \textrm{if \( P \) can perform an output action via \( x \).}
\end{align}

\emph{Strong barbed bisimilarity}, written \( \sbbisim{}{} \), is then the largest symmetric relation such that, whenever \( \sbbisim{P}{Q} \):
\begin{gather}
  \observable{P}{\mu}~\textrm{implies}~\observable{Q}{\mu} \\
  \transition{P}{\Atau}{P'}~\textrm{implies}~\transition{Q}{\Atau}{\sbbisim{}{} P'}
\end{gather}
We say that a relation is a \emph{strong barbed bisimulation} if it satisfies the conditions given above, but is not necessarily the largest such relation, and that \( P \) and \( Q \) are \emph{strong barbed bisimilar} if \( \sbbisim{P}{Q} \).
Note that, since our systems have potentially infinite behaviours, and strong barbed bisimilarity is defined as the largest relation that satisfies the conditions, bisimulation cannot be defined inductively.

\begin{theorem}
  \( \sbbisim{}{} \) is an equivalence relation, that is, the relation is reflexive, symmetric, and transitive.
\end{theorem}

Unfortunately, strong barbed bisimilarity is not a good notion of process equivalence, since it does not consider the environment of processes.
For instance, \( \sbbisim{\Pout{x}{a}{\Pout{y}{b}{\Pend}}}{\Pout{x}{a}{\Pend}} \) since \( \obsout{x} \) is the only observable in both processes and they cannot perform a \( \Atau \)-action, but \( \nsbbisim{\Ppar{\Pout{x}{a}{\Pout{y}{b}{\Pend}}}{\Pin{x}{l}{\Pend}}}{\Pout{x}{a}{\Pend}} \) since the left process can perform a \( \Atau \)-action such that \( \obsout{y} \) becomes observable, whereas the right process cannot.

\subsubsection{Contexts and congruences}
Before we can fix the issue with strong barbed bisimilarity, we need few more definitions.

A \emph{context} is obtained by taking a process and replacing a single occurrence of \( \Pend \) in it with the special \emph{hole} symbol \( \ctxhole \).
As a convention, we do \emph{not} identify \( \alpha \)-convertible contexts.

We can think of contexts as functions between processes.
A context \( C \) can be \emph{applied} to a process \( P \), written \( \applyctx{C}{P} \), by replacing the hole in C by \( P \), thus obtaining another process.
The replacement should be literal, so names and variable names that are free in \( P \) can become bound in \( \applyctx{C}{P} \).

We say that an equivalence relation \( \mathcal{S} \) is a \emph{congruence} if \( (P,Q) \in \mathcal{S} \) implies that for any context \( C \), \( (\applyctx{C}{P}, \applyctx{C}{Q}) \in \mathcal{S} \).

\subsubsection{Strong barbed congruence}
A congruence is exactly what we need to make strong barbed bisimilarity consider the environment (i.e.\ the context) of processes.

We define \emph{strong barbed congruence}, written \( \sbcong{}{} \), by saying that two processes \( P \) and \( Q \) are \emph{strong barbed congruent}, written \( \sbcong{P}{Q} \), if \( \sbbisim{\applyctx{C}{P}}{\applyctx{C}{Q}} \) for every context \( C \).

\begin{lemma}
  \( \sbcong{}{} \) is the largest congruence included in \( \sbbisim{}{} \).
\end{lemma}

\subsubsection{Challenge}
The objective of this challenge is to prove the following theorem:
\begin{theorem}
  \( \sbcong{P}{Q} \) if and only if for any process \( R \) and substitution \( \sigma \), \( \sbbisim{\Ppar{\applysubst{\sigma}{P}}{R}}{\Ppar{\applysubst{\sigma}{Q}}{R}} \).
\end{theorem}

%\paragraph{Notes about this challenge}
%Strong barbed bisimilarity is interesting by itself for this challenge, but the proof that it is an equivalence relation does not require up-to techniques, which are wide spread and interesting.
%The definition of strong barbed congruence requires introducing contexts, but this is actually nice, since they are so prevalent in the field and we want to cover as many basics as possible.
%The theorem about the strong barbed congruence requires up-to techniques, but also seems to involve introducing a structural equivalence, which is possibly not desirable.
%On the other hand, it is a nice application of structural congruence which does not ``ruin'' the semantics.

%% Local Variables:
%% mode: latex
%% TeX-master: "main"
%% End:
