
In mechanised meta-theory, addressing linearity means choosing an
appropriate representation of the linear context. The crux of the matter
is \emph{context splitting}, as showcased by the typing rule for parallel composition.
% the introduction rule of the tensor.
While a linear context is perhaps best seen as a multi-set, most proof
assistants have better support for list processing.  The
latter representation is intuitive, but may require establishing a 
large number of technical lemmata that are
orthogonal to the mathematics of the problem under study, say proving
type preservation for session types. In this space, several designs
are possible: one can label occurrences of resources to constrain
their usage (\eg~\cite{CicconeP20}) or \alb{something --- ask Francisco} as in \cite{Castro2020}. Some authors instead impose a
multiset structure over lists (\eg~\cite{ChaudhuriLR19,Danielsson12});
still, the effort required to develop the infrastructure is significant.

A different approach is to bypass the problem of context splitting by
adopting
% to  meta-theoretic properties
familiar ideas from algorithmic linear type checking: this is known as
 \emph{typing with leftovers}, as exemplified
in~\cite{DBLP:conf/forte/ZalakainD21}. Whatever the choice, list-based
encodings can be refined to be intrinsically-typed, if the proof
assistant does support dependent types (see~\cite{Thiemann2019,CicconeP20,RouvoetPKV20}).

For something completely different, one could adopt a
\emph{sub-structural} meta-logical framework: here the framework
itself handles resource distribution, and users need only map
their linear operations to the ones offered by the framework. In fact,
the linear function space will handle all context operations
implicitly, including splitting and substitution. One such framework
is \emph{Celf}~\cite{Schack-Nielsen:IJCAR08} (see the encoding of
session types in~\cite{Bock2016}). Unfortunately, \emph{Celf} does not
yet fully support the verification of meta-theoretic properties. A compromise is the so called
\emph{two-level} approach, where one encodes a sub-structural
specification logic in a mainstream proof assistant and then uses that
logic to state and prove linear properties (for a recent
example, see~\cite{Felty:MSCS21}).



%%% Local Variables:
%%% mode: latex
%%% TeX-master: "short-paper"
%%% End:
