The encoding of binding constructs has been extensively studied for type systems based on $\lambda$-calculi, but process calculi present unique challenges.
$\pi$-calculi typically include several different binding constructs: inputs bind a received name or value, recursive processes bind recursion variables, and restrictions bind names.
The first two act similarly to the binders known from $\lambda$-calculi, but restrictions are more challenging due to scope extrusion.

The case study \cite{AmbalLS21} compares four approaches to encoding binders in Coq as a first step towards developing tools for working with higher-order process calculi.
They found that working directly with de Bruijn indices was easiest and shortest since the existing approaches developed for $\lambda$-calculus binders do not work well with scope extrusion.
Specifically, the locally nameless representation~\cite{Chargueraud2012} cannot avoid direct manipulation of de Bruijn indices when defining the semantics of scope extrusion; cofinite quantification provides no benefits when working under binders; nominal approaches~\cite{Pitts2003} require explicitly giving and validating sets of free names during scope extrusion; and higher-order abstract syntax (HOAS)~\cite{Pfenning1988} requires additional axioms to work with contexts.

Additionally, many proof assistants lack support for the proof principles required to easily work with scope extrusion, \ie, induction principles on functions to work with HOAS encodings.
Some proof assistants such as Abella~\cite{Baelde2014} support HOAS natively, but working with open terms as required to encode scope extrusion in these systems is generally very challenging~\cite{Momigliano2012}.

% Start of related work
\cite{Maksimovic2015} does not have name restriction, thus dodging the challenge of scope extrusion. \dz{Not sure if relevant to mention.}
