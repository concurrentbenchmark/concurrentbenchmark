\documentclass[runningheads]{llncs}
\synctex=1
\usepackage[utf8]{inputenc}
\usepackage[T1]{fontenc}
\usepackage{amsmath}
\usepackage{amssymb}
\usepackage{graphicx}
\usepackage{mathpartir}
\usepackage{cite}
\usepackage{bm}
\usepackage{hyperref}
\usepackage{cleveref}
\usepackage{hyperref}
\usepackage[inline]{enumitem}
\usepackage{color}
\renewcommand\UrlFont{\color{blue}\rmfamily}

\usepackage[status=draft]{fixme}
\fxsetup{theme=color}

\newcommand{\Pend}{\bm{0}}
\newcommand{\Ppar}[2]{#1 \mid #2}
\newcommand{\Pres}[2]{(\bm{\nu} #1)~#2}
\newcommand{\Pout}[3]{#1 ! #2 . #3}
\newcommand{\Pin}[3]{#1 ? (#2) . #3}
\newcommand{\Pchoice}[2]{#1 + #2}
\newcommand{\Preplicate}[1]{{!}#1}

\newcommand{\freenames}[1]{\textrm{fn}(#1)}
\newcommand{\boundnames}[1]{\textrm{bn}(#1)}
\newcommand{\names}[1]{\textrm{n}(#1)}

\newcommand{\freevars}[1]{\textrm{fv}(#1)}
\newcommand{\boundvars}[1]{\textrm{bv}(#1)}
\newcommand{\vars}[1]{\textrm{vars}(#1)}

\newcommand{\alphacon}[2]{#1 =_{\alpha} #2}
\newcommand{\subst}[3]{#1\{#2/#3\}}

\newcommand{\Aoutf}[2]{#1 ! #2}
\newcommand{\Aoutb}[2]{#1 ! (#2)}
\newcommand{\Ain}[2]{#1 ? #2}
\newcommand{\Atau}{\tau}

\newcommand{\transition}[3]{#1 \xrightarrow{#2} #3}

\newcommand{\observable}[2]{#1\downarrow_{#2}}
\newcommand{\obsin}[1]{#1?}
\newcommand{\obsout}[1]{#1!}

\newcommand\sbullet[1][.5]{\mathbin{\hbox{\scalebox{#1}{$\bullet$}}}}
\newcommand{\sbbisim}[2]{#1 \overset{\sbullet}{\sim} #2}
\newcommand{\nsbbisim}[2]{#1 \not\overset{\sbullet}{\sim} #2}

\newcommand{\ctxhole}{[\cdot]}
\newcommand{\applyctx}[2]{#1[#2]}
\newcommand{\sbcong}[2]{#1 \simeq^c #2}

\newcommand{\applysubst}[2]{#2#1}

\newcommand{\Presd}[3]{(\bm{\nu} #1#2)~#3}
\newcommand{\scong}[2]{#1 \equiv #2}

\newcommand{\reduces}[2]{#1 \rightarrow #2}

\newcommand{\Tend}{\mathbf{end}}
\newcommand{\Tbase}{\mathbf{base}}
\newcommand{\Tin}[1]{{?}.#1}
\newcommand{\Tout}[1]{{!}.#1}

\newcommand{\hastype}[2]{#1 : #2}

\newcommand{\un}[1]{\mathbf{un}(#1)}
\newcommand{\lin}[1]{\mathbf{lin}(#1)}

\newcommand{\Cempty}{\varnothing}
\newcommand{\Cadd}[2]{#1, #2}
\newcommand{\Csplit}[2]{#1 \circ #2}
\newcommand{\Cupdate}[2]{#1 + #2}

\newcommand{\dual}[1]{\overline{#1}}

\newcommand{\types}[2]{#1 \vdash #2}

%% Local Variables:
%% mode: latex
%% TeX-master: "main"
%% End:


\begin{document}

\title{The Concurrent Calculi Formalisation Benchmark}

\author{
     Marco Carbone \inst{1}\orcidID{0000-0001-9479-2632}
\and David Castro-Perez \inst{2}\orcidID{0000-0002-6939-4189}
\and Francisco Ferreira \inst{3}\orcidID{0000-0001-8494-7696}
\and Lorenzo Gheri \inst{4}\orcidID{0000-0002-3191-7722}
\and Frederik Krogsdal Jacobsen \inst{5}\orcidID{0000-0003-3651-8314}
\and Alberto Momigliano \inst{6}\orcidID{0000-0003-0942-4777}
\and Luca Padovani \inst{7}\orcidID{0000-0001-9097-1297}
\and Alceste Scalas \inst{5}\orcidID{0000-0002-1153-6164}
\and Martin Vassor \inst{8}\orcidID{0000-0002-2057-0495}
\and Nobuko Yoshida \inst{8}\orcidID{0000-0002-3925-8557}
\and Daniel Zackon \inst{9}\orcidID{0009-0008-6153-2955}
}

\institute{
     IT University of Copenhagen, Copenhagen, Denmark \email{maca@itu.dk}
\and University of Kent, Canterbury, United Kingdom \email{D.Castro-Perez@kent.ac.uk}
\and Royal Holloway, University of London, Egham, United Kingdom \email{Francisco.FerreiraRuiz@rhul.ac.uk}
\and University of Liverpool, Liverpool, United Kingdom \email{Lorenzo.Gheri@liverpool.ac.uk}
\and Technical University of Denmark, Kgs. Lyngby, Denmark \email{fkjac@dtu.dk}, \email{alcsc@dtu.dk}
\and Università degli Studi di Milano, Milan, Italy \email{momigliano@di.unimi.it}
\and Università di Camerino, Camerino, Italy \email{luca.padovani@unicam.it}
\and University of Oxford, Oxford, United Kingdom \email{martin.vassor@cs.ox.ac.uk}, \email{nobuko.yoshida@cs.ox.ac.uk}
\and McGill University, Montreal, Canada \email{daniel.zackon@mail.mcgill.ca}
}

\authorrunning{M. Carbone et al.}

\maketitle

\begin{abstract}
  The POPLMark challenge and its follow-up POPLMark Reloaded sparked a flurry of
  work on machine-checked proofs, and fostered the adoption of proof
  mechanisation in programming language research.  However, both POPLMark and
  POPLMark Reloaded were purposely limited in scope, with benchmark problems
  that do not address concurrency-related issues.
  %
  For this reason, we propose a new collection of benchmark problems with a
  focus on the challenges and issues that typically arise when mechanising
  message-passing concurrency using process calculi.  Our benchmark problems
  address three key topics: linearity, scope extrusion, and coinductive
  reasoning.  The goal of this new challenge is to clarify, compare, and advance
  the state of the art, fostering the adoption of proof mechanisation in future
  research on message-passing concurrency.

\keywords{Mechanisation \and Process calculi \and Benchmark \and Coinduction \and Scope extrusion \and Linearity}
\end{abstract}

\section{Introduction}


The abstract of the influential POPLMark challenge~\cite{POPLMark} starts with
an important question: \emph{``How close are we to a world where every paper on
programming languages is accompanied by an electronic appendix with
machine-checked proofs?''} The introduction of the POPLMark
challenge spearheaded a shift towards publications that include
mechanised proofs, fostering the progress of proof assistants,
libraries, and best practices.  Later on, the follow-up POPLMark
Reloaded challenge~\cite{POPLMarkReloaded} has encouraged the development of
best practices and tool support for proofs using logical relations.

The authors of both POPLMark and POPLMark Reloaded explicitly
note that their benchmarks were only a beginning.  Furthermore, they
specifically mention reasoning about concurrency using coinduction and
linear environments as points for future work.  In this spirit, in order to
further increase the reach of the POPLMark challenges, we propose a new
collection of benchmark problems specifically designed to address typical issues
that arise when mechanising message-passing concurrency using process calculi.

%(Paragraphs about related work in this vein)
Several papers in the realm of process calculi already include mechanised proofs
(e.g.~\cite{DBLP:conf/pldi/Castro-Perez0GY21,DBLP:conf/tacas/CastroFY20,lmcs:9985,
  DBLP:journals/jar/CruzFilipeMP23, Tirore:2023}), or
discuss proof mechanisation techniques
(e.g.~\cite{DBLP:journals/jar/BengtsonPW16, DBLP:conf/tphol/Gay01,
  DBLP:conf/ppdp/Thiemann19, DBLP:conf/forte/ZalakainD21}).  However, our
experience is that choosing the appropriate proof mechanisation techniques and
tools remains a significant challenge with few guidelines.  This often leads
researchers towards trial-and-error, attempting sub-optimal and ad-hoc
solutions; this increases the overall effort required for proof mechanisation,
and hampers the reuse of techniques and results.
%
Therefore, it is our opinion that the research community in message-passing
concurrency (and more specifically, in process calculi) will benefit from a set
of solutions exploring different ways of mechanising typical problems in the
field.

Like the authors of the previous POPLMark challenges, we seek to
answer a number of questions:
\begin{enumerate}[label=\textbf{(Q\arabic*)},leftmargin=10mm]
\item What is the current state of the art in mechanising process calculi?
\item What techniques and best practices can be recommended when starting formalisation projects within concurrency theory?
\item What improvements are needed to make mechanisation tools more user-friendly with regards to the issues faced when mechanising process calculi?
\end{enumerate}

We have identified three key aspects which typically cause difficulties when
mechanising concurrency theory: \emph{linearity}, \emph{scope extrusion},
% in name-passing systems,
and \emph{coinductive reasoning} for infinite behaviours.
Our benchmark problems are designed to focus on these aspects (discussed
in more detail in \cref{sec:challenge-problems}), with three main goals:

\begin{enumerate}[label=\textbf{(G\arabic*)},leftmargin=10mm]
\item\label{item:goal-comperison-accessibility} Enable the comparison of
  different proof mechanisation approaches, making the challenges accessible to
  mechanisation experts who may not be familiar with concurrency theory.

\item\label{item:goal-tutorials} Encourage the development of guidelines and
  tutorials demonstrating and comparing the many available proof mechanisation
  techniques, libraries, and proof assistant features.

\item\label{item:goal-reusability} Prioritise the exploration of mechanisation
  techniques that may be useful and reusable for future research.
\end{enumerate}

We also aim at strengthening the culture of mechanisation, by rallying the
community to collaborate on exploring and developing new tools and techniques.

The three aspects we selected (linearity, scope extrusion, coinduction) are of
course not the only ones that may cause issues in mechanisations --- but they
are fundamental to concurrency theory, and they emerge in most mechanisations.
The mechanisation of novel research results often requires addressing more than
just one of these aspects at once --- and we see the combination of techniques
as a next step beyond the scope of this challenge (we discuss this in
\cref{sec:going-beyond}).

% Our challenge problems are designed to stimulate progress on answering these questions by providing simple environments in which to demonstrate, evaluate, and compare the many tools and techniques available.
% While the challenge problems themselves are thus not of scientific interest, we expect that the best practices and mechanisations developed by solving the challenge problems will also be useful for future research.

We have begun collecting solutions to our challenge problems on our website:
%
\begin{center}
  \url{https://concurrentbenchmark.github.io/}
\end{center}
%
In the longer term, we expect to use the website for promoting best practices,
tutorials and guidelines derived from solutions to our challenge problems.
%The interested reader can find more information about the current status at:
We encourage anyone interested to try the challenge problems using their
favourite tools and techniques, and to send us their solutions.

\paragraph{Outline of this paper.}
In \cref{sec:challenge-problems} we describe in more detail the three key
aspects of our challenge problems.  We discuss our design choices in
\cref{sec:design-discussion}, and possible extensions and further topics in
\cref{sec:going-beyond}.  We conclude in \cref{sec:conclusion}.
The full challenge problems, including proofs, are available in \cref{app:challenges}.

\section{The challenge problems}\label{sec:challenge-problems}
In this section we briefly describe the three aspects of our challenge problems, referring to the appendix for the full text of each one.

\subsection{Linearity}
Our \textbf{linearity challenge} (Appendix~\ref{sec:challenge:linearity-beh-types}) is to prove type preservation (also known as subject reduction) for a process calculus with a minimal
session typing system (without delegation of communication channels, to avoid overlapping with the scope
extrusion challenge below).
Session type systems require linear handling of typing contexts, and the key
issue of this challenge is reasoning about the linearity of context splitting operations.

\subsection{Scope extrusion}
In name-passing systems, scope extrusion is the expansion of the scope of a name (i.e.~a communication channel) that is sent from one process to another.
Our \textbf{scope extrusion challenge} (Appendix~\ref{sec:challenge:name-passing-scope-extrusion}) is to prove the equivalence between the reduction semantics and the labelled transition system (LTS) semantics of a process calculus (up-to structural congruence).
The key issue of this challenge is reasoning about names that are ``in the process'' of being scope-extruded, which often presents difficulties for the mechanisation of binders.

\subsection{Coinduction}
Our \textbf{coinduction challenge} (Appendix~\ref{sec:challenge:coinduction}) is to prove that \emph{strong barbed bisimilarity} is a congruence for a limited calculus with (infinite) replication of processes.
The key issue of this challenge is the coinductive reasoning about the infinite behaviours of the replication operator.

\section{Design of the benchmark}\label{sec:design-discussion}

In this section, we outline the factors considered in designing the
benchmark challenges. We start with some general remarks, then
describe the individual design considerations for each challenge
problem.

% We have identified three key aspects fundamental to mechanisations of concurrent calculi and designed our challenge problems to exercise these.
Like the previous POPLMark and POPLMark Reloaded challenges, our challenge is not meant to be comprehensive: applications such as multiparty session types, choreographies, conversation types, psi-calculi, or encodings between different calculi are not directly covered.  Still, these (and other) proof mechanisation applications will need the basic techniques that our challenge problems exercise:
as per our design goal \ref{item:goal-reusability},
our problems are drawn from the basic metatheory of process calculi (instead
of requiring the development of new theory), and focus on well-known theorems
and results that involve interesting proof techniques that may be reused in further work.

To achieve our design goal \ref{item:goal-comperison-accessibility},
we have designed our challenge problems to exercise the three aspects (linearity, scope extrusion, coinduction) independently, and they can be solved individually and in any order.
We have designed each problem to be small and easy to understand with basic knowledge of textbook concurrency theory, process calculi and type theory.  The process calculus used in each challenge
focuses on the features we want to exercise, and we have omitted all constructs
(such as choices) that would complicate the mechanisation without bringing tangible
benefits and insights.  (Re-adding the omitted constructs if of course possible, but we leave
such extensions as a possibility for future benchmark challenge.)

The minimality and uniformity of the calculi also allows us to target our
design goal \ref{item:goal-tutorials}, i.e.~encouraging the development of
concurrency theory proof mechanisation guidelines and tutorials.

% While our challenges require reasoning about binders, which was the main topic of the original POPLMark, they also require other techniques.
% Our challenges also require reasoning about logical relations which goes above the reasoning involved in POPLMark Reloaded.

% Our challenge problems do not include reasoning about encodings. While
% this kind of reasoning is common in the field, it does not require any
% different reasoning techniques in a concurrency setting than in any
% other setting.

\subsection{Linearity}
- Linearity type system dual vs. Vasco's?
- Linearity: the context is polymorphic
- Asynchrony? Not necessary, that is a bit more advanced

For our challenge on linear reasoning we have chosen a safety theorem
for a session type system since this allows us to involve linear
reasoning without many definitions. Accordingly, we have chosen not to
include channel delegation and to use a reduction instead of a
transition system semantics.

Linearity is of course interesting in many other subfields of programming languages, and we have thus chosen to base our challenge on a  session type system primarily due to the common use of session types in concurrency theory.

Inspired by Vasconcelos~\cite{Vasconcelos2012}, we use a syntax where
restriction binds two variables. This presentation makes the duality
of the two names obvious in the type system.

We have simple notion of well-formedness for our challenge problem,
and does not for instance consider deadlocked processes. While more
sophisticated notions of well-formedness are also interesting, the
proofs of these properties are more complicated and would thus make
the challenge problem more difficult and less focused on the linear
aspects of the system.

\subsection{Scope extrusion}
- Why do we keep the restrictions around in the LTS semantics in scope extrusion?

This is the challenge most closely related to the original POPLMark
challenge since it concerns the properties of binders in restrictions.
It is not clear that the techniques used to solve the POPLMark challenges about binders can be easily extended to reasoning about scope extrusion.
Notably, techniques like Higher Order Abstract Syntax, or well-scoped
De\ Bruijn indices do not have the concept of free variables. It is
not immediately obvious how to encode the operational semantics in
this kind of system.


\subsection{Coinduction}
- Coinduction: why is congruence interesting?
- ``Proper'' barbs with a structural congruence in coinduction instead of LTS semantics

Our coinduction challenge concerns strong barbed bisimulation and congruence.
Though weak barbed congruence is a more common behavioural equivalence for \(\pi\)-terms, we avoid having to abstract over the number of internal transitions when considering the strong notions of equivalence, thus simplifying the theory.

We have chosen not to include delegation in the coinduction challenge since it is orthogonal to our primary aim of exploring coinductive proof techniques.

\subsection{Evaluation criteria}
Solutions to our challenge problems should be compared on three measures: the mechanisation overhead, this being the amount of manually written infrastructure and setup needed to express the definitions in the mechanisation; the adequacy of the formal statements in the mechanisation, i.e.\ whether the proven theorems are easily recognisable as the theorems from the challenge; and the cost of entry for the tools and techniques employed, i.e.\ the difficulty of learning to use the techniques.
Solutions to our challenges do not need to strictly follow the definitions and lemmas set out in the problem text, but solutions which deviate from the original challenges need to present more elaborate argumentation for their adequacy.


\section{Going beyond the challenge problems}\label{sec:going-beyond}

As with the challenges that came before, this one is intended to be
useful on its own, but not to exhaustively cover every possible
problem in the area. The current challenge can be extended in two
dimensions: the first is taking the existing challenges further.
And the second is devising new challenges not covered by the current
benchmarks.

The proposed challenges try to minimise the overlap of the techniques
they exercise, however, most systems will combine some or all aspects
of the challenges. An interesting second round of formalisations could
entail combining the features of the challenges. In the following
table we consider the following three additions: First the addition of
choice to the type level and the process level. Second, the addition
of recursion (and recursion types for typed systems). And finally, the
addition of channel delegation to those calculi that do not already
support it.

The following table shows which extension should be suitable (that we
mark with $+$) for each challenge:

\vspace{.5em}

\begin{center}\small
  \begin{tabular}{|r|c|c|c|}
    \hline
    & Linearity & \shortstack{Scope \\ extrusion} & Coinduction \\
    \hline
    Choice & $+$ & $+$  & $+$ \\
    \hline
    \shortstack{Recursive \\ types} & $+$  & $+$ & \\
    \hline
    Delegation & $+$ & & $+$ \\
    \hline
  \end{tabular}
\end{center}

\vspace{.5em}

The other dimension to extend these calculi corresponds to systems
that are not covered by our challenges. Extending in this dimension
means proposing new challenges to cover for different features of
message passing calculi, or to cover different features of mechanised
proof.

Some interesting aspects of message passing calculi to explore in
further challenges could be: Multiparty session types, and
choreographies because their definitions and meta-theory have elements
that we do not cover here, however the elements that we do cover will
appear in these problems. Also, while we explore a form of barbed
bisimilarity one could easily design a challenge exploring different
notions of bisimilarity (barbed, or weak, etc.), and trace
equivalence.

On the side of proof assistants, the current challenge proposes
mechanising the proofs of theorems. An interesting venue to explore
would be to take advantage of other aspects of proof assistants. For
example, proof assistants are often able to produce certified code (by
using code extraction, or compiling and running their definitions).
Additionally, while having effective automation in our current
challenge is possible, a challenge could be designed with the explicit
objective of automating aspects of the proofs. And finally, a
challenge could propose the integration with other formal reasoning
tools, namely model checkers. These tools are in common use in the
field. However, combining automated proofs with proof assistants
offers the potential to ease the path for attempting larger proofs
with less effort.

Ultimately, the current challenges can be extended in several
worthwhile directions, and we look forward to a future when they
indeed are extended. Moreover, we see the current challenge as setting
the foundation for those future extensions. It is our expectation that
the solutions of those hypothetical challenges will build on solutions
to this challenge.

\section{Conclusion}\label{sec:conclusion}

\subsubsection*{Acknowledgements}

\appendix
\section{Challenges}\label{app:challenges}
\documentclass[a4paper]{article}

\usepackage{amsmath}
\usepackage{amssymb}
\usepackage{amsthm}
\usepackage[utf8]{inputenc}
\usepackage{mathpartir}
\usepackage{bm}
\usepackage{graphicx}

\newtheorem{lemma}{Lemma}
\newtheorem{theorem}{Theorem}

\newcommand{\Pend}{\bm{0}}
\newcommand{\Ppar}[2]{#1 \mid #2}
\newcommand{\Pres}[2]{(\bm{\nu} #1)~#2}
\newcommand{\Pout}[3]{#1 ! #2 . #3}
\newcommand{\Pin}[3]{#1 ? (#2) . #3}
\newcommand{\Pchoice}[2]{#1 + #2}
\newcommand{\Preplicate}[1]{{!}#1}

\newcommand{\freenames}[1]{\textrm{fn}(#1)}
\newcommand{\boundnames}[1]{\textrm{bn}(#1)}
\newcommand{\names}[1]{\textrm{n}(#1)}

\newcommand{\alphacon}[2]{#1 =_{\alpha} #2}
\newcommand{\subst}[3]{#1\{#2/#3\}}

\newcommand{\Aoutf}[2]{#1 ! #2}
\newcommand{\Aoutb}[2]{#1 ! (#2)}
\newcommand{\Ain}[2]{#1 ? #2}
\newcommand{\Atau}{\tau}

\newcommand{\transition}[3]{#1 \xrightarrow{#2} #3}

\newcommand{\observable}[2]{#1\downarrow_{#2}}
\newcommand{\obsin}[1]{#1?}
\newcommand{\obsout}[1]{#1!}

\newcommand\sbullet[1][.5]{\mathbin{\hbox{\scalebox{#1}{$\bullet$}}}}
\newcommand{\sbbisim}[2]{#1 \overset{\sbullet}{\sim} #2}

\begin{document}

\section{Challenge: Scope Extrusion}
The idea of this challenge is to formalize a proof that requires scope extrusion.
Scope extrusion is the notion that a process can send restricted names to another process, as long as the restriction can safely be ``extruded'' (i.e.\ expanded) to include the recieving process.
This e.g.\ allows a process to set up a private connection by sending a restricted name to another process, then using this name for further communication.

The setting for this challenge is an untyped \( \pi \)-calculus, the syntax and semantics of which are presented below.
The objective of this challenge is to prove that barbed bisimulation for this calculus is an equivalence relation.

\subsection{Syntax}
We assume the existence of some type of \emph{names}, values of which we will denote by \( x, y, \dots \).
The syntax is then:
\begin{align*}
  P,Q :=&&& \Pend \\
  |&&& \Pout{x}{y}{P} \\
  |&&& \Pin{x}{y}{P} \\
  |&&& \Ppar{P}{Q} \\
  |&&& \Pres{x}{P} \\
  |&&& \Pchoice{P}{Q}
\end{align*}
The process \( \Pend \) is \emph{inaction}: a process which can do nothing.
The process \( \Pout{x}{y}{P} \) is an \emph{output}, which can send the name \( y \) via \( x \), then continue as \( P \).
The process \( \Pin{x}{y}{P} \) is an \emph{input}, which can receive a name via \( x \), then continue as \( P \) with the received name substituted for \( y \).
The input operator thus binds the name \( y \) in \( P \).
The process \( \Ppar{P}{Q} \) is the \emph{composition} of process \( P \) and process \( Q \).
The two components can proceed independently of each other, or they can interact via shared names.
The process \( \Pres{x}{P} \) is the \emph{restriction} of the name \( x \) to \( P \).
Components in \( P \) can use the name \( x \) to interact with each other, but not with processes outside of the restriction.
The restriction operator thus binds the name \( x \) in \( P \).
Note that the scope of a restriction may change when processes interact.
The process \( \Pchoice{P}{Q} \) is a non-deterministic \emph{choice} between continuing as the process \( P \) or as the process \( Q \).
Note that there is no recursion or replication in the syntax, and thus no infinite behaviours can be expressed.

We will use the notation \( \freenames{P} \) to denote the set of names that occur free (i.e.\ not bound by a restriction or an input) in \( P \).
We will use the notation \( \boundnames{P} \) to denote the set of names that occur bound (by a restriction or an input) in \( P \).
We will use the notation \( \subst{P}{x}{y} \) to denote the process \( P \) with \( x \) substituted for \( y \).

Two processes \( P \) and \( Q \) are \( \alpha \)-convertible, written \( \alphacon{P}{Q} \), if \( Q \) can be obtained from \( P \) by a finite number of substitutions of bound names.
As a convention, we will identify \( \alpha \)-convertible processes.

As a convention, we assume that the bound names occurring in any collection of processes are chosen to be different from the free names occurring in those processes and from the names occurring in any substitutions applied to the processes.
This is justified because any overlapping names may be \( \alpha \)-converted such that the assumption is satisfied.

\subsection{Semantics}
The semantics of the system describe the actions that the system can perform by defining a labelled transition relation on processes.
The transitions are labelled by \emph{actions}, the syntax of which are as follows:
\begin{align*}
  \alpha := &&& \Aoutf{x}{y} \\
  |&&& \Ain{x}{y} \\
  |&&& \Aoutb{x}{y} \\
  |&&& \Atau
\end{align*}
The \emph{free output action} \( \Aoutf{x}{y} \) is sending the name \( y \) via \( x \).
The \emph{input action} \( \Ain{x}{y} \) is receiving the name \( y \) via \( x \).
The \emph{bound output action} \( \Aoutb{x}{y} \) is sending a fresh name \( y \) via \( x \).
The \emph{internal action} \( \Atau \) is performing some unobservable action, e.g.\ internal communication.

We will again use the notation \( \freenames{\alpha} \) to denote the set of names that occur free in the action \( \alpha \) and the notation \( \boundnames{\alpha} \) to denote the set of names that occur bound in the action \( \alpha \).
In the free output action \( \Aoutf{x}{y} \) and the input action \( \Ain{x}{y} \), both \( x \) and \( y \) are free names.
In the bound output action \( \Aoutb{x}{y} \), \( x \) is a free name, while \( y \) is a bound name.
We will further use the notation \( \names{\alpha} \) to denote the union of \( \freenames{\alpha} \) and \( \boundnames{\alpha} \), i.e.\ the set of all names that occur in the action \( \alpha \).

The transition relation is then defined by the following rules:
\begin{mathpar}
  \inferrule[Out]{ }{\transition{\Pout{x}{y}{P}}{\Aoutf{x}{y}}{P}}
  \and
  \inferrule[In]{ }{\transition{\Pin{x}{z}{P}}{\Ain{x}{y}}{\subst{P}{y}{z}}}
  \and
  \inferrule[Sum-L]{\transition{P}{\alpha}{P'}}{\transition{\Pchoice{P}{Q}}{\alpha}{P'}}
  \and
  \inferrule[Sum-R]{\transition{P}{\alpha}{P'}}{\transition{\Pchoice{Q}{P}}{\alpha}{P'}}
  \and
  \inferrule[Par-L]{\transition{P}{\alpha}{P'} \\ \boundnames{\alpha} \cap \freenames{Q} = \emptyset}{\transition{\Ppar{P}{Q}}{\alpha}{\Ppar{P'}{Q}}}
  \and
  \inferrule[Par-R]{\transition{Q}{\alpha}{Q'} \\ \boundnames{\alpha} \cap \freenames{P} = \emptyset}{\transition{\Ppar{P}{Q}}{\alpha}{\Ppar{P}{Q'}}}
  \and
  \inferrule[Comm-L]{\transition{P}{\Aoutf{x}{y}}{P'} \\ \transition{Q}{\Ain{x}{y}}{Q'}}{\transition{\Ppar{P}{Q}}{\Atau}{\Ppar{P'}{Q'}}}
  \and
  \inferrule[Comm-R]{\transition{P}{\Ain{x}{y}}{P'} \\ \transition{Q}{\Aoutf{x}{y}}{Q'}}{\transition{\Ppar{P}{Q}}{\Atau}{\Ppar{P'}{Q'}}}
  \and
  \inferrule[Close-L]{\transition{P}{\Aoutb{x}{z}}{P'} \\ \transition{Q}{\Ain{x}{z}}{Q'} \\ z \notin \freenames{Q}}{\transition{\Ppar{P}{Q}}{\tau}{\Pres{z}{\Ppar{P'}{Q'}}}}
  \and
  \inferrule[Open]{\transition{P}{\Aoutf{x}{z}}{P'} \\ z \neq x}{\transition{\Pres{z}{P}}{\Aoutb{x}{z}}{P'}}
  \and
  \inferrule[Close-R]{\transition{P}{\Ain{x}{z}}{P'} \\ \transition{Q}{\Aoutb{x}{z}}{Q'} \\ z \notin \freenames{P}}{\transition{\Ppar{P}{Q}}{\tau}{\Pres{z}{\Ppar{P'}{Q'}}}}
  \and
  \inferrule[Res]{\transition{P}{\alpha}{P'} \\ z \notin \names{\alpha}}{\transition{\Pres{z}{P}}{\alpha}{\Pres{z}{P'}}}
\end{mathpar}
Note that there is no rule for inferring transitions from \( \Pend \), and that there is no rule for inferring an action of an input or output process except those that match the input/output capability.
Note also that due to rule \TirName{In}, the process \( \Pin{x}{z}{P} \) can receive \emph{any} name.

As a convention, we assume that bound names of any processes or actions are chosen to be different from the names that occur free in any other entities under consideration, such as processes, actions, substitutions, and sets of names.
The convention has one exception, namely that in the transition \( \transition{P}{\Aoutb{x}{z}}{Q} \), the name \( z \) (which occurs bound in \( P \) and the action \( \Aoutb{x}{z} \)) may occur free in \( Q \).
Without this exception it would be impossible to express scope extrusion.

\subsection{Bisimilarity}
Our notion of process equivalence relations builds on a notion of \emph{observables}.
If we allow ourselves only to observe internal transitions, we will relate either too few processes (in the strong case where we relate only processes with exactly the same number of internal transitions) or every process (in the weak case where we relate processes with any amount of internal transitions).
We must therefore allow ourselves to observe more, and here choose to define the observables of a process as the names it can use for sending and receiving.

To this end, we define the \emph{observability predicate} \( \observable{P}{\mu} \) by the following rules:
\begin{align}
  \observable{P}{\obsin{x}}  &\quad \textrm{if \( P \) can perform an input action via \( x \).} \\
  \observable{P}{\obsout{x}} &\quad \textrm{if \( P \) can perform an output action via \( x \).}
\end{align}

Two processes \( P \) and \( Q \) are then \emph{strongly barbed bisimilar}, written \( \sbbisim{P}{Q} \), if:
\begin{gather}
  \observable{P}{\mu}~\textrm{implies}~\observable{Q}{\mu} \\
  \transition{P}{\Atau}{P'}~\textrm{implies}~\transition{Q}{\Atau}{Q'}~\textrm{for some \( Q' \) with}~\sbbisim{P'}{Q'} \\
  \observable{Q}{\mu}~\textrm{implies}~\observable{P}{\mu} \\
  \transition{Q}{\Atau}{Q'}~\textrm{implies}~\transition{P}{\Atau}{P'}~\textrm{for some \( P' \) with}~\sbbisim{P'}{Q'}
\end{gather}
Note that, since our calculus does not have infinite behaviour, we do not specify that the relation is the largest of its kind, and the relation is thus not coinductively defined.

\subsection{Challenge}
The objective of this challenge is to prove the following theorem:
\begin{theorem}
  \( \sbbisim{}{} \) is an equivalence relation, that is, the relation is reflexive, symmetric, and transitive.
\end{theorem}

\section{Challenge: Coinduction}
The idea of this challenge is to formalize a proof that requires coinductive techniques.
Coinduction is a proof technique for infinite structures, which arise in this context due to systems with behaviours that continue indefinitely.
Coinduction is the dual of induction: whereas induction is useful for proving properties of least fixed points, coinduction is useful for proving properties of greatest fixed points.

The setting for this challenge is an untyped calculus of communicating systems with replication of processes, the syntax and semantics of which are presented below.
The objective of this challenge is to prove that strong barbed bisimulation for this calculus is an equivalence relation.

\section{Challenge: Linearity}
The setting for this challenge is a restricted \( \pi \)-calculus with a session type system.

The objective of this challenge is to prove type preservation.

\section{Bonus challenges}

\subsection{Linearity and Scope Extrusion}
The setting for this challenge is the \( \pi \)-calculus with a session type system.

The objective of this challenge is to prove type preservation.

\subsection{Scope Extrusion and Coinduction}
The setting for this challenge is a simply typed \( \pi \)-calculus.

The objective of this challenge is to prove bisimulation.

\subsection{Coinduction and Linearity}
The setting for this challenge is the Calculus of Communicating Systems with a session type system.

The objective of this challenge is to prove type preservation.

\section{Future work}
\begin{itemize}
\item Multiparty session types
\item Choreography
\item Encodings
\item Code extraction
\end{itemize}

\end{document}

\bibliographystyle{splncs04}
\bibliography{../references}

\end{document}
