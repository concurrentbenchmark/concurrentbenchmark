\documentclass[runningheads]{llncs}
\synctex=1
\usepackage[utf8]{inputenc}
\usepackage[T1]{fontenc}
\usepackage{amsmath}
\usepackage{amssymb}
\usepackage{mathtools}
\usepackage{graphicx}
\usepackage{mathpartir}
\usepackage{cite}
\usepackage{bm}
\usepackage{hyperref}
\usepackage{cleveref}
\usepackage{hyperref}
\usepackage[inline]{enumitem}
\usepackage{color}
\renewcommand\UrlFont{\color{blue}\rmfamily}

\usepackage[status=draft]{fixme}
\fxsetup{theme=color}

\newcommand{\Pend}{\bm{0}}
\newcommand{\Ppar}[2]{#1 \mid #2}
\newcommand{\Pres}[2]{(\bm{\nu} #1)~#2}
\newcommand{\Pout}[3]{#1 ! #2 . #3}
\newcommand{\Pin}[3]{#1 ? (#2) . #3}
\newcommand{\Pchoice}[2]{#1 + #2}
\newcommand{\Preplicate}[1]{{!}#1}

\newcommand{\freenames}[1]{\textrm{fn}(#1)}
\newcommand{\boundnames}[1]{\textrm{bn}(#1)}
\newcommand{\names}[1]{\textrm{n}(#1)}

\newcommand{\freevars}[1]{\textrm{fv}(#1)}
\newcommand{\boundvars}[1]{\textrm{bv}(#1)}
\newcommand{\vars}[1]{\textrm{vars}(#1)}

\newcommand{\alphacon}[2]{#1 =_{\alpha} #2}
\newcommand{\subst}[3]{#1\{#2/#3\}}

\newcommand{\Aoutf}[2]{#1 ! #2}
\newcommand{\Aoutb}[2]{#1 ! (#2)}
\newcommand{\Ain}[2]{#1 ? #2}
\newcommand{\Atau}{\tau}

\newcommand{\transition}[3]{#1 \xrightarrow{#2} #3}

\newcommand{\observable}[2]{#1\downarrow_{#2}}
\newcommand{\obsin}[1]{#1?}
\newcommand{\obsout}[1]{#1!}

\newcommand\sbullet[1][.5]{\mathbin{\hbox{\scalebox{#1}{$\bullet$}}}}
\newcommand{\sbbisim}[2]{#1 \overset{\sbullet}{\sim} #2}
\newcommand{\nsbbisim}[2]{#1 \not\overset{\sbullet}{\sim} #2}

\newcommand{\ctxhole}{[\cdot]}
\newcommand{\applyctx}[2]{#1[#2]}
\newcommand{\sbcong}[2]{#1 \simeq^c #2}

\newcommand{\applysubst}[2]{#2#1}

\newcommand{\Presd}[3]{(\bm{\nu} #1#2)~#3}
\newcommand{\scong}[2]{#1 \equiv #2}

\newcommand{\reduces}[2]{#1 \rightarrow #2}

\newcommand{\Tend}{\mathbf{end}}
\newcommand{\Tbase}{\mathbf{base}}
\newcommand{\Tin}[1]{{?}.#1}
\newcommand{\Tout}[1]{{!}.#1}

\newcommand{\hastype}[2]{#1 : #2}

\newcommand{\un}[1]{\mathbf{un}(#1)}
\newcommand{\lin}[1]{\mathbf{lin}(#1)}

\newcommand{\Cempty}{\varnothing}
\newcommand{\Cadd}[2]{#1, #2}
\newcommand{\Csplit}[2]{#1 \circ #2}
\newcommand{\Cupdate}[2]{#1 + #2}

\newcommand{\dual}[1]{\overline{#1}}

\newcommand{\types}[2]{#1 \vdash #2}

%% Local Variables:
%% mode: latex
%% TeX-master: "main"
%% End:


\begin{document}

\title{The Concurrent Calculi Formalisation Benchmark}

\maketitle

\begin{abstract}
  The POPLMark challenge and its follow-up POPLMark Re\-loaded sparked a flurry of
  work on machine-checked proofs, and fostered the adoption of proof
  mechanisation in programming language research.  However, both POPLMark and
  POPLMark Reloaded were purposely limited in scope, with benchmark problems
  that do not address concurrency-related issues.
  %
  For this reason, we propose a new collection of benchmark problems with a
  focus on the challenges and issues that typically arise when mechanising
  message-passing concurrency using process calculi.  Our benchmark problems
  address three key topics: linearity, scope extrusion, and coinductive
  reasoning.  The goal of this new challenge is to clarify, compare, and advance
  the state of the art, fostering the adoption of proof mechanisation in future
  research on message-passing concurrency.

\keywords{Mechanisation \and Process calculi \and Benchmark \and Coinduction \and Scope extrusion \and Linearity}
\end{abstract}

\section{Introduction}
The abstract of the influential POPLMark challenge~\cite{POPLMark} starts with
an important question: \emph{``How close are we to a world where every paper on
programming languages is accompanied by an electronic appendix with
machine-checked proofs?''} The introduction of the POPLMark
challenge spearheaded a shift towards publications that include
mechanised proofs, fostering the progress of proof assistants,
libraries, and best practices.  Later on, the follow-up POPLMark
Reloaded challenge~\cite{POPLMarkReloaded} has encouraged the development of
best practices and tool support for proofs using logical relations.

The authors of both POPLMark and POPLMark Reloaded explicitly
note that their benchmarks were only a beginning.  Furthermore, they
specifically mention reasoning about concurrency using coinduction and
linear environments as points for future work.  In this spirit, in order to
further increase the reach of the POPLMark challenges, we propose a new
collection of benchmark problems specifically designed to address typical issues
that arise when mechanising message-passing concurrency using process calculi.

Several papers in the realm of process calculi already include mechanised proofs
(e.g.~\cite{DBLP:conf/pldi/Castro-Perez0GY21,DBLP:conf/tacas/CastroFY20,lmcs:9985,
  DBLP:journals/jar/CruzFilipeMP23, Tirore:2023}), or
discuss proof mechanisation techniques
(e.g.~\cite{DBLP:journals/jar/BengtsonPW16, DBLP:conf/tphol/Gay01,
  DBLP:conf/ppdp/Thiemann19, DBLP:conf/forte/ZalakainD21}).  However, our
experience is that choosing the appropriate proof mechanisation techniques and
tools remains a significant challenge with few guidelines.  This often leads
researchers towards trial-and-error, attempting sub-optimal and ad-hoc
solutions; this increases the overall effort required for proof mechanisation,
and hampers the reuse of techniques and results.
%
Therefore, it is our opinion that the research community in message-passing
concurrency (and more specifically, in process calculi) will benefit from a set
of solutions exploring different ways of mechanising typical problems in the
field.
Like the authors of the POPLMark challenges, we seek to
answer a number of questions:
\begin{enumerate}[label=\textbf{(Q\arabic*)},leftmargin=10mm]
\item\label{item:rq1} What is the current state of the art in mechanising process calculi?
\item\label{item:rq2} What techniques and best practices can be recommended when starting formalisation projects within concurrency theory?
\item\label{item:rq3} What improvements are needed to make mechanisation tools more user-friendly with regards to the issues faced when mechanising process calculi?
\end{enumerate}
We have identified three key aspects which typically cause difficulties when
mechanising concurrency theory: \emph{linearity}, \emph{scope extrusion},
and \emph{coinductive reasoning} for infinite behaviours.
Our benchmark problems are designed to focus on these aspects (discussed
in more detail in \cref{sec:challenge-problems}), with three main goals:

\begin{enumerate}[label=\textbf{(G\arabic*)},leftmargin=10mm]
\item\label{item:goal-comperison-accessibility} Enable the comparison of
  different proof mechanisation approaches, making the challenges accessible to
  mechanisation experts who may not be familiar with concurrency theory.

\item\label{item:goal-tutorials} Encourage the development of guidelines and
  tutorials demonstrating and comparing the many available proof mechanisation
  techniques, libraries, and proof assistant features.

\item\label{item:goal-reusability} Prioritise the exploration of mechanisation
  techniques that may be useful and reusable for future research.
\end{enumerate}
We also aim at strengthening the culture of mechanisation, by rallying the
community to collaborate on exploring and developing new tools and techniques.

The three aspects we selected (linearity, scope extrusion, coinduction) are of
course not the only ones that may cause issues in mechanisations --- but they
are fundamental to concurrency theory, and they emerge in most mechanisations.
The mechanisation of novel research results often requires addressing more than
just one of these aspects at once --- and we see the combination of techniques
as a next step beyond the scope of this challenge (we discuss this in
\cref{sec:going-beyond}).

% Our challenge problems are designed to stimulate progress on answering these questions by providing simple environments in which to demonstrate, evaluate, and compare the many tools and techniques available.
% While the challenge problems themselves are thus not of scientific interest, we expect that the best practices and mechanisations developed by solving the challenge problems will also be useful for future research.

We have begun collecting solutions to our challenges on our website:
%
\begin{center}
  \url{https://concurrentbenchmark.github.io/}
\end{center}
%
In the longer term, we expect to use the website for promoting best practices,
tutorials and guidelines derived from solutions to our challenges.
We encourage anyone interested to try the challenges using their
favourite tools and techniques, and to send us their solutions.

\paragraph{Outline of this paper.}
In \cref{sec:challenge-problems} we describe our three challenges.
We discuss our design choices in \cref{sec:design-discussion}, and possible extensions and further topics in \cref{sec:going-beyond}.

\section{The challenges}\label{sec:challenge-problems}
\label{sec:prelim}
First, we list some common notions and conventions that we use in the
challenges. Since the calculi under study are somewhat different, each section
lists the changes that apply. The challenges are inspired by simplified versions of calculi (and proofs) from~\cite{picalcbook} and~\cite{Vasconcelos2012}.

We assume the existence of some set of \emph{base values}, represented by the symbols
\( a, b, \dots \), the existence of some set
of \emph{variables}, represented by the symbols
\( l, m, \dots \), and the existence of some set of \emph{names},
represented by the symbols \( x, y, \dots \).%
\footnote{%
  Unlike the standard $\pi$-calculus, we distinguish variables from names to
  better control the expressiveness of the calculi under study, and the scope of
  the corresponding challenges: the key distinction is that names are used as communication channels (and can be sent and received in the scope extrusion challenge), whereas variables are only bound by inputs and cannot be restricted, sent nor received, in the style of value-passing CCS \cite{Milner1989}.%
} %
We assume that all of these sets are infinite and that their elements can be
compared for equality.

The syntax of processes includes:
the process \( \Pend \) or \emph{inaction}, a process which can do nothing. The process \( \Ppar{P}{Q} \) is the \emph{parallel composition} of process \( P \) and process \( Q \).
The two components can proceed independently of each other, or they can interact via shared names.
%The process \( \Pchoice{P}{Q} \) is a non-deterministic \emph{choice} between continuing as the process \( P \) or as the process \( Q \).

For communication, processes include \emph{input} and \emph{output}, whose
signature depends on the calculus being value-passing or name-passing.
We use here the metavariables $c,k$ to abstract over this choice
--- i.e.~$c$ may be either a value or a variable or a name, whereas $k$ may
be a variable or a name.
The process \( \Pout{x}{c}{P} \) is an \emph{output}, which can send
\( c \) via \( x \), then continue as \( P \).  % The intention is that
% the value \( v \) must be a base value when it is actually sent, and
% this will be enforced in the semantics later on
The process \( \Pin{x}{k}{P} \) is an \emph{input}, which can receive a $c$
via \( x \), then continue as \( P \) with the received element
substituted for \( k \).  The input operator thus
binds \( k \) in \( P \).

The process \( \Pres{x}{P} \) is the \emph{restriction} of the name
\( x \) to \( P \), binding \( x \) in \( P \).


The process \( \Preplicate{P} \) is the \emph{replication} of the process \( P \).
It can be thought of as the infinite composition \( \Ppar{P}{\Ppar{P}{\cdots}} \).
Replication makes it possible to express infinite behaviours.

We use the notation \( \freenames{P} \) to denote the set of
names that occur free,
% (i.e.\ not bound by a restriction) in \( P \),
\( \boundnames{P} \) to denote the set of names that occur bound
% (by a restriction)
in \( P \) and  \( \freevars{P} \)
to denote the set of variables that occur free
% (i.e.\ not bound by an input)
in \( P \).  We use the notation \( \boundvars{P} \)
for the set of variables that occur bound
% (by an input)
in \( P \).  We use the notation \( \subst{P}{a}{l} \) to denote
the process \( P \) with base value \( a \) substituted for variable
\( l \). Similarly, \( \subst{P}{x}{y} \) denotes the process
\( P \) with name \( x \) substituted for name \( y \).
We use the notation \( \applysubst{\sigma}{P} \) to denote the process \( P \) with a finite number of arbitrary substitutions applied to it.

Two processes \( P \) and \( Q \) are \( \alpha \)-convertible,
written \( \alphacon{P}{Q} \), if \( Q \) can be obtained from \( P \)
by a finite number of substitutions of bound variables.  As a
convention, we identify \( \alpha \)-convertible processes and we
assume that bound names and bound variables of any processes are
chosen to be different from the names and variables that occur
free in any other entities under consideration, such as processes,
substitutions, and sets of names or variables.  This is justified
because any overlapping names and variables may be
\( \alpha \)-converted such that the assumption is satisfied.


A \emph{context} is obtained by taking a process and replacing a single occurrence of \( \Pend \) in it with the special \emph{hole} symbol \( \ctxhole \).
As a convention, we do \emph{not} identify \( \alpha \)-convertible contexts.
%
A context acts as a function between processes:
a context \( C \) can be \emph{applied} to a process \( P \), written \( \applyctx{C}{P} \), by replacing the hole in C by \( P \), thus obtaining another process.
The replacement is literal, so names and variables that are free in \( P \) can become bound in \( \applyctx{C}{P} \).

We say that an equivalence relation \( \mathcal{S} \) is a \emph{congruence} if \( (P,Q) \in \mathcal{S} \) implies that for any context \( C \), \( (\applyctx{C}{P}, \applyctx{C}{Q}) \in \mathcal{S} \).

\subsection{Challenge: Linearity and behavioral type systems}
\label{sec:challenge:linearity-beh-types}
This challenge formalises a proof that requires reasoning about linearity of channels.
Linearity is the notion that a channel must be used exactly once in a process.
This is necessary to prove properties about session type systems, and the key
issue of this challenge is reasoning about the linearity of context splitting operations.
Linear reasoning is also necessary to formalise e.g.\ linear and affine types for the pi-calculus and cut elimination in linear logics.

The setting for this challenge is a small calculus with a session type
system, the syntax and semantics of which are given below. The
calculus is a fragment of the one presented in~\cite{Vasconcelos2012},
formulated in the dual style of~\cite{barber96tr}.

The main objective of this challenge is to prove type preservation (also
known as subject reduction), i.e.\ that well-typed
processes can only transition to processes which are also well-typed
in the same context.
The second objective is to prove type safety, i.e.\ that well-typed processes are also well-formed in the sense that they do not use endpoints in a non-dual way.

\subsubsection{Syntax.}
The syntax is given by the grammar
\[
\begin{array}{r@{\qquad}c@{\qquad}l}
  v,w & \Coloneqq & a \quad\mid\quad l \\
   P,Q & \Coloneqq & \Pend \quad\mid\quad \Pout{x}{v}{P} \quad\mid\quad \Pin{x}{l}{P} \quad\mid\quad \PBpar{P}{Q} \quad\mid\quad  \Presd{x}{y}{P}
\end{array}
\]
where a \emph{value} \( v, w, \dots \) is either a base value \( a \) or a variable \( l \).

The output process \( \Pout{x}{v}{P} \) sends the value \( v \) via \( x \) and then continues as \( P \).
The intention is that the value \( v \) must be a base value when it is actually sent, and this is enforced in the semantics later on.
The input process \( \Pin{x}{l}{P} \) waits for a base value from \( x \) and then continues as \( P \) with the received value substituted for the variable \( l \).
%
The process \( \Presd{x}{y}{P} \) represents a \emph{session} with endpoints named \( x \) and \( y \) which are bound in \( P \). In \( P \), the names \( x \) and \( y \) can be used to exchange messages over the session (sending on \( x \) and receiving on \( y \) or vice versa).
%
%Finally, the process $\PBpar{P}{Q}$ represents the parallel composition of $P$ and $Q$.
%
Note that in this calculus channels cannot be sent in messages, therefore the topology of the communication network described by a process cannot change.
Also, there is no recursion or replication in the syntax, hence no infinite behaviors can be expressed. In particular, we only model linear (as opposed to shared) channels.

\subsubsection{Semantics.}
We describe the actions that the system can perform through a small step operational semantics.
As usual, we use  a \emph{structural congruence} relation that equates processes that we deem to be indistinguishable. Structural congruence is the smallest congruence relation that satisfies the following axioms:
\begin{mathpar}
  \inferrule[Sc-Par-Comm]{ }{\scong{\Ppar{P}{Q}}{\Ppar{Q}{P}}}
  \and
  \inferrule[Sc-Par-Assoc]{ }{\scong{\Ppar{(\Ppar{P}{Q})}{R}}{\Ppar{P}{(\Ppar{Q}{R})}}}
  \and
  \inferrule[Sc-Par-Inact]{ }{\scong{\Ppar{P}{\Pend}}{P}}
  \and
  \inferrule[Sc-Res-Par]{\{x,y\} \cap \freenames{Q} = \emptyset}{\scong{\Ppar{\Presd{x}{y}{P}}{Q}}{\Presd{x}{y}{(\Ppar{P}{Q})}}}
  \and
  \inferrule[Sc-Res-Inact]{ }{\scong{\Presd{x}{y}{\Pend}}{\Pend}}
  \and
  \inferrule[Sc-Res]{ }{\scong{\Presd{x_1}{y_1}{\Presd{x_2}{y_2}{P}}}{\Presd{x_2}{y_2}{\Presd{x_1}{y_1}{P}}}}
\end{mathpar}

The operational semantics is defined as the following relation on processes:
\begin{mathpar}
  \inferrule[R-Com]{ }{\reduces{\Presd{x}{y}{(\Ppar{\Pout{x}{a}{P}}{\Ppar{\Pin{y}{l}{Q}}{R}})}}{\Presd{x}{y}{(\Ppar{P}{\Ppar{\subst{Q}{a}{l}}{R}})}}}
  \and
  \inferrule[R-Res]{\reduces{P}{Q}}{\reduces{\Presd{x}{y}{P}}{\Presd{x}{y}{Q}}}
  \and
  \inferrule[R-Par]{\reduces{P}{Q}}{\reduces{\Ppar{P}{R}}{\Ppar{Q}{R}}}
  \and
  \inferrule[R-Struct]{\scong{P}{P'} \\ \reduces{P'}{Q'} \\ \scong{Q}{Q'}}{\reduces{P}{Q}}
\end{mathpar}

Note that reductions are allowed only for restricted pairs of session endpoints. This makes it possible to formulate subject reduction so that the typing context is exactly the same before and after each reduction.
%
Note also that due to rule \TirName{R-Com}, the process \( \Pin{y}{l}{P} \) can receive \emph{any} base value.
On the other hand, since the rule \TirName{R-Com} only applies to sending base values, there is no way to send a variable or a name.

\subsubsection{Session types.}
Our process syntax allows us to write processes that are ill formed in
the sense that they either use the endpoints bound by a restriction to
communicate in a way that does not follow the intended duality, or
attempt to send something which is not a base value.  As an example,
the process
\( \Presd{x}{y}{(\Ppar{\Pout{x}{a}{\Pend}}{\Pout{y}{a}{\Pend}})} \)
attempts to send a base value on both \( x \) and \( y\), whereas one
of the names should be used for receiving in order to guarantee
progress.  Another example is the process
\( \Presd{x}{y}{(\Ppar{\Pout{x}{l}{\Pend}}{\Pin{y}{l}{\Pend}})} \),
which attempts to send a variable that is not instantiated at the time
of sending.
%
To prevent these issues, we introduce a \emph{session type system} which
rules out ill-formed processes.

\subsubsection{Syntax.}
Our type system does not type processes directly, but instead focuses on the channels used in the process.
The syntax of \emph{session types} \( S, T \), unrestricted type contexts \( \Gamma \) and linear type contexts \( \Delta \) is as follows:
\[
  \begin{array}{r@{\qquad}c@{\qquad}l}
  S,T & \Coloneqq & \Tend \quad\mid\quad \Tbase \quad\mid\quad \Tin{S} \quad\mid\quad \Tout{S} \\
    \Gamma & \Coloneqq & \Cempty \quad\mid\quad \Gamma, l \\
                    \Delta &::= & \Cempty \quad\mid\quad \Cadd{\Delta}{\hastype{x}{S}}
  \end{array}
\]
The \emph{end type} \( \Tend \) describes an endpoint on which no further interaction is possible.
The \emph{base type} \( \Tbase \) describes base values.
The \emph{input type} \( \Tin{S} \) describes endpoints used for receiving a value and then according to \( S \).
The \emph{output type} \( \Tout{S} \) describes endpoints used for sending a value and then according to \( S \).

Typing contexts gather type information about names and variables.
\emph{Unrestricted} contexts are simply sets of names since we only have one
base type. \emph{Linear contexts} associate a type to endpoints. We use
the comma as split/union, overloaded to singletons, and \( \Cempty \) as the
empty context. We extend the Barendregt convention to contexts, so that all
entries are distinct.  Note that the order in which information is added to a
type context does not matter.

Since we need to determine whether endpoints are used in complementary ways to determine whether processes are well formed, we need to formally define the dual of a type as follows:
\begin{mathpar}
  \inferrule{}{\dual{\Tin{S}} = \Tout{\dual{S}}}
  \and
  \inferrule{}{\dual{\Tout{S}} = \Tin{\dual{S}}}
  \and
  \inferrule{}{\dual{\Tend} = \Tend}
\end{mathpar}
Note that the dual function is partial since it is undefined for the base type.

\subsubsection{Typing rules.}
Our type system is aimed at maintaining two invariants:
\begin{enumerate}
\item No endpoint is used simultaneously by parallel processes;
\item The two endpoints of the same session have dual types.
\end{enumerate}
The first invariant is maintained by linearly splitting type contexts when typing compositions of processes, the second by requiring duality when typing restrictions.

We have two typing judgments: one for values, and one for processes.
The typing rules for values are:
\begin{mathpar}
  \inferrule[T-Base]{ }{\typev{\Gamma}{\hastype{a}{\Tbase}}} \and
  \inferrule[T-Var]{ }{\typev{\Cadd{\Gamma}l }{\hastype{l}{\Tbase}}}
\end{mathpar}
% In the typing rule for names, $\tend \Phi$ denotes that all types in the range of $\Phi$ are $\Tend$:
% \begin{mathpar}
%   \inferrule[T-Name]{{\tend\Delta }}{\types{{\Gamma; (\Delta,\hastype{x}{T})}}{\hastype{x}{T}}}
% \end{mathpar}
The typing rules for processes are as follows:
\begin{mathpar}

  \inferrule[T-Inact]{\tend\Delta }{\types{\Gamma;\Delta}{\Pend}}
  \and
  \inferrule[T-Par]{\types{\Gamma;\Delta_1}{P} \\ \types{\Gamma;\Delta_2}{Q}}
  {\types{\Gamma; \Csplit{\Delta_1}{\Delta_2}}{\Ppar{P}{Q}}}
  \and
  \inferrule[T-Res]{\types{\Gamma; (\Cadd{\Cadd{\Delta}{\hastype{x}{T}}}{\hastype{y}{\dual{T}}}}{P})}{\types{\Gamma}{\Presd{x}{y}{P}}}
  \and
    \inferrule[T-Out]{
      \typev{\Gamma}{\hastype{v}{\Tbase}} \\ \types{\Gamma; \Cupdate{\Delta}{\hastype{x}{T}}}{P}}{\types{\Gamma; (\Csplit{\Delta}{\hastype{x}{\Tout{T}}})}{\Pout{x}{v}{P}}}
    \and
    \inferrule[T-IN]{
      \types{(\Gamma,  l ); (\Cupdate{\Delta}{\hastype{x}{T}})}{P}}{\types{\Gamma; (\Csplit{\Delta}{\hastype{x}{\Tin{T}}})}{\Pin{x}{l}{P}}}
\end{mathpar}
Note that we do not need a judgment for typing channels, since it is already folded into the \TirName{T-In} and \TirName{T-Out} rules.

\subsubsection{Challenge.}
The objective of this challenge is to prove subject reduction and type safety for our calculus with session types. We start with:
\begin{theorem}[Subject reduction]
  If \( \types{\Gamma;\Delta}{P} \) and \( \reduces{P}{Q} \) then \( \types{\Gamma;\Delta}{Q} \).
\end{theorem}

The following lemmas are needed:
\begin{lemma}[Weakening]\mbox{}
  \label{lemma:weak}
  \begin{enumerate}
  \item If \( \types{\Gamma; \Delta}{P} \) then
    \( \types{(\Cadd{\Gamma}{l});\Delta}{P} \).
      \item If \( \types{\Gamma; \Delta}{P} \) % and \( \tend{\Delta} \)
        then
    \( \types{\Gamma;(\Cadd{\Delta}{\hastype{x}{\Tend}})}{P} \).
  \end{enumerate}
  % \begin{metanote}
  %   Note from Dawit: Does \( \Delta \) really have to be only ends to be weakened? I had expected that if \( P \) is typed by \( \Delta \) and \( \Delta' \) is only ends, then \( P \) is typed by \( \Cadd{\Delta}{\Delta'} \).

  %   AM:  I agree that it's weird: we should not be able to weaken a linear context, if not
  %   with unrestricted types --- but this is what we need for Sc-Res-Inact to be type preserving:
  %   from $\Delta\vdash 0$ we need $\Delta\vdash (\nu x y)0$. BTW, this is also a problem in Vasco's system
  % \end{metanote}
\end{lemma}

\begin{lemma}[Strengthening]\mbox{}
  \label{lemma:strenD}
  If \( \types{\Gamma; (\Delta, x : T)}{P} \) and $x\not\in \freenames{P}$, then \( \types{\Gamma; \Delta}{P} \).
\end{lemma}

\begin{lemma}[Substitution]\mbox{}
  If $\types{(\Gamma, l);\Delta}{P}$ and ${\typev{\Gamma}{\hastype{a}{\Tbase}}}$ then
  \( \types{\Gamma;\Delta}{\subst{P}{l}{a}} \).
\end{lemma}

\begin{lemma}[Preservation for $\equiv$]
  If \( \types{\Gamma;\Delta}{P} \) and \( \scong{P}{Q} \) then \( \types{\Gamma;\Delta}{Q} \).
\end{lemma}

To formulate safety, we need to formally define what we mean by well-formed process.
We say that a process \( P \) is \emph{prefixed at variable \( x \)} if it is of the form \( \Pout{x}{v}{P} \) or \( \Pin{x}{l}{P} \).
A process is then \emph{well formed} if, for each of its structurally congruent processes of the form \( \Presd{x_1}{y_1}{\dots \Presd{x_n}{y_n}{(\Ppar{\Ppar{P}{Q}}{R})}} \), with \( n \geq 0 \), it holds that, if \( P \) is prefixed at \( x_1 \) and \( Q \) is prefixed at \( y_1 \), then \( \Ppar{P}{Q} \) is of the form \( \Ppar{\Pout{x_1}{a}{P'}}{\Pin{y_1}{l}{Q'}} \).

Note that well-formed processes do not necessarily reduce. For example, the process:
\begin{equation*}
  \Presd{x_1}{y_1}{\Presd{x_2}{y_2}{(\Ppar{\Pout{x_1}{a}{\Pin{y_2}{l}{\Pend}}}{\Pout{y_2}{x_2}{\Pin{y_1}{l}{\Pend}}})}}
\end{equation*}
is well formed but also irreducible.

\begin{theorem}[Type safety]
  If \( \types{\Cempty}{P} \), then \( P \) is well formed.
\end{theorem}

\begin{corollary}
  If \( \types{\Cempty}{P} \) and \( \reduces{P}{Q} \) then \( Q \) is well formed.
\end{corollary}

\subsection{Challenge: Name passing and scope extrusion}
\label{sec:challenge:name-passing-scope-extrusion}

This challenge formalises a proof that requires explicit scope extrusion.
Scope extrusion is the notion that a process can send restricted names to another process, as long as the restriction can safely be ``extruded'' (i.e.\ expanded) to include the receiving process.
This e.g.\ allows a process to set up a private connection by sending a restricted name to another process, then using this name for further communication
The key issue of this challenge is reasoning about names that are ``in the process'' of being scope-extruded, which often presents difficulties for the mechanisation of binders.

Reasoning about scope extrusion explicitly can sometimes be avoided by introducing a structural congruence rule into the semantics, but doing this means we lose information about the scope when reasoning about the semantics.
Explicitly reasoning about scope extrusion is necessary to describe e.g.\ runtime monitors and compositions of systems.

The setting for this challenge is a ``classic'' untyped \( \pi \)-calculus, where (unlike the calculi in the other challenges) names can be sent and received, and bound by input constructs (similarly to variables in the other calculi).
We define two different semantics for our system: one that avoids explicit reasoning about scope extrusion, and one that does not.
The objective of this challenge is to prove that the two semantics are equivalent up to structural congruence.

\subsubsection{Syntax.}
The syntax of processes is given by:
\[
\begin{array}{r@{\qquad}c@{\qquad}l}
  P,Q & \Coloneqq & \Pend \quad\mid\quad \PBpar{P}{Q} \quad\mid\quad \Pout{x}{y}{P} \quad\mid\quad \Pin{x}{y}{P} \quad\mid\quad \Pres{x}{P}
\end{array}
\]


%The process \( \Pend \) is \emph{inaction}: a process which can do nothing.
The process \( \Pout{x}{y}{P} \) is an \emph{output}, which can send the name \( y \) via \( x \), then continue as \( P \).
The process \( \Pin{x}{y}{P} \) is an \emph{input}, which can receive a name via \( x \), then continue as \( P \) with the received name substituted for \( y \).
The input operator thus binds the name \( y \) in \( P \).
% The process \( \Ppar{P}{Q} \) is the \emph{composition} of process \( P \) and process \( Q \).
% The two components can proceed independently of each other, or they can interact via shared names.
% The process \( \Pres{x}{P} \) is the \emph{restriction} of the name \( x \) to \( P \).
% Components in \( P \) can use the name \( x \) to interact with each other, but not with processes outside of the restriction.
% The restriction operator thus binds the name \( x \) in \( P \).
Note that the scope of a restriction may change when processes interact. Namely, a restricted name may be sent \emph{outside} of its scope.
% The process \( \Pchoice{P}{Q} \) is a non-deterministic \emph{choice} between continuing as the process \( P \) or as the process \( Q \).
Note that there is no recursion or replication in the syntax, and thus no infinite behaviors can be expressed.
This simplifies the theory and is orthogonal to the concept of scope extrusion.

% We will use the notation \( \freenames{P} \) to denote the set of names that occur free (i.e.\ not bound by a restriction or an input) in \( P \).
% We will use the notation \( \boundnames{P} \) to denote the set of names that occur bound (by a restriction or an input) in \( P \).
% We will use the notation \( \subst{P}{x}{y} \) to denote the process \( P \) with \( x \) substituted for \( y \).

% Two processes \( P \) and \( Q \) are \( \alpha \)-convertible, written \( \alphacon{P}{Q} \), if \( Q \) can be obtained from \( P \) by a finite number of substitutions of bound names.
% As a convention, we will identify \( \alpha \)-convertible processes.

% As a convention, we assume that the bound names occurring in any collection of processes are chosen to be different from the free names occurring in those processes and from the names occurring in any substitutions applied to the processes.
% This is justified because any overlapping names may be \( \alpha \)-converted such that the assumption is satisfied.

\subsubsection{Reduction semantics.}
The first semantics is an operational reduction semantics, which avoids reasoning explicitly about scope extrusion by way of a structural congruence rule.
\emph{Structural congruence} is the smallest congruence relation that satisfies the following axioms:

\begin{mathpar}
  \inferrule[Sc-Par-Assoc]{ }{\scong{\Ppar{P}{(\Ppar{Q}{R})}}{\Ppar{(\Ppar{P}{Q})}{R}}}
  \and
  \inferrule[Sc-Par-Comm]{ }{\scong{\Ppar{P}{Q}}{\Ppar{Q}{P}}}
  \and
  \inferrule[Sc-Par-Inact]{ }{\scong{\Ppar{P}{\Pend}}{P}}
  \and
  \inferrule[Sc-Res-Par]{x \notin \freenames{Q}}{\scong{\Ppar{\Pres{x}{P}}{Q}}{\Pres{x}{(\Ppar{P}{Q})}}}
  \and
  \inferrule[Sc-Res-Inact]{ }{\scong{\Pres{x}{\Pend}}{\Pend}}
  \and
  \inferrule[Sc-Res]{ }{\scong{\Pres{x}{\Pres{y}{P}}}{\Pres{y}{\Pres{x}{P}}}}
\end{mathpar}

The operational semantics is defined as the following relation on processes:
\begin{mathpar}
  \inferrule[R-Com]{ }{\reduces{\Ppar{\Pout{x}{y}{P}}{\Pin{x}{z}{Q}}}{\Ppar{P}{\subst{Q}{y}{z}}}}
  \and
  \inferrule[R-Res]{\reduces{P}{Q}}{\reduces{\Pres{x}{P}}{\Pres{x}{Q}}}
  \and
  \inferrule[R-Par]{\reduces{P}{Q}}{\reduces{\Ppar{P}{R}}{\Ppar{Q}{R}}}
  \and
  \inferrule[R-Struct]{\scong{P}{P'} \\ \reduces{P'}{Q'} \\ \scong{Q}{Q'}}{\reduces{P}{Q}}
\end{mathpar}
Note that there is no rule for inferring an action of an input or output process except those that match the input/output capability.
Note also that due to rule \TirName{R-Com}, the process \( \Pin{x}{z}{P} \) can receive \emph{any} name.
Finally, note that rule \TirName{R-Struct} allows for applying the structural congruence both before and after the reduction:
this makes the reduction relation closed under structural congruence.

\subsubsection{Transition system semantics.}
The second semantics of the system describe the actions that the system can perform by defining a labelled transition relation on processes.
The transitions are labelled by \emph{actions}, the syntax of which is:
\[
\begin{array}{r@{\qquad}c@{\qquad}l}
  \alpha & \Coloneqq & \Aoutf{x}{y} \quad\mid\quad \Ain{x}{y} \quad\mid\quad \Aoutb{x}{y} \quad\mid\quad \Atau
\end{array}
\]

The \emph{free output action} \( \Aoutf{x}{y} \) is sending the name \( y \) via \( x \).
The \emph{input action} \( \Ain{x}{y} \) is receiving the name \( y \) via \( x \).
The \emph{bound output action} \( \Aoutb{x}{y} \) is sending a fresh name \( y \) via \( x \).
The \emph{internal action} \( \Atau \) is performing some unobservable action, e.g.\ internal communication.

We extend the notion of free and bound occurrences with
\( \freenames{\alpha} \) to denote the set of names that occur free in
the action \( \alpha \) and  \( \boundnames{\alpha} \) to
denote the set of names that occur bound in the action \( \alpha \).
In the free output action \( \Aoutf{x}{y} \) and the input action
\( \Ain{x}{y} \), both \( x \) and \( y \) are free names.  In the
bound output action \( \Aoutb{x}{y} \), \( x \) is a free name, while
\( y \) is a bound name.  We also use the notation
\( \names{\alpha} \) to denote the union of \( \freenames{\alpha} \)
and \( \boundnames{\alpha} \), i.e.\ the set of all names that occur
in the action \( \alpha \).
The transition relation is then defined by the following rules:
\begin{mathpar}
  \inferrule[Out]{ }{\transition{\Pout{x}{y}{P}}{\Aoutf{x}{y}}{P}}
  \and
  \inferrule[In]{ }{\transition{\Pin{x}{z}{P}}{\Ain{x}{y}}{\subst{P}{y}{z}}}
  \and
  \inferrule[Par-L]{\transition{P}{\alpha}{P'} \\ \boundnames{\alpha} \cap \freenames{Q} = \emptyset}{\transition{\Ppar{P}{Q}}{\alpha}{\Ppar{P'}{Q}}}
  \and
  \inferrule[Par-R]{\transition{Q}{\alpha}{Q'} \\ \boundnames{\alpha} \cap \freenames{P} = \emptyset}{\transition{\Ppar{P}{Q}}{\alpha}{\Ppar{P}{Q'}}}
  \and
  \inferrule[Comm-L]{\transition{P}{\Aoutf{x}{y}}{P'} \\ \transition{Q}{\Ain{x}{y}}{Q'}}{\transition{\Ppar{P}{Q}}{\Atau}{\Ppar{P'}{Q'}}}
  \and
  \inferrule[Comm-R]{\transition{P}{\Ain{x}{y}}{P'} \\ \transition{Q}{\Aoutf{x}{y}}{Q'}}{\transition{\Ppar{P}{Q}}{\Atau}{\Ppar{P'}{Q'}}}
  \and
  \inferrule[Close-L]{\transition{P}{\Aoutb{x}{z}}{P'} \\ \transition{Q}{\Ain{x}{z}}{Q'} \\ z \notin \freenames{Q}}{\transition{\Ppar{P}{Q}}{\tau}{\Pres{z}{\Ppar{P'}{Q'}}}}
  \and
  \inferrule[Open]{\transition{P}{\Aoutf{x}{z}}{P'} \\ z \neq x}{\transition{\Pres{z}{P}}{\Aoutb{x}{z}}{P'}}
  \and
  \inferrule[Close-R]{\transition{P}{\Ain{x}{z}}{P'} \\ \transition{Q}{\Aoutb{x}{z}}{Q'} \\ z \notin \freenames{P}}{\transition{\Ppar{P}{Q}}{\tau}{\Pres{z}{\Ppar{P'}{Q'}}}}
  \and
  \inferrule[Res]{\transition{P}{\alpha}{P'} \\ z \notin \names{\alpha}}{\transition{\Pres{z}{P}}{\alpha}{\Pres{z}{P'}}}
\end{mathpar}
Note that there is no rule for inferring transitions from \( \Pend \), and that there is no rule for inferring an action of an input or output process except those that match the input/output capability.
Note also that due to rule \TirName{In}, the process \( \Pin{x}{z}{P} \) can receive \emph{any} name.

We keep the convention that bound names of any processes or actions
are chosen to be different from the names that occur free in any other
entities under consideration, such as processes, actions,
substitutions, and sets of names.  The convention has one exception,
namely that in the transition \( \transition{P}{\Aoutb{x}{z}}{Q} \),
the name \( z \) (which occurs bound in \( P \) and the action
\( \Aoutb{x}{z} \)) may occur free in \( Q \).  Without this exception
it would be impossible to express scope extrusion.

Before getting to our challenge, we prove some intermediate results.
\begin{lemma}\label{se-lemma-harmony-fact}
  If \( \scong{P}{Q} \) and \( \transition{P}{\alpha}{P'} \), then for some \( Q' \) we have \( \transition{Q}{\alpha}{Q'} \) and \( \scong{P'}{Q'} \).
\end{lemma}

For the proof of the challenge theorem, we introduce the notion of a \emph{normalized derivation} of a reduction \( \reduces{P}{Q} \), which is of the following form.
The first rule applied is \TirName{R-Com}. The derivation continues with an application of \TirName{R-Par}, followed by zero or more applications of \TirName{R-Res}.
The last rule is an application of \TirName{R-Struct}.
\begin{lemma}\label{se-lemma-normalized-derivation}
  Every reduction has a normalized derivation.
\end{lemma}

\begin{lemma}\label{se-lemma-normalized-reduction}
  If \( \reduces{P}{Q} \) then there are \( x, y, z, z_1, \dots, z_n, R_1, R_2, \) and \( S\) such that
  \begin{align*}
    &\scong{P}{\Pres{z_1}{\!\dots \Pres{z_n}{(\Ppar{(\Ppar{\Pout{x}{y}{R_1}}{\Pin{x}{z}{R_2}})}{S})}}} \\
    &\scong{Q}{\Pres{z_1}{\!\dots \Pres{z_n}{(\Ppar{(\Ppar{R_1}{\subst{R_2}{y}{z}})}{S})}}}
  \end{align*}
\end{lemma}

\subsubsection{Challenge.}
The objective of this challenge is to prove the following theorems, which together show the equivalence between the reduction semantics and the transition system semantics up to structural congruence.
The first of the theorems involves reasoning about scope extrusion more directly than the other, and if time does not permit proving both of the theorems, \cref{thm:se-trans-implies-red} should be proven first.
\begin{theorem}\label{thm:se-trans-implies-red}
  \( \transition{P}{\Atau}{Q} \) implies \( \reduces{P}{Q} \).
\end{theorem}

\begin{theorem}
  \( \reduces{P}{Q} \) implies the existence of a \( Q' \) such that \( \transition{P}{\Atau}{Q'} \) and \( \scong{Q}{Q'} \).
\end{theorem}

\subsection{Challenge: Coinduction and infinite processes}
\label{sec:challenge:coinduction}
This challenge is a about the mechanisation of proofs  concerning infinite behaviours.
% formalises a proof
% that require coinduction.
% coinductive techniques.
%
% Coinduction is a proof technique for infinite structures,
This is usually connected to \emph{coinductive} definitions where an infinite
structure is defined as the greatest fixed point of a recursive
definition.
%
Coinduction is  a technique for defining and proving properties of
such infinite structures,
%
which, in the context of the concurrent benchmark, arise
% in this context
due to systems with behaviors that continue indefinitely.
%
Coinduction is the dual of induction: whereas induction is useful for
proving properties of least fixed points, coinduction is useful for
proving properties of greatest fixed points.
%
% \marco{Should we be more specific here and give a deeper explanation
%   of what coinduction is? If not, then why do we even say that it's
%   the dual of induction?}
%

For this challenge, we adopt a fragment of the untyped $\pi$-calculus
that includes process replication.
% The setting for this challenge is an untyped calculus of
% communicating systems with replication of processes, the syntax and
% semantics of which are presented below.
The objective of this challenge is to % prove % a context theorem for
draw a formal connection between strong barbed congruence and strong
barbed bisimilarity.  % I.e., The context lemma
The result establishes that two processes are strong barbed congruent
if the processes obtained by applying a finite number of substitutions to
them and composing them in parallel with an arbitrary process are
strongly barbed bisimilar.
The key issue of this challenge is the coinductive reasoning about the infinite behaviours of the replication operator.

\subsubsection{Syntax.}
% We assume the existence of some type of \emph{base values}, values
% of which we will denote by \( a, b, \dots \), the existence of some
% type of \emph{variables}, values of which we will denote by
% \( l, m, \dots \), and the existence of some type of \emph{names},
% values of which we will denote by \( x, y, \dots \).
The syntax of the calculus is identical to that of
Section~\ref{sec:challenge:linearity-beh-types} (challenge on
linearity), except from restriction, now unary, and the addition of
replication:
% for including replication and having unary restriction:
\begin{displaymath}
  \begin{array}{r@{\qquad}c@{\qquad}l}
    v,w & \Coloneqq & a \quad\mid\quad l \\
    P,Q & \Coloneqq & \Pend
               \quad \mid\quad \Pout{x}{v}{P}
               \quad \mid\quad \Pin{x}{l}{P}
               \quad \mid\quad \PBpar{P}{Q}
               \quad \mid\quad \Pres{x}{}{P}
               \quad \mid\quad !P
  \end{array}
\end{displaymath}
Note that, unlike the process calculus from
Section~\ref{sec:challenge:linearity-beh-types}, communications occur
over the same channel name. Moreover, since replication $!P$ allows
for infinite copies of the process $P$, processes can dynamically
create an infinite number of names during execution when combined with
restriction $\Pres{x}{}{P}$.
%
% \begin{align*}
%   v,w :=&&& a \\
%   |&&& l \\
%   P,Q :=&&& \Pend \\
%   |&&& \Pout{x}{v}{P} \\
%   |&&& \Pin{x}{l}{P} \\
%   |&&& \Ppar{P}{Q} \\
%   |&&& \Pres{x}{P} \\
%   |&&& \Preplicate{P}
% \end{align*}
% A \emph{value} \( v, w, \dots \) is either a base value \( a \) or a
% variable \( l \).  The process \( \Pend \) is \emph{inaction}:
% a process which can do nothing.  The process \( \Pout{x}{a}{P} \) is
% an \emph{output}, which can send the value \( v \) via \( x \), then
% continue as \( P \).  The intention is that the value \( v \) must
% be a base value when it is actually sent, and this will be enforced
% in the semantics later on.  The process \( \Pin{x}{l}{P} \) is an
% \emph{input}, which can receive a base value via \( x \), then
% continue as \( P \) with the received value substituted for the
% variable \( l \).  The input operator thus binds the variable
%  \( l \) in \( P \).  The process \( \Ppar{P}{Q} \) is the
% \emph{composition} of process \( P \) and process \( Q \).  The two
% components can proceed independently of each other, or they can
% interact via shared names.  The process \( \Pres{x}{P} \) is the
% \emph{restriction} of the name \( x \) to \( P \).  Components in
% \( P \) can use the name \( x \) to interact with each other, but
% not with processes outside of the restriction.  The restriction
% operator thus binds the name \( x \) in \( P \).  Note that the
% scope of a restriction may not change when processes interact, since
% it is only possible to send and receive values, and not names.  The
% process \( \Preplicate{P} \) is the \emph{replication} of the
% process \( P \).  It can be thought of as the infinite composition
% \( \Ppar{P}{\Ppar{P}{\cdots}} \).  Replication makes it possible to
% express infinite behaviours.

% We will use the notation \( \subst{P}{v}{l} \) to denote the process
% \( P \) with value \( v \) substituted for variable \( l \).

% % We will use the notation \( \subst{P}{v}{w} \) to denote the process \( P \) with value \( v \) substituted for value \( w \).
% Two processes \( P \) and \( Q \) are \( \alpha \)-convertible,
% written \( \alphacon{P}{Q} \), if \( Q \) can be obtained from
% \( P \) by a finite number of substitutions of bound variables.
% As a convention, we will identify \( \alpha \)-convertible
% processes.

% Also as a convention, we assume that the bound names and bound
% variables occurring in any collection of processes are chosen
% to be different from the free names and free variables
% occurring in those processes and from the names and variables
% occurring in any substitutions applied to the processes.  This is
% justified because any overlapping names and variables may be
% \( \alpha \)-converted such that the assumption is satisfied.

\subsubsection{Semantics.} We choose to give an LTS semantics for this
challenge. As an alternative, it is also possible to adopt a reduction
semantics which features fewer rules but needs a structural
congruence. %  The semantics of the system describes the actions that the
% system can perform by defining a labelled transition relation on
% processes. 
The transitions are labelled by \emph{actions}, the syntax
of which is as follows:
% \begin{align*}
%   \alpha := &&& \Aoutf{x}{a} \\
%   |&&& \Ain{x}{a} \\
%   |&&& \Atau
% \end{align*}
\begin{align*}
  \alpha\ \ \Coloneqq\ \ \Aoutf{x}{a} \ \mid\ \Ain{x}{a} \ \mid\ \Atau
\end{align*}
The \emph{output action} \( \Aoutf{x}{y} \) is sending the base value
\( a \) via \( x \).  The \emph{input action} \( \Ain{x}{y} \) is
receiving the base value \( y \) via \( x \).  The \emph{internal
  action} \( \Atau \) is performing some unobservable action, e.g.\
internal communication.
%
We use the notation \( \names{\alpha} \) to denote the set of
names that occur in the action \( \alpha \).
The transition relation is defined by the following rules:
\begin{mathpar}
  \inferrule[Out]{ }{\transition{\Pout{x}{a}{P}}{\Aoutf{x}{a}}{P}}
  \and
  \inferrule[In]{ }{\transition{\Pin{x}{l}{P}}{\Ain{x}{a}}{\subst{P}{a}{l}}}
  \and
  \inferrule[Par-L]{\transition{P}{\alpha}{P'}}{\transition{\Ppar{P}{Q}}{\alpha}{\Ppar{P'}{Q}}}
  \and
  \inferrule[Par-R]{\transition{Q}{\alpha}{Q'}}{\transition{\Ppar{P}{Q}}{\alpha}{\Ppar{P}{Q'}}}
  \and
  \inferrule[Comm-L]{\transition{P}{\Aoutf{x}{a}}{P'} \\ \transition{Q}{\Ain{x}{a}}{Q'}}{\transition{\Ppar{P}{Q}}{\Atau}{\Ppar{P'}{Q'}}}
  \and
  \inferrule[Comm-R]{\transition{P}{\Ain{x}{a}}{P'} \\ \transition{Q}{\Aoutf{x}{a}}{Q'}}{\transition{\Ppar{P}{Q}}{\Atau}{\Ppar{P'}{Q'}}}
  \and
  \inferrule[Res]{\transition{P}{\alpha}{P'} \\ x \notin \names{\alpha}}{\transition{\Pres{x}{P}}{\alpha}{\Pres{x}{P'}}}
  \and
  \inferrule[Rep]{\transition{P}{\alpha}{P'}}{\transition{\Preplicate{P}}{\alpha}{\Ppar{P'}{\Preplicate{P}}}}
\end{mathpar}
Note that there is no rule for inferring transitions from \( \Pend \), and that there is no rule for inferring an action of an input or output process except those that match the input/output capability.
Note also that due to rule \TirName{In}, the process \( \Pin{x}{l}{P} \) can receive \emph{any} base value.
On the other hand, since the rule \TirName{Out} only applies to base values, there is no way to send a variable.

% As a convention, we assume that bound names and bound variables of any processes or actions are chosen to be different from the names and variables that occur free in any other entities under consideration, such as processes, actions, substitutions, and sets of names or variables.

\subsubsection{Bisimilarity.}
Bisimilarity is a notion of equivalence for processes and
% .  Our notion of process equivalence relations
builds on the idea of \emph{observables}, i.e., what we can externally
observe from the semantics of a process. If we allowed ourselves only
to observe internal transitions (i.e., observe that a process is
internally performing a step of computation) we would relate either
too few processes (in the strong case where we relate only processes
with exactly the same number of internal transitions) or every process
(in the weak case where we relate processes with any amount of
internal transitions).  As a result, we must allow ourselves to
monitor more than just internal transitions, and we choose to describe
a process's observables as the names it might use for sending and
receiving.
% We must therefore allow ourselves to observe more than
% just internal transitions, and here we choose to define the observables
% of a process as the names it can use for sending and receiving.
To this end, we define the \emph{observability predicate}
\( \observable{P}{\mu} \) as follows:
\begin{align*}
  \observable{P}{\obsin{x}}  &\quad \textrm{if \( P \) can perform an input action via \( x \).} \\
  \observable{P}{\obsout{x}} &\quad \textrm{if \( P \) can perform an output action via \( x \).}
\end{align*}

\emph{Strong barbed bisimilarity}, written \( \sbbisim{}{} \), is the largest symmetric relation such that, whenever \( \sbbisim{P}{Q} \):
\begin{gather*}
  \observable{P}{\mu}~\textrm{implies}~\observable{Q}{\mu} \\%\label{eq:bisim1} \\
  \transition{P}{\Atau}{P'}~\textrm{implies}~\transition{Q}{\Atau}{\sbbisim{}{} P'} %\label{eq:bisim2}
\end{gather*}
We say that a relation is a \emph{strong barbed bisimulation} if it satisfies the conditions given above, but is not necessarily the largest such relation, and that \( P \) and \( Q \) are \emph{strong barbed bisimilar} if \( \sbbisim{P}{Q} \).
Note that, since our systems have potentially infinite behaviors, and strong barbed bisimilarity is defined as the largest relation that satisfies the conditions, bisimulation cannot be defined inductively.

\begin{theorem}
  \( \sbbisim{}{} \) is an equivalence relation. %, that is, the relation is reflexive, symmetric, and transitive.
\end{theorem}

Unfortunately, strong barbed bisimilarity is not a good process
equivalence since it is not a congruence, hence it does not allow for
substituting a process with an equivalent one in any context.
% equivalent processes to be since it does not consider the
% environment of processes.
For instance, the processes $\Pout{x}{a}{\Pout{y}{b}{\Pend}}$ and
$\Pout{x}{a}{\Pend}$ are strong barbed bisimilar, i.e.,
$\sbbisim{\Pout{x}{a}{\Pout{y}{b}{\Pend}}}{\Pout{x}{a}{\Pend}}$. This
is because \( \obsout{x} \) is the only observable in both processes
and they cannot perform a \( \Atau \)-action. However, in the context
$C = \Ppar{\ctxhole}{\Pin{x}{l}{\Pend}}$, the relation no
longer holds: in fact,
$\nsbbisim{\Ppar{\Pout{x}{a}{\Pout{y}{b}{\Pend}}}{\Pin{x}{l}{\Pend}}}{\Pout{x}{a}{\Pend}\Ppar{}{\Pin{x}{l}{\Pend}}}$
%
because the left process can perform a \( \Atau \)-action such that
\( \obsout{y} \) becomes observable, whereas the right process cannot.

% For instance,
% \( \sbbisim{\Pout{x}{a}{\Pout{y}{b}{\Pend}}}{\Pout{x}{a}{\Pend}} \)
% since \( \obsout{x} \) is the only observable in both processes and
% they cannot perform a \( \Atau \)-action, but
% \(
% \nsbbisim{\Ppar{\Pout{x}{a}{\Pout{y}{b}{\Pend}}}{\Pin{x}{l}{\Pend}}}{\Pout{x}{a}{\Pend}}
% \) since the left process can perform a \( \Atau \)-action such that
% \( \obsout{y} \) becomes observable, whereas the right process cannot.

% \subsubsection{Contexts and congruences}
% Before we can fix the issue with strong barbed bisimilarity, we need few more definitions.

% A \emph{context} is obtained by taking a process and replacing a single occurrence of \( \Pend \) in it with the special \emph{hole} symbol \( \ctxhole \).
% As a convention, we do \emph{not} identify \( \alpha \)-convertible contexts.

% We can think of contexts as functions between processes.
% A context \( C \) can be \emph{applied} to a process \( P \), written \( \applyctx{C}{P} \), by replacing the hole in C by \( P \), thus obtaining another process.
% The replacement should be literal, so names and variables that are free in \( P \) can become bound in \( \applyctx{C}{P} \).

% We say that an equivalence relation \( \mathcal{S} \) is a \emph{congruence} if \( (P,Q) \in \mathcal{S} \) implies that for any context \( C \), \( (\applyctx{C}{P}, \applyctx{C}{Q}) \in \mathcal{S} \).

\subsubsection{Strong barbed congruence.}
In order to detect cases like the one above, we need to restrict strong barbed bisimilarity so that it becomes a congruence,
i.e., % we need to make strong barbed bisimilarity consider
we have to consider the environment in which processes may be placed.
% (i.e.\ the context) of processes.

% We define \emph{strong barbed congruence}, written \( \sbcong{}{} \), by saying
We say that two processes \( P \) and \( Q \) are \emph{strong barbed congruent}, written \( \sbcong{P}{Q} \), if \( \sbbisim{\applyctx{C}{P}}{\applyctx{C}{Q}} \) for every context \( C \).

\begin{lemma}
  \( \sbcong{}{} \) is the largest congruence included in
  \( \sbbisim{}{} \).
\end{lemma}

\subsubsection{Challenge.}
The objective of this challenge is to prove a theorem that shows that
making strong barbed bisimilarity sensitive to substitution and
parallel composition is enough to show strong barbed
congruence. Formally,
\begin{theorem}
  \( \sbcong{P}{Q} \) if for any process \( R \) and
  substitution \( \sigma \),
  \(
  \sbbisim{\Ppar{\applysubst{\sigma}{P}}{R}}{\Ppar{\applysubst{\sigma}{Q}}{R}}
  \).
\end{theorem}

\section{Design of the benchmark}\label{sec:design-discussion}

In this section, we outline the factors considered in designing the
benchmark challenges. We start with some general remarks, then
describe the individual design considerations for each challenge problem,
and the evaluation criteria.
% We have identified three key aspects fundamental to mechanisations of concurrent calculi and designed our challenge problems to exercise these.

\subsubsection{General design remarks.}
Like the previous POPLMark and POPLMark Reloaded challenges, our challenge is not meant to be comprehensive:
applications such as multiparty session types~\cite{10.1145/2827695,10.1145/3290343}, choreographies~\cite{DBLP:journals/jar/CruzFilipeMP23}, conversation types~\cite{DBLP:journals/tcs/CairesV10}, psi-calculi~\cite{lmcs:696}, or encodings between different calculi~\cite{DBLP:journals/iandc/Gorla10,DBLP:conf/forte/CairesP16,DBLP:journals/iandc/DardhaGS17,DBLP:conf/ecoop/ScalasDHY17,DBLP:journals/iandc/KouzapasPY19,10.1145/3479394.3479407} are not directly covered.  Still, these (and other) proof mechanisation applications need the basic techniques that our challenge problems exercise:
as per our design goal \ref{item:goal-reusability},
our problems are drawn from the basic metatheory of process calculi (without requiring the development of new theory), and focus on well-known theorems
and results that involve interesting proof techniques that may be reused in further work.
To achieve our design goal \ref{item:goal-comperison-accessibility},
we have drafted our challenge problems to exercise the three aspects (linearity, scope extrusion, coinduction) independently, so that they can be solved individually and in any order;
each problem should be reasonably small and easy to understand with basic knowledge of textbook concurrency theory, process calculi, and type theory.  The process calculus used in each challenge
focuses on the features that we want to exercise, and omits all constructs
(such as choices) that would complicate the mechanisation without bringing tangible
benefits and insights.\footnote{It is of course possible to reintroduce the omitted
constructs, but we leave this as a possibility for a future benchmark challenge.}
The minimality and uniformity of the calculi also allows us to target the design
goal \ref{item:goal-tutorials}, i.e.~encouraging the creation of guidelines and
tutorials for the mechanisation of concurrency theory.

% While our challenges require reasoning about binders, which was the main topic of the original POPLMark, they also require other techniques.
% Our challenges also require reasoning about logical relations which goes above the reasoning involved in POPLMark Reloaded.

% Our challenge problems do not include reasoning about encodings. While
% this kind of reasoning is common in the field, it does not require any
% different reasoning techniques in a concurrency setting than in any
% other setting.

\subsubsection{Linearity.}
For our challenge on linear reasoning we have chosen a type preservation theorem
for a session type system since this allows us to involve linear
reasoning without many definitions. Accordingly, we have chosen not to
include channel delegation and to use a reduction instead of a
transition system semantics.
Linearity is of course interesting in many other subfields of programming languages, but we have chosen to base our challenge on a session type system primarily due to the common use of session types in concurrency theory.
Inspired by Vasconcelos~\cite{Vasconcelos2012}, we use a syntax where
restriction binds two names. This presentation makes the duality
of the two names obvious in the type system.
We have chosen a simple notion of well-formedness for our challenge,
and do not for instance consider deadlocked processes. While more
sophisticated notions of well-formedness are also interesting, the
proofs of these properties are more complicated and would thus make
the challenge more difficult and less focused on the linear
aspects.

\subsubsection{Scope extrusion.}
This is the challenge most closely related to the POPLMark
challenge since it concerns the properties of binders in restrictions.
It is not clear that the techniques used to solve the original challenges about binders can be easily extended to reasoning about scope extrusion.
Notably, techniques like Higher Order Abstract Syntax, or well-scoped
De\ Bruijn indices do not have the concept of free variables. It is
not immediately obvious how to encode the operational semantics in
this kind of system.


\subsubsection{Coinduction.}
%\fxerror{
% ``Proper'' barbs with a structural congruence in coinduction instead of LTS semantics?
%}

Our coinduction challenge concerns strong barbed bisimulation and congruence.
Though weak barbed congruence is a more common behavioural equivalence for \(\pi\)-terms, here we prefer strong equivalences to avoid the need to abstract over the number of internal transitions, % when considering the strong notions of equivalence,
thus simplifying the theory.
We have chosen not to include delegation in the coinduction challenge since it is orthogonal to our primary aim of exploring coinductive proof techniques.

\subsubsection{Evaluation criteria.}
The idea of our benchmark is to obtain evidence towards answering questions \cref{item:rq1,item:rq2,item:rq3}: therefore, we are interested in not only the solutions, but in the experience of solving the challenges with the chosen approach.
Solutions to our challenges should be compared on three measures:
\begin{itemize}
\item the mechanisation overhead, i.e.~the amount of manually written infrastructure and setup needed to express the definitions in the mechanisation; \item the adequacy of the formal statements in the mechanisation, i.e.~whether the proven theorems are easily recognisable as the theorems from the challenge; \item the cost of entry for the tools and techniques employed, i.e.~the difficulty of learning to use the techniques.
\end{itemize}
Solutions to our challenges do not need to strictly follow the definitions and lemmas set out in the problem text, but solutions which deviate from the original challenges need to present more elaborate argumentation for their adequacy.

\section{Going beyond the challenge problems}\label{sec:going-beyond}

As with the POPLMark and POPLMark Reloaded challegnes, this challenge does not aim to cover every possible
problem in the area. Our benchmark can be extended in two
dimensions: further expanding and combining the current challenges,
and addressing features not covered by the current challenges.

\vspace{-1mm}%
\paragraph{Extending and combining the current challenges.}
The proposed challenges try to minimise the overlap of the techniques
and aspects they exercise (linearity, scope extrusion, coinduction) --- but most
research on concurrency theory combines some or all such aspects. A second round of formalisations could
entail combining the features of the challenges.
In \cref{tab:extensions} we consider the following three extensions:
\emph{(1)} the addition of choice to the type level and the process level,
\emph{(2)} the addition of recursion (and recursive types for typed systems),
and \emph{(3)} the
addition of channel delegation to those calculi that do not already
support it.

\begin{table}[btp]
  \caption{Possible extensions of the challenges. Each suitable extension is marked ``+''. \label{tab:extensions}}
  \begin{center}\small
    \vspace{-4mm}%
    \begin{tabular}{|r|c|c|c|}
      \hline
      & Linearity & Scope extrusion & Coinduction \\
      \hline
      Choice & $+$ & $+$  & $+$ \\
      \hline
      Recursive types & $+$  & $+$ & \\
      \hline
      Delegation & $+$ & & $+$ \\
      \hline
    \end{tabular}
    \vspace{-3mm}%
  \end{center}
\end{table}

\vspace{-1mm}%
\paragraph{Addressing new calculi and new features.}
Extending in this dimension
means proposing new challenges covering different features of
process calculi, or different proof mechanisation features.
Some interesting aspects of message passing calculi to explore in
further challenges could be multiparty session types and
choreographies, as their meta-theory includes aspects
that we do not cover here (e.g.: well-formed global types,
projection as a partial function) --- although, dually, some aspects
that we cover here do appear in their theory. Also, one could easily
design a challenge that goes beyond our coinduction challenge, by
exploring different notions of bisimilarity (barbed, weak, etc.),
and trace equivalence.

On the side of proof assistants, the current challenge proposes mechanising the
proofs of theorems. An interesting venue to explore would be to take advantage
of other proof assistant features. For example, proof assistants can often
produce certified code (via code extraction, or compiling and running their
definitions). Additionally, while our current challenge is agnostic about proof
automation, a more specific challenge could be designed with the objective of
automating aspects of the proofs. Finally, a challenge could propose the
integration with other formal reasoning tools, namely model checkers. These
tools are in common use in the field --- and combining automated proofs with
proof assistants offers the potential to ease the path towards larger proofs.

Ultimately, the current challenges can be extended in several
worthwhile directions, and we look forward to a future when they
indeed are extended. We see the current challenge as setting
the foundation for those future extensions. It is our expectation that
the solutions of those future challenges will build on solutions
to this challenge.

\bibliographystyle{splncs04}
\bibliography{../references}

\appendix
\section{Challenges}\label{app:challenges}
\documentclass[a4paper]{article}

\usepackage{amsmath}
\usepackage{amssymb}
\usepackage{amsthm}
\usepackage[utf8]{inputenc}
\usepackage{mathpartir}
\usepackage{bm}
\usepackage{graphicx}

\newtheorem{lemma}{Lemma}
\newtheorem{theorem}{Theorem}

\newcommand{\Pend}{\bm{0}}
\newcommand{\Ppar}[2]{#1 \mid #2}
\newcommand{\Pres}[2]{(\bm{\nu} #1)~#2}
\newcommand{\Pout}[3]{#1 ! #2 . #3}
\newcommand{\Pin}[3]{#1 ? (#2) . #3}
\newcommand{\Pchoice}[2]{#1 + #2}
\newcommand{\Preplicate}[1]{{!}#1}

\newcommand{\freenames}[1]{\textrm{fn}(#1)}
\newcommand{\boundnames}[1]{\textrm{bn}(#1)}
\newcommand{\names}[1]{\textrm{n}(#1)}

\newcommand{\alphacon}[2]{#1 =_{\alpha} #2}
\newcommand{\subst}[3]{#1\{#2/#3\}}

\newcommand{\Aoutf}[2]{#1 ! #2}
\newcommand{\Aoutb}[2]{#1 ! (#2)}
\newcommand{\Ain}[2]{#1 ? #2}
\newcommand{\Atau}{\tau}

\newcommand{\transition}[3]{#1 \xrightarrow{#2} #3}

\newcommand{\observable}[2]{#1\downarrow_{#2}}
\newcommand{\obsin}[1]{#1?}
\newcommand{\obsout}[1]{#1!}

\newcommand\sbullet[1][.5]{\mathbin{\hbox{\scalebox{#1}{$\bullet$}}}}
\newcommand{\sbbisim}[2]{#1 \overset{\sbullet}{\sim} #2}

\begin{document}

\section{Challenge: Scope Extrusion}
The idea of this challenge is to formalize a proof that requires scope extrusion.
Scope extrusion is the notion that a process can send restricted names to another process, as long as the restriction can safely be ``extruded'' (i.e.\ expanded) to include the recieving process.
This e.g.\ allows a process to set up a private connection by sending a restricted name to another process, then using this name for further communication.

The setting for this challenge is an untyped \( \pi \)-calculus, the syntax and semantics of which are presented below.
The objective of this challenge is to prove that barbed bisimulation for this calculus is an equivalence relation.

\subsection{Syntax}
We assume the existence of some type of \emph{names}, values of which we will denote by \( x, y, \dots \).
The syntax is then:
\begin{align*}
  P,Q :=&&& \Pend \\
  |&&& \Pout{x}{y}{P} \\
  |&&& \Pin{x}{y}{P} \\
  |&&& \Ppar{P}{Q} \\
  |&&& \Pres{x}{P} \\
  |&&& \Pchoice{P}{Q}
\end{align*}
The process \( \Pend \) is \emph{inaction}: a process which can do nothing.
The process \( \Pout{x}{y}{P} \) is an \emph{output}, which can send the name \( y \) via \( x \), then continue as \( P \).
The process \( \Pin{x}{y}{P} \) is an \emph{input}, which can receive a name via \( x \), then continue as \( P \) with the received name substituted for \( y \).
The input operator thus binds the name \( y \) in \( P \).
The process \( \Ppar{P}{Q} \) is the \emph{composition} of process \( P \) and process \( Q \).
The two components can proceed independently of each other, or they can interact via shared names.
The process \( \Pres{x}{P} \) is the \emph{restriction} of the name \( x \) to \( P \).
Components in \( P \) can use the name \( x \) to interact with each other, but not with processes outside of the restriction.
The restriction operator thus binds the name \( x \) in \( P \).
Note that the scope of a restriction may change when processes interact.
The process \( \Pchoice{P}{Q} \) is a non-deterministic \emph{choice} between continuing as the process \( P \) or as the process \( Q \).
Note that there is no recursion or replication in the syntax, and thus no infinite behaviours can be expressed.

We will use the notation \( \freenames{P} \) to denote the set of names that occur free (i.e.\ not bound by a restriction or an input) in \( P \).
We will use the notation \( \boundnames{P} \) to denote the set of names that occur bound (by a restriction or an input) in \( P \).
We will use the notation \( \subst{P}{x}{y} \) to denote the process \( P \) with \( x \) substituted for \( y \).

Two processes \( P \) and \( Q \) are \( \alpha \)-convertible, written \( \alphacon{P}{Q} \), if \( Q \) can be obtained from \( P \) by a finite number of substitutions of bound names.
As a convention, we will identify \( \alpha \)-convertible processes.

As a convention, we assume that the bound names occurring in any collection of processes are chosen to be different from the free names occurring in those processes and from the names occurring in any substitutions applied to the processes.
This is justified because any overlapping names may be \( \alpha \)-converted such that the assumption is satisfied.

\subsection{Semantics}
The semantics of the system describe the actions that the system can perform by defining a labelled transition relation on processes.
The transitions are labelled by \emph{actions}, the syntax of which are as follows:
\begin{align*}
  \alpha := &&& \Aoutf{x}{y} \\
  |&&& \Ain{x}{y} \\
  |&&& \Aoutb{x}{y} \\
  |&&& \Atau
\end{align*}
The \emph{free output action} \( \Aoutf{x}{y} \) is sending the name \( y \) via \( x \).
The \emph{input action} \( \Ain{x}{y} \) is receiving the name \( y \) via \( x \).
The \emph{bound output action} \( \Aoutb{x}{y} \) is sending a fresh name \( y \) via \( x \).
The \emph{internal action} \( \Atau \) is performing some unobservable action, e.g.\ internal communication.

We will again use the notation \( \freenames{\alpha} \) to denote the set of names that occur free in the action \( \alpha \) and the notation \( \boundnames{\alpha} \) to denote the set of names that occur bound in the action \( \alpha \).
In the free output action \( \Aoutf{x}{y} \) and the input action \( \Ain{x}{y} \), both \( x \) and \( y \) are free names.
In the bound output action \( \Aoutb{x}{y} \), \( x \) is a free name, while \( y \) is a bound name.
We will further use the notation \( \names{\alpha} \) to denote the union of \( \freenames{\alpha} \) and \( \boundnames{\alpha} \), i.e.\ the set of all names that occur in the action \( \alpha \).

The transition relation is then defined by the following rules:
\begin{mathpar}
  \inferrule[Out]{ }{\transition{\Pout{x}{y}{P}}{\Aoutf{x}{y}}{P}}
  \and
  \inferrule[In]{ }{\transition{\Pin{x}{z}{P}}{\Ain{x}{y}}{\subst{P}{y}{z}}}
  \and
  \inferrule[Sum-L]{\transition{P}{\alpha}{P'}}{\transition{\Pchoice{P}{Q}}{\alpha}{P'}}
  \and
  \inferrule[Sum-R]{\transition{P}{\alpha}{P'}}{\transition{\Pchoice{Q}{P}}{\alpha}{P'}}
  \and
  \inferrule[Par-L]{\transition{P}{\alpha}{P'} \\ \boundnames{\alpha} \cap \freenames{Q} = \emptyset}{\transition{\Ppar{P}{Q}}{\alpha}{\Ppar{P'}{Q}}}
  \and
  \inferrule[Par-R]{\transition{Q}{\alpha}{Q'} \\ \boundnames{\alpha} \cap \freenames{P} = \emptyset}{\transition{\Ppar{P}{Q}}{\alpha}{\Ppar{P}{Q'}}}
  \and
  \inferrule[Comm-L]{\transition{P}{\Aoutf{x}{y}}{P'} \\ \transition{Q}{\Ain{x}{y}}{Q'}}{\transition{\Ppar{P}{Q}}{\Atau}{\Ppar{P'}{Q'}}}
  \and
  \inferrule[Comm-R]{\transition{P}{\Ain{x}{y}}{P'} \\ \transition{Q}{\Aoutf{x}{y}}{Q'}}{\transition{\Ppar{P}{Q}}{\Atau}{\Ppar{P'}{Q'}}}
  \and
  \inferrule[Close-L]{\transition{P}{\Aoutb{x}{z}}{P'} \\ \transition{Q}{\Ain{x}{z}}{Q'} \\ z \notin \freenames{Q}}{\transition{\Ppar{P}{Q}}{\tau}{\Pres{z}{\Ppar{P'}{Q'}}}}
  \and
  \inferrule[Open]{\transition{P}{\Aoutf{x}{z}}{P'} \\ z \neq x}{\transition{\Pres{z}{P}}{\Aoutb{x}{z}}{P'}}
  \and
  \inferrule[Close-R]{\transition{P}{\Ain{x}{z}}{P'} \\ \transition{Q}{\Aoutb{x}{z}}{Q'} \\ z \notin \freenames{P}}{\transition{\Ppar{P}{Q}}{\tau}{\Pres{z}{\Ppar{P'}{Q'}}}}
  \and
  \inferrule[Res]{\transition{P}{\alpha}{P'} \\ z \notin \names{\alpha}}{\transition{\Pres{z}{P}}{\alpha}{\Pres{z}{P'}}}
\end{mathpar}
Note that there is no rule for inferring transitions from \( \Pend \), and that there is no rule for inferring an action of an input or output process except those that match the input/output capability.
Note also that due to rule \TirName{In}, the process \( \Pin{x}{z}{P} \) can receive \emph{any} name.

As a convention, we assume that bound names of any processes or actions are chosen to be different from the names that occur free in any other entities under consideration, such as processes, actions, substitutions, and sets of names.
The convention has one exception, namely that in the transition \( \transition{P}{\Aoutb{x}{z}}{Q} \), the name \( z \) (which occurs bound in \( P \) and the action \( \Aoutb{x}{z} \)) may occur free in \( Q \).
Without this exception it would be impossible to express scope extrusion.

\subsection{Bisimilarity}
Our notion of process equivalence relations builds on a notion of \emph{observables}.
If we allow ourselves only to observe internal transitions, we will relate either too few processes (in the strong case where we relate only processes with exactly the same number of internal transitions) or every process (in the weak case where we relate processes with any amount of internal transitions).
We must therefore allow ourselves to observe more, and here choose to define the observables of a process as the names it can use for sending and receiving.

To this end, we define the \emph{observability predicate} \( \observable{P}{\mu} \) by the following rules:
\begin{align}
  \observable{P}{\obsin{x}}  &\quad \textrm{if \( P \) can perform an input action via \( x \).} \\
  \observable{P}{\obsout{x}} &\quad \textrm{if \( P \) can perform an output action via \( x \).}
\end{align}

Two processes \( P \) and \( Q \) are then \emph{strongly barbed bisimilar}, written \( \sbbisim{P}{Q} \), if:
\begin{gather}
  \observable{P}{\mu}~\textrm{implies}~\observable{Q}{\mu} \\
  \transition{P}{\Atau}{P'}~\textrm{implies}~\transition{Q}{\Atau}{Q'}~\textrm{for some \( Q' \) with}~\sbbisim{P'}{Q'} \\
  \observable{Q}{\mu}~\textrm{implies}~\observable{P}{\mu} \\
  \transition{Q}{\Atau}{Q'}~\textrm{implies}~\transition{P}{\Atau}{P'}~\textrm{for some \( P' \) with}~\sbbisim{P'}{Q'}
\end{gather}
Note that, since our calculus does not have infinite behaviour, we do not specify that the relation is the largest of its kind, and the relation is thus not coinductively defined.

\subsection{Challenge}
The objective of this challenge is to prove the following theorem:
\begin{theorem}
  \( \sbbisim{}{} \) is an equivalence relation, that is, the relation is reflexive, symmetric, and transitive.
\end{theorem}

\section{Challenge: Coinduction}
The idea of this challenge is to formalize a proof that requires coinductive techniques.
Coinduction is a proof technique for infinite structures, which arise in this context due to systems with behaviours that continue indefinitely.
Coinduction is the dual of induction: whereas induction is useful for proving properties of least fixed points, coinduction is useful for proving properties of greatest fixed points.

The setting for this challenge is an untyped calculus of communicating systems with replication of processes, the syntax and semantics of which are presented below.
The objective of this challenge is to prove that strong barbed bisimulation for this calculus is an equivalence relation.

\section{Challenge: Linearity}
The setting for this challenge is a restricted \( \pi \)-calculus with a session type system.

The objective of this challenge is to prove type preservation.

\section{Bonus challenges}

\subsection{Linearity and Scope Extrusion}
The setting for this challenge is the \( \pi \)-calculus with a session type system.

The objective of this challenge is to prove type preservation.

\subsection{Scope Extrusion and Coinduction}
The setting for this challenge is a simply typed \( \pi \)-calculus.

The objective of this challenge is to prove bisimulation.

\subsection{Coinduction and Linearity}
The setting for this challenge is the Calculus of Communicating Systems with a session type system.

The objective of this challenge is to prove type preservation.

\section{Future work}
\begin{itemize}
\item Multiparty session types
\item Choreography
\item Encodings
\item Code extraction
\end{itemize}

\end{document}

\end{document}
